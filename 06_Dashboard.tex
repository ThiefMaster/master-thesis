\chapter{Dashboard mit relevanten Vorschlägen}

Im Anschluss an die Migration soll das Benutzerdashboard von Indico noch um für den Benutzer
relevante Vorschläge erweitert werden.


\section{Aktueller und gewünschter Zustand}

Derzeit zeigt das Dashboard aktuelle Events an, mit denen der User verbunden ist, d.h. an denen er
teilnimmt, die er organisiert oder an deren \emph{Paper-Reviewing}-Prozess er beteiligt ist. Neben
diesen Events sieht er eine Liste aller Kategorien, in denen er Manager-Rechte hat oder die auf
seiner Favoritenliste sind, und welche Events aktuell in diesen Kategorien stattfinden. Bei dem
gesamten Dashboard handelt es sich um ein noch sehr neues Feature, welches aber bereits großen
Zuspruch seitens der Indico-Benutzer findet.

Zusätzlich zu diesen Informationen wäre es hilfreich, wenn dem Benutzer sowohl potenziell
interessante Kategorien als auch weitere Events vorgeschlagen werden könnten. Benutzer sind solche
Vorschläge bereits aus größeren Onlineshops gewohnt und gerade in einer großen Indico-Installation
mit sehr vielen Events und Kategorien ist die Chance groß, dass solche Vorschläge auch wirklich
hilfreich sind und nicht nur auf Events verweisen, die der User bereits kennt.


\section{Kategorien}

Alle Events in Indico sind in genau einer Kategorie, die wiederum Teil einer Baumstruktur ist. Dabei
werden auf der Indico-Startseite die in der Rootkategorie enthaltenen Kategorien angezeigt und in
jeder Kategorie werden entweder die enthaltenen Events oder Unterkategorien aufgelistet. Um
Kategorien sinnvoll vorschlagen zu können, muss jeder Kategorie ein Score zugewiesen werden, die die
Relevanz dieser Kategorie für den jeweiligen Benutzer angibt. Wenn dieser Score entsprechend hoch
ist, wird die Kategorie im Dashboard als Vorschlag angezeigt.

Da Kategorien in Indico sehr unterschiedliche Events enthalten können, macht es wenig Sinn, komplett
neue Kategorien vorzuschlagen. Stattdessen bieten sich Kategorien an, mit denen der Benutzer bereits
in Kontakt gekommen ist, weil er an einem darin enthaltenen Event teilgenommen hat. Dies alleine ist
jedoch nicht ausreichend, um eine Kategorie vorzuschlagen - gerade bei Benutzern, die Indico schon
länger verwenden, ist die Chance groß, dass sie in einigen Kategorien an einem Event teilgenommen
haben und seitdem nichts mehr mit dieser Kategorie zu tun hatten. Solch eine Kategorie vorzuschlagen
würde also letzendlich nur dazu führen, dass der Benutzer sich über die schlechten Vorschläge ärgert
und sie daraufhin grundsätzlich nicht mehr beachtet. Ein gutes Relevanzkriterium ist also, an
wievielen aktuellen Events in der jeweiligen Kategorie der User teilgenommen hat. Da dies jedoch
durch längere Pausen ohne Teilnahme negativ beeinflusst wird, selbst wenn im aktuellsten Block 100\%
Teilnahme herrscht, sollten dabei alle älteren Blöcke verworfen werden, sodass nur das erste Event
im aktuellsten Block als frühstes Teilnahmedatum zählt. Um Kategorien, in denen der User regelmäßig
an Event teilgenommen hat, dies nun aber nicht mehr tut, auszufiltern, kann der Score abhängig von
der Anzahl Events, die nach der letzten Teilnahme stattfinden, verringert oder erhöht werden.

Damit zeigen sich bereits relativ gute Ergebnisse, allerdings tauchen weiterhin Kategorien auf, die
die bisherigen Kriterien erfüllen, aber nicht mehr aktiv genutzt werden und somit keine neuen Events
enthalten. Um diese nicht mehr vorzuschlagen, kann der Score abhängig davon, wie weit das neueste
Event in der Vergangenheit liegt, verringert werden - je länger das ist, desto weniger relevant ist
die Kategorie für den Benutzer.

Unabhängig von der vergangenen Aktivität des Benutzers ist eine Kategorie, an deren Events man
bereits lange im Voraus teilnimmt, ein sehr guter Kandidat. Daher erhöht jedes solche Event den
Score der Kategorie. Abhängig davon, wie weit das Event in der Zukunft liegt, wird der Score
ebenfalls erhöht - ein in Kürze stattfindendes Event ist ein gutes Indiz dafür, dass der User Indico
aktiver nutzt und evtl die Kategorie des Events seinen Favoriten hinzufügen will.

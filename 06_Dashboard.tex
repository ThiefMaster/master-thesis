\chapter{Dashboard-Erweiterung}

Im Anschluss an die Migration soll das Benutzerdashboard von Indico noch um für den Benutzer
relevante Vorschläge erweitert werden.


\section{Aktueller und gewünschter Zustand}

Derzeit zeigt das Dashboard aktuelle Events an, mit denen der User verbunden ist, d.h. an denen er
teilnimmt, die er organisiert oder an deren \emph{Paper-Reviewing}-Prozess er beteiligt ist. Neben
diesen Events sieht er eine Liste aller Kategorien, in denen er Manager-Rechte hat oder die auf
seiner Favoritenliste stehen, und welche Events aktuell in diesen Kategorien stattfinden. Bei dem
gesamten Dashboard handelt es sich um ein noch sehr neues Feature, welches aber bereits großen
Zuspruch durch die Indico-Benutzer findet.

Zusätzlich zu diesen Informationen wäre es hilfreich, wenn dem Benutzer sowohl potenziell
interessante Kategorien als auch weitere Events vorgeschlagen werden könnten. Benutzer sind solche
Vorschläge bereits aus größeren Onlineshops gewohnt und gerade in einer großen Indico-Installation
mit sehr vielen Events und Kategorien ist die Chance groß, dass solche Vorschläge auch wirklich
hilfreich sind und nicht nur auf Events verweisen, die der User bereits kennt.


\section{Kategorien}\label{categorysuggest}

Alle Events in Indico sind in genau einer Kategorie, die wiederum Teil einer Baumstruktur ist. Dabei
werden auf der Indico-Startseite die in der Rootkategorie enthaltenen Kategorien angezeigt und in
jeder Kategorie werden entweder die enthaltenen Events oder Unterkategorien aufgelistet. Um
Kategorien sinnvoll vorschlagen zu können, muss jeder Kategorie ein Score zugewiesen werden, die die
Relevanz dieser Kategorie für den jeweiligen Benutzer angibt. Wenn dieser Score entsprechend hoch
ist, wird die Kategorie im Dashboard als Vorschlag angezeigt.

Da Kategorien in Indico sehr unterschiedliche Events enthalten können, macht es wenig Sinn, komplett
neue Kategorien vorzuschlagen. Stattdessen bieten sich Kategorien an, mit denen der Benutzer bereits
in Kontakt gekommen ist, weil er an einem darin enthaltenen Event teilgenommen hat. Dies alleine ist
jedoch nicht ausreichend, um eine Kategorie vorzuschlagen. Gerade bei Benutzern, die Indico schon
länger verwenden, ist die Chance groß, dass sie in einigen Kategorien an einem Event teilgenommen
haben und seitdem nichts mehr mit dieser Kategorie zu tun hatten. Solch eine Kategorie vorzuschlagen
würde also letztendlich nur dazu führen, dass der Benutzer sich über die schlechten Vorschläge ärgert
und sie daraufhin grundsätzlich nicht mehr beachtet. Ein gutes Relevanzkriterium ist also, an wie
vielen aktuellen Events in der jeweiligen Kategorie der User bereits teilgenommen hat. Da dies jedoch
durch längere Pausen ohne Teilnahme negativ beeinflusst wird, selbst wenn im aktuellsten Block 100\%
Teilnahme herrscht, sollten dabei alle älteren Blöcke verworfen werden, sodass nur das erste Event
im aktuellsten Block als frühstes Teilnahmedatum zählt. Um Kategorien auszufiltern, in denen der
User regelmäßig an Event teilgenommen hat, dies nun aber nicht mehr tut, kann der Score abhängig von
der Anzahl Events, die nach der letzten Teilnahme stattfinden, verringert oder erhöht werden.

Damit zeigen sich bereits ziemlich gute Ergebnisse, allerdings tauchen weiterhin Kategorien auf, die
die bisherigen Kriterien erfüllen, aber nicht mehr aktiv genutzt werden und somit keine neuen Events
enthalten. Um diese nicht mehr vorzuschlagen, kann der Score abhängig davon, wie weit das neueste
Event in der Vergangenheit liegt, verringert werden. Je älter das Event ist, desto weniger relevant
ist die Kategorie für den Benutzer.

Unabhängig von der vergangenen Aktivität des Benutzers ist eine Kategorie, an deren Events man
bereits lange im Voraus teilnimmt, ein sehr guter Kandidat. Daher erhöht jedes solche Event den
Score der Kategorie. Abhängig davon, wie weit das Event in der Zukunft liegt, wird der Score
ebenfalls erhöht - ein in Kürze stattfindendes Event ist ein gutes Indiz dafür, dass der User Indico
aktiver nutzt und evtl. die Kategorie des Events seinen Favoriten hinzufügen will.


\section{Events}

Anders als bei Kategorien ist es bei Events durchaus sinnvoll, \enquote{ähnliche} Events
vorzuschlagen. Diese Ähnlichkeit kann auf verschiedene Wege bestimmt werden. Da der Use-Case,
ähnliche \enquote{Artikel} vorzuschlagen, insbesondere in Onlineshops häufig auftritt, existieren
bereits einige fertige Lösungen. Es bietet sich also an, zu überprüfen, ob eine solche Lösung für
Indico Sinn macht und somit den mit einer Eigenentwicklung verbundenen Aufwand zu vermeiden.

Da das Backend von Indico selbst komplett in Python geschrieben ist, sollte auch die Recommendation
Engine entweder auf Python basieren oder entsprechende Python-Bindings besitzen, sodass sie einfach
integriert werden kann. Ebenfalls ist es wichtig, dass sie, genau wie Indico, Open Source ist und
keine schwierig zu erfüllenden Systemanforderungen hat. Ein Beispiel dafür wäre eine Software, die
zwingend eine kommerzielle Oracle-Datenbank benötigt.


\subsection{Crab}

\emph{Crab}\footnote{\href{https://github.com/muricoca/crab/}{https://github.com/muricoca/crab/}}
ist ein \emph{Recommender Framework}, welches auf den Libraries \emph{numpy} und \emph{scipy}
basiert. Dabei handelt es sich um Python-Bibliotheken, die diverse mathematische Standardalgorithmen
und Hilfsfunktionen zur Verfügung stellen und dank in C geschriebenen Erweiterungen auch in einer
Scriptsprache wie Python sehr hohe Performance bieten.

Auf die verwendeten Algorithmen näher einzugehen würde den Umfang dieses Kapitels sprengen. Daher
wird nur die eigentliche Funktionalität von Crab betrachtet und inwiefern sie für Indico geeignet
ist. Als Datenbasis wird ein Mapping benötigt, welches Benutzer mit Events verknüpft, wobei jede
Verknüpfung ein bestimmtes Gewicht hat. Anhand dieser Daten berechnet Crab nun die Ähnlichkeit
zwischen den einzelnen Usern. Der Recommender kann schlussendlich mithilfe dieser Daten und
den Initialdaten Events vorschlagen, mit denen ein ähnlicher Benutzer eine Verbindung hat.

Die API von Crab ist sehr einfach zu nutzen, allerdings spricht ein wichtiger Faktor gegen eine
Verwendung in Indico. Unabhängig von den genutzten Algorithmen benötigt Crab alle Daten gleichzeitig
im Speicher.  Selbst wenn man ältere Events nicht mit einbeziehen würde, wären dennoch Daten für
sehr viele Benutzer zu laden. Mit der CERN-Datenbank wären das ungefähr 50000 Benutzer und deutlich
mehr Events. Da Crab keine temporären Dateien anlegt, müssen jegliche Berechnung bei weiteren
Durchläufen erneut ausgeführt werden und somit genauso rechen- und speicherintensiv.


\subsection{Probleme}

Wenn man einmal näher betrachtet, wie größere Indico-Installationen meist genutzt werden und welche
Events relevant sind, bemerkt man schnell, dass sich das Problem stark von dem zuvor erwähnten
Use-Case in einem Onlineshop unterscheidet:

\begin{itemize}

\item In einem Shop kann ein Benutzer oftmals Artikel bewerten oder entscheidet sich dafür, einen
Artikel zu kaufen oder ihn in den Warenkorb zu legen bzw. vor dem Kauf wieder daraus zu entfernen.
Mit diesen Informationen lässt sich verhältnismäßig einfach eine Gewichtung zwischen Kunden und
Artikeln aufbauen. In Indico ist dies jedoch nicht der Fall - bei Events vom Typ \enquote{Lecture}
oder \enquote{Meeting} wird häufig nicht einmal eine Teilnehmerliste geführt, weshalb nur aktiv
teilnehmende Benutzer, die beispielsweise durch einen im Zeitplan festgehaltenen Beitrag eine
sichtbare Verknüpfung mit dem Event haben.

\item Prinzipiell sind zunächst alle Artikel in einem Shop für einen Benutzer relevant bzw könnten es
sein. Eine große Indico-Installation enthält jedoch sehr viele unterschiedliche Events. Im Beispiel
der CERN-Installation geht das von Human-Resources-Meetings über Meetings der diversen
Physik-Experimente bis hin zu Meetings der Systemadministratoren. Obwohl es durchaus möglich sein
kann, dass jemand dort an mehreren Meetings teilnimmt, weil er z.B. Systemadministrator bei einem
der Experimente ist, führt dies nicht zwangsläufig dazu, dass Events aus beiden Kategorien relevant
für seine Kollegen sind.

\end{itemize}

An diesen Punkten sieht man deutlich, dass eine fertige Recommendation Engine, die mit größter
Wahrscheinlichkeit auf den Onlineshop-Use-Case zugeschnitten ist, in Indico nicht wirklich hilfreich
ist.


\subsection{Alternative}

Deutlich sinnvoller sind Vorschläge zu Konferenzen. Diese finden in der Regel höchstens jährlich
statt und dementsprechend nützlich ist ein Hinweis, dass eine bestimmten Konferenz, an der man zuvor
bereits einmal teilgenommen hat, wieder stattfindet. Oftmals haben solche jährlich stattfindenden
Konferenzen jeweils denselben Titel mit der entsprechenden Jahreszahl am Ende.

Technisch lassen sich auf ähnlichen Titeln basierende Vorschläge sehr einfach realisieren. Indico
enthält bereits Plugins, um alle Events einer externen Suchmaschine zugänglich zu machen. Wenn man
nun also automatisiert nach Events mit Titeln sucht, die Ähnlichkeit mit den Titeln bereits
besuchter Events bzw. Konferenzen haben und noch nicht stattgefunden haben, ist die Chance groß,
dass es sich um genau die gewünschten Vorschläge handelt. Die einzige Schwierigkeit liegt daran, die
Titel der bestehenden Events auszulesen ohne alle mit dem Benutzer verbundenen Events einbeziehen zu
müssen. Während dies zwar problemlos möglich ist, wenn man Meetings grundsätzlich nicht einbezieht,
hat diese Lösung den Nachteil, einen Großteil des Datenbestandes zu ignorieren. Einfacher ist es,
die Liste der Events des Users zu filtern und ähnliche Events zusammenzufassen bzw. Duplikate zu
entfernen. Insbesondere regelmäßige Meetings haben meist denselben Namen, da sie einfach nur kopiert
werden, und sofern Jahreszahlen oder andere Datumsangaben im Titel sind, kann man sie ebenfalls
leicht entfernen. Damit hat man eine Liste mit einzigartigen Eventtiteln, die man an die
Suchmaschine übergeben kann.

Die Ergebnisse dieser Suche sind jedoch noch nicht sofort für Vorschläge geeignet. Jede Fuzzy-Suche
enthält Ergebnisse, die man nicht erwarten würde, da kein Suchalgorithmus unfehlbar ist. Während an
sich einzelne unpassende Vorschläge kein Problem sind, dürfen grundsätzlich keine Events angezeigt
werde, auf die der Benutzer keinen Zugriff hat - je nach Event könnte bereits der Titel und eine
eventuell vorhandene Beschreibung sensible Informationen enthalten. Darüber hinaus hat die
Suchmaschine keine Informationen darüber, an welchen Events der Benutzer bereits teilnimmt. Diese
müssen daher ebenfalls in Indico ausgefiltert werden.

\section{Aktueller Stand}

Der derzeitig \emph{master}-Entwicklungszweig von Indico enthält noch keinerlei Code für Vorschläge,
allerdings ist das auf dem in \autoref{categorysuggest} beschriebenen Algorithmus basierte
Vorschlagssystem für Kategorien fertiggestellt und kann nach weiteren Tests in \emph{master}
übernommen werden. Die Implementation ist dabei darauf ausgelegt, auch mit sehr vielen Benutzern
performant zu arbeiten.  Dies wurde dadurch realisiert, dass nur Benutzer, die das Dashboard auch
nutzen, Vorschläge erhalten. Darüber hinaus werden die Vorschläge nachts in einem separaten Prozess
generiert, sodass sie nicht während der Hauptarbeitszeit tagsüber die Datenbank belasten.

Für Event-Vorschläge existiert ein einfacher Prototyp, der aktuelle Events in den favorisierten
Kategorien des Benutzers durchsucht und Events vorschlägt, deren Teilnehmerlisten zu mindestens 25\%
übereinstimmen oder denselben Vorsitzenden haben. Jedoch haben sich die damit gefundenen Vorschläge
als wenig hilfreich herausgestellt, da sie hauptsächlich regelmäßig stattfindende Meetings in der
Ergebnisliste hatten, die aber bereits aufgrund der favorisierten Kategorie im bestehenden Dashboard
angezeigt werden.

Manuelle Tests mit der am CERN genutzten Indico-Suchmaschine lassen jedoch darauf schließen, dass
titelbasierte Vorschläge nützlich sind und der Entwicklungsaufwand sich daher lohnt.

\chapter{Grundlagen}

\section{Python}

\subsection{Geschichte}

Python wurde 1989 von Guido van Rossum am \emph{CWI}\footnote{Centrum Wiskunde \& Informatica - das
nationale niederländische Forschungsinstitut für Mathematik und Informatik} entwickelt und 1991 als
Betaversion in der Usenet-Newsgroup \emph{alt.sources} veröffentlicht.

Seitdem wurde die Sprache kontinuierlich weiterentwickelt und 1994 in der Version 1.0
veröffentlicht. Die erste Version von Python 2.0 erschien im Herbst 2000 und legte den Grundstein
für die heute weit verbreiteten Versionen 2.6 und 2.7, wobei es sich bei 2.7 um die letzte Version
von Python 2 handelt.

Version 3 der Programmiersprache ist seit 2008 verfügbar, wobei sie in der Verbreitung bisher noch
nicht an Python 2 anknüpfen konnte. Dies liegt hauptsächlich an den teilweise nicht zu Python 2
kompatiblen Änderungen, die zu einem Henne-Ei-Problem führen: Einige populäre Python-Bibliotheken
und -Frameworks sind noch nicht mit Python 3 kompatibel, was Entwickler neuer Anwendungen oftmals
davon abhält, auf Python 3 zu setzen. Dies führt allerdings wieder zu geringer Nachfrage nach mit
Python 3 kompatiblen Bibliotheken. \citep{pywiki:2or3}

Um diesem Problem entgegenzuwirken, wurden in den Versionen 2.6 und 3.3 einige Syntaxelemente
hinzugefügt, die es Entwicklern einfacher machen, mit beiden Versionen kompatiblen Code zu
schreiben. \citep{pep:414}


\subsection{Anwendungsgebiete}

Bei Python handelt es sich um eine höhere \emph{general-purpose}-Programmiersprache mit der sich
nahezu beliebige Anwendungen einfach realisieren lassen. Es existieren diverse Frameworks sowohl für
Webanwendungen als auch für Desktopanwendungen mit grafischer Oberfläche, sodass man als Entwickler
in der Regel nur wenig nicht-anwendungsspezifischen Code schreiben muss.

Während es sich bei den Webframeworks in der Regel um reine Python-Frameworks handelt, so sind die
GUI-Frameworks meist Bindings für in C oder C++ geschriebene Frameworks wie \emph{Qt}, \emph{GTK+}
oder \emph{wxWidgets}, was natürlich den großen Vorteil hat, dass diese Frameworks bereits außerhalb
der Python-Community häufig genutzt werden und dementsprechend stabil sind.

Ein weiteres Gebiet, in dem Python häufig verwendet wird, ist das Scripting von diversen Tools.
Populäre Editoren wie \emph{Vim} können Python-Scripts zur Automatisierung von Aufgaben nutzen oder
auch um beliebige andere Aktionen im Editor auszuführen.

Wie die meisten Open-Source-Programmiersprachen ist Python insbesondere in der Linux-Welt weit
verbreitet und wird dort auch oftmals für einfache Systemtools eingesetzt, wo eine Implementierung
in C deutlich aufwändiger wäre. So ist der Paketmanager \emph{Portage} der
Linux-Distribution \emph{Gentoo} fast vollständig in Python geschrieben; nur einige Bibliotheken
sind in C implementiert, da C-Code deutlich performanter ist als interpretierter Code einer
Scriptsprache.


\subsection{Erweiterbarkeit}

Anders als PHP, wo Erweiterungen ausschließlich in C geschrieben werden können,
bietet Python verschiedene Möglichkeiten, mit nicht-Python-Code zu interagieren. Natürlich können
native Erweiterungen für Python in C oder einer anderen Sprache, die zu einer \emph{Shared Library}
kompilierbar ist, implementiert werden und haben dadurch vollen Zugriff auf alle Lowlevel-APIs
von Python. Dies ist allerdings oftmals nicht notwendig und führt nur zu unnötig komplexem Code, da
die C-APIs auf einer sehr niedrigen Ebene arbeiten und auch die \emph{Reference Counts}
des Garbage Collectors vom Entwickler penibel aktualisiert werden müssen, um Speicherlecks zu
vermeiden.

Ein häufiger Grund für native Erweiterungen ist das Zugänglichmachen von existierenden Bibliotheken,
die in C geschrieben sind. Auf den ersten Blick würde man vermuten, dass es in diesem Fall sinnvoll
ist, eine native Erweiterung zu schreiben. Wenn man bedenkt, was das Ziel eines solchen Interfaces
ist, merkt man jedoch schnell, dass meist weder Lowlevel-APIs noch die erhöhte Performance von
C-Code notwendig ist - alles was benötigt wird ist eine Möglichkeit, Funktionen aus einer
kompilierten Bibliothek aufzurufen. Dazu bietet Python mit dem in der Standardbibliothek enthaltenen
\emph{ctypes}-Paket ein einfach nutzbares FFI\footnote{Foreign Function Interface}, mit welchem ein
Entwickler in reinem Python-Code beliebige Funktionen aus \emph{Shared Libraries} aufrufen und sogar
Python-Funktionen als Callback übergeben kann.

\lstinputlisting[caption=ctypes-Beispiel]{code/ffi-ctypes.py}

Auch wenn die Python-Erweiterung performancekritische Operationen enthält und deshalb nicht in
Python implementiert werden soll, gibt es eine Möglichkeit, fast reinen Python-Code dafür zu
schreiben. \emph{Cython}\footnote{\href{http://cython.org}{http://cython.org}} ist eine mit
C-Elementen erweiterte Abwandlung von Python, die zu nativem Code kompiliert wird. Insbesondere um
Python-Klassen zu erstellen ist Cython ideal geeignet, da die Cython-Syntax dabei große Ähnlichkeit
mit der Python-Syntax hat und man nicht die unintuitive C-API nutzen muss.


\subsection{Implementationen}

Wenn die Rede von \enquote{Python} ist, ist in den meisten Fällen die Referenzimplementation
\emph{CPython} gemeint. Dabei handelt es sich um den ersten und auch am weitesten verbreiteten
Python-Interpreter def, wie der Name schon erahnen lässt, in C geschrieben ist. Er ist Open Source
und wird unter anderem von dem Python-Erfinder Guido van Rossum entwickelt.

Neben CPython existierten weitere Python-Implementationen. Eine der bekanntesten ist das
Java-basierte \emph{Jython}, welches Python-Code zu Java-Bytecode kompiliert und in einer JVM
ausführt. Dadurch lässt sich Python-Code einfach in eine existierende Java-Infrastruktur integrieren
und kann auf beliebige Java-Packages zugreifen.

Insbesondere für Windows-Desktopanwendungen interessant ist das von Microsoft als Open Source
veröffentlichte \emph{IronPython}. Ähnlich wie Jython läuft auch IronPython in einer
sprachunabhängigen Runtime-Umgebung - allerdings nutzt IronPython Microsofts .NET-Framework oder
alternativ das plattformunabhängige \emph{Mono} und kann damit unter anderem auf das
WinForms-GUI-Framework und andere .NET-APIs zugreifen.

Neben diesen Implementierungen existieren noch weitere, die auf spezielle Anwendungen zugeschnitten
sind: \emph{Stackless Python} bietet extrem speicherschonende Pseudo-Threads, während \emph{PyPy}
ein in Python geschriebener Python-Interpreter ist, der einen
JIT-Maschinencode-Compiler enthält und somit teilweise bessere Performance als CPython erzielen
kann. Ein eher unbekannter Python-Interpreter ist \emph{PyMite}, bei dem es sich um eine für
8-Bit-Mikrocontroller optimierte Python-Version handelt, die zwar weder die
Python-Standardbibliothek noch den vollen Funktionsumfang von Python unterstützt, aber dafür trotz
der Einschränkungen einfacher Mikrocontroller funktioniert.


\subsection{Programmierparadigmen}

Das Paradigma einer Programmiersprache ist \enquote{die Sichtweise auf und den Umgang mit
den zu verarbeiteten Daten und Operationen}. \citep[Kap. 1.3.1]{progsprachen} In Python kann man
man sich zwischen drei verschiedenen Paradigmen entscheiden.

\begin{description}
\item[Imperative/prozedurale Programmierung] \hfill \\
Bei der imperativen Programmierung wird eine Folge von Anweisungen sequentiell abgearbeitet. Zur
Kapselung und Wiederverwendung von Funktionalität werden Funktionen genutzt.
\citep[Kap. 1.3.1]{progsprachen}

Abgesehen von diversen Modulen in Pythons Standardbibliothek und Exceptions, die, um eine Hierarchie
zu ermöglichen, als Klassen realisiert sind, kann ein Python-Programm komplett imperativ geschrieben
werden. Dies hat den großen Vorteil, dass insbesondere kleine Kommendozeilenprogramme oftmals sehr
einfach realisiert werden können, da man nicht von der Programmiersprache dazu gezwungen wird,
Klassen einzusetzen ohne dass sie für die Strukturierung der Anwendung nützlich sind.



\item[Objektorientierte Programmierung] \hfill \\
Eine objektorientierte Programmiersprache unterstützt Klassen und Objekte, wobei, anders als bei
einer nur objekt\emph{basierenden} Programmiersprache, auch komplexere Konzepte wie Vererbung und
Polymorphie genutzt werden können.
\citep[Kap. 1.3.1]{progsprachen}

Wie zuvor schon erwähnt, nutzt Python unter anderem in Teilen der Standardbibliothek und für
Exceptions Klassen. Dementsprechend ist es nur logisch, dass in Python objektorientiert entwickelt
werden kann. Dazu bietet Python bietet dem Entwickler ein ausgereiftes Klassensystem, das neben der
in objektorientierten Sprachen üblichen Vererbung auch Mehrfachvererbung und sogenannte
\emph{Metaclasses} unterstützt. Da es sich gerade bei letzterem um ein in Python einzigartiges
Feature handelt, wird im Verlauf dieses Kapitels näher darauf eingegangen.



\item[Funktionale Programmierung] \hfill \\
Eine rein funktionale Programmiersprache basiert nicht auf Wertzuweisungen sondern benutzt
ausschließlich Funktionsdefinitionen, die eine Eingabe in eine Ausgabe transformieren.
\citep[Kap. 1.3.1]{progsprachen}

Python ist allerdings nicht direkt eine funktionale Programmiersprache und damit erst recht keine
rein funktionale Programmiersprache. Dadurch, dass Funktionen \emph{First-Class-Objekte} sind und
sowohl Lambdafunktionen als auch Closures unterstützt werden, kann man in Python sehr einfach
funktionale Elemente mit objektorientierten bzw. prozeduralen Elementen mischen. Die prominentesten
aus anderen funktionalen Programmiersprachen bekannten Methoden sind \lstinline{map()}, um eine
Liste mittels einer Funktion in eine neue Liste zu transformieren, \lstinline{filter()}, um nur die
Elemente aus einer Liste zu übernehmen, die von der Filter-Funktion durch die Rückgabe von
\lstinline{True} akzeptiert wurden, und diverse Werkzeuge für Funktionen höherer Ordnung wie
\lstinline{lambda} und partielle Funktionsapplikation.
\end{description}



\subsection{First-Class-Objekte}\label{firstclass}

Eines der Designziele bei der Entwicklung von Python war, allen Objekten \emph{First-Class}-Status
\citep{pyhist:firstclass} zu geben. Dieses Ziel wurde bei allen Sprachelementen konsequent beachtet,
sodass neben Funktionen auch Klassen, Datentypen und Module diesen Status haben und dementsprechend
Attribute und damit auch Methoden besitzen können. Details zu den verschiedenen Objekttypen finden
sich in der Python-Dokumentation\footnote{\href{http://docs.python.org/2/reference/datamodel.html}{http://docs.python.org/2/reference/datamodel.html}}.

\emph{First-Class} beschreibt Eigenschaften die man bei Objekten, d.h. meist Instanzen von Klassen,
i.a. erwartet, aber im Zusammenhang mit Funktionen, Klassen und anderen Datentypen durchaus
ungewöhnlich sind:

\begin{itemize}
\item Sie können in Variablen und anderen Datenstrukturen gespeichert werden.
\begin{lstlisting}[caption=Zuweisung einer Funktion an eine Variable]
def func():
    pass
something = func
\end{lstlisting}

\item Sie können Funktionsparameter sein.
\begin{lstlisting}[caption=Funktion als Funktionsparameter]
roots = map(math.sqrt, [4, 9, 16])
\end{lstlisting}

\item Sie können Rückgabewert einer Funktion sein.

\item Sie können zur Laufzeit erstellt werden.
\begin{lstlisting}[caption=Klasse wird zur Laufzeit erstellt]
def make_class(cls=object):
    class _extended(cls):
        foo = 'bar'
    return _extended
\end{lstlisting}

\item Sie sind nicht an einen Namen gebunden.
\begin{lstlisting}[caption=Anonyme lambda-Funktion]
def make_power(power):
    return lambda num: num ** power
\end{lstlisting}
\end{itemize}



\subsection{Klassen und Metaklassen}
Die Grundfunktionen des Klassensystems in Python sind mit denen anderer objektorientierter
Programmiersprachen vergleichbar. Eine Klasse kann von beliebig vielen Klassen erben
(Mehrfachvererbung) und Methoden dieser Klassen überschreiben, wobei diese Methoden mittels der
\lstinline{super()}-Funktion weiterhin erreichbar sind.

Die Tatsache, dass in Python keine \lstinline{private}- oder \lstinline{protected}-Schlüsselwörter
existieren, dürfte gerade für einen Java- oder C++-Entwickler sehr ungewohnt sein. Dies liegt
daran, dass in Python das Motto \enquote{we're all consenting adults here} \citep{pymail:adults}
gilt und dementsprechend private Elemente einer Klasse nicht technisch erzwungen sondern alleine
anhand des Namens als privat markiert werden. So gilt eine Variable oder Methode, die mit einem
einzelnen Unterstrich beginnt (\lstinline{_foo}), als Teil der internen API der Klasse. Dies
bedeutet in der Regel, dass sie nicht Teil der Dokumentation ist und die Nutzung auf eigene Gefahr
geschieht - man muss damit rechnen, dass sie in einer beliebigen neuen Version wegfällt oder sich
ihr Verhalten anderweitig ändert. Da es aber durchaus auch einen technischen Grund für private
Variablem gibt - nämlich die Vermeidung von Konflikten in Subklassen - kann durch die Verwendung von
zwei Unterstrichen (\lstinline{__foo}) das sogenannte \emph{name mangling} aktiviert werden. Dies
führt dazu, dass bei jedem Zugriff auf die Variable bzw. Methode der Name in
\lstinline{_ClassName__foo} umgeschrieben wird und somit eine Subklasse denselben Namen nutzen kann,
ohne Konflikte zu verursachen. Es ist aber jederzeit möglich, direkt auf \lstinline{_ClassName__foo}
zuzugreifen.

Eine ähnliche Namensgebung wird für den Konstruktor \lstinline{__init__} und die Überladung von
Operatoren wie \lstinline{__add__} für den binären Plus-Operator verwendet. Der
Python-Styleguide empfielt daher, niemals eigene Namen zu verwenden, die mit zwei Unterstrichen
beginnen und enden, sofern sie nicht bereits dokumentiert sind und dementsprechend eine besondere
Bedeutung haben.

Eines dieser \enquote{magischen} Attribute ist \lstinline{__metaclass__}, mit dem die Metaklasse
einer Klasse beeinflusst werden kann. Um das Konzept von Metaklassen zu verstehen ist zunächst ein
Einblick nötig, wie Python eine Klasse erstellt. Als Beispiel soll dazu die sehr einfache Klasse in
\autoref{lst:python-class} dienen. Sie erbt von der internen Klasse \lstinline{object} und enthält
eine einzige Variable \lstinline{a} mit dem Standardwert \lstinline{1}.

\begin{lstlisting}[caption=Python-Klassendefinition,label=lst:python-class]
class Test(object):
    a = 1
\end{lstlisting}

Wie zuvor schon erwähnt sind Klassen in Python First-Class-Objekte und können zur Laufzeit erstellt
werden. Dies ist natürlich auch bei der Beispielklasse möglich, der Code dazu ist
\lstinline{type('Test', (object,), dict(a=1))}.
Auch ohne Blick in die Dokumentation ist die Funktionsweise von \lstinline{type()} offensichtlich:
Die Funktion erwartet als Parameter den Namen der Klasse, die Liste der Superklassen und das
\emph{dict} der Klasse, d.h. ein Mapping, welches alle Variablen und Methoden der Klasse enthält.

Bekanntermaßen sind die meisten Objekte Instanzen von Klassen. Dies trifft in Python auch für
Klassen selbst zu, die ebenfalls Objekte sind. Jede Klasse ist eine Instanz ihrer Metaklasse; im
Standardfall ist dies \lstinline{type}, allerdings kann eine beliebige andere Klasse oder Funktion
als \lstinline{__metaclass__} angegeben werden, solange sie dieselben Parameter wie
\lstinline{type()} akzeptiert. Dies ermöglicht es einem fortgeschrittenen Entwickler, die Definition
einer Klasse beliebig zu beeinflussen und somit z.B. eine deklarative Syntax für ein
ORM\footnote{Object Relational Mapper} zu ermöglichen oder alle Subklassen einer bestimmten Klasse
einer Liste hinzuzufügen.



\subsection{Lesbarkeit}

Die Sprache legt großen Wert auf lesbaren Code, was insbesondere daran zu erkennen ist, dass Blöcke
nicht wie in den meisten anderen Sprachen durch geschweifte Klammern oder
\emph{begin}/\emph{end}-Statements definiert werden, sondern einzig und allein durch die Einrückung.
Dementsprechend führt inkonsistente Einrückung auch zu einer entsprechenden Fehlermeldung, wie man
in \autoref{lst:python-ident} sehen kann.

Während die erzwungene Einrückung insbesondere bei Python-Anfängern oftmals negativ aufgenommen
wird, ist der Sinn dahinter klar zu erkennen: Gerade bei Projekten mit mehreren Entwicklern wird so
eine einheitliche Einrückung erzwungen, was die Lesbarkeit des Codes stark erhöht.

Allgemein legt Python großen Wert auf \enquote{schönen} Code, was auch im \emph{Zen of
Python}\footnote{Kann über das Easter-Egg \lstinline{import this} in der Python-Shell angezeigt
werden}
festgeschrieben ist:
\enquote{Beautiful is better than ugly.} und \enquote{Readability counts.} \citep{zenofpython}

\begin{lstlisting}[caption=Fehlerhafte Einrückung,label=lst:python-ident]
>>> def fail():
...     a = 'b'
...       b = 'c'
  File "<stdin>", line 3
    b = 'c'
    ^
IndentationError: unexpected indent
\end{lstlisting}




\section{WSGI - Web Server Gateway Interface}

\subsection{Motivation}

Um den Nutzen eines standardisierten, auf die Programmiersprache zugeschnittenen, Interfaces
zwischen Webserver und Anwendung deutlich zu machen, ist zunächst ein Blick auf die älteste und
einfachste Methode zum Ausführen interaktiver Scripts innerhalb eines Webservers angebracht: Das
Common Gateway Interface\footnote{CGI, \href{http://www.ietf.org/rfc/rfc3875}{RFC 3875}
\citep{rfc3875}}).

Bei der Nutzung von CGI wird für jeden Aufruf einer entsprechenden URL ein neuer Prozess gestartet;
Metadaten wie die IP-Adresse des Clients oder die abgerufene URL werden via Umgebungsvariablen
weitergegeben. Die Standarddatenströme \emph{stdin} und \emph{stdout} dienen der Übergabe von
Request-Bodies (z.B. bei \emph{HTTP POST}) bzw. dazu, die Antwort des Programms an den
Webserver zu übergeben.

CGI ist daher sehr einfach zu implementieren. Python bietet mit dem \lstinline{cgi}-Modul
Hilfsfunktionen an, die die Kommunikation über das Common Gateway Interface abstrahieren und
Standardaufgaben wie das Parsen von Formulardaten übernehmen.
Allerdings muss für jeden Request ein neuer Prozess gestartet und jeglicher Initialisierungscode
erneut ausgeführt werden. Dieser Overhead ist außer bei sehr einfachen Anwendungen mit wenigen
Benutzern problematisch und selbst dort würde er zu unangenehm langen Ladezeiten führen, wenn
komplexe Systeme wie ein ORM-Framework initialisiert werden müssten.

\autoref{lst:python-cgi} zeigt ein kleines Beispielscript, welches einen GET-Parameter bzw. einen
Standardtext ausgibt. Man erkennt an dem manuell ausgegebenen \emph{Content-type}-Header gut, dass
CGI sehr simpel ist und das HTTP-Protokoll nur minimalst abstrahiert.

\begin{lstlisting}[caption=Python-CGI-Script,label=lst:python-cgi]
#!/usr/bin/env python

import cgi

form = cgi.FieldStorage()
print 'Content-type: text/plain'
print
print 'Hello %s' % form.getfirst('name', 'World')
\end{lstlisting}

Um die CGI-typischen Probleme zu vermeiden, wurde vor etwa dreizehn Jahren das Modul
\emph{mod\_python} für den weit verbreiteten Apache-Webserver veröffentlicht. Es ermöglichte
Entwicklern erstmals, Python ähnlich komfortabel wie PHP zu nutzen, aber ohne die Nachteile von CGI
in Kauf nehmen zu müssen. Der große Nachteil bei \emph{mod\_python} ist jedoch, dass es
ausschließlich für den Apache-Webserver verfügbar ist und auch die zur Verfügung gestellten APIs
sehr Apache-spezifisch sind. Dies bedeutet, dass man bei der Nutzung von \emph{mod\_python} die in
CGI vorhandene Plattformunabhängigkeit verliert und der Umstieg auf einen anderen Webserver nur mit
viel Aufwand möglich ist.

Im Gegensatz zum sprachunabhängigen CGI und \emph{mod\_python} bietet Java mit der \emph{Servlet}-API
eine saubere und gut dokumentierte Schnittstelle für Webanwendungen, sodass man ein beliebiges
Framework und einen beliebigen Webserver nutzen kann, sofern beide die Servlet-API implementieren.
Daher wurde mit dem \emph{Web Server Gateway Interface} ein Python-Standard verabschiedet, der für
das Zusammenspiel zwischen Webserver und Webanwendung eine einfach zu implementierende Schnittstelle
definiert.


\subsection{Interface}\label{wsgi-interface}

Der grundlegende Aufbau von WSGI lässt sich am Einfachsten anhand eines Beispiels beschreiben.
\autoref{lst:wsgi-app} zeigt eine sehr einfache \emph{Hello World}-WSGI-Anwendung, an der man aber
dennoch gut erkennen kann, wie das Interface aufgebaut ist.

\begin{lstlisting}[caption=Einfache WSGI-Applikation,label=lst:wsgi-app]
def application(environ, start_response):
    start_response('200 OK', [('Content-Type', 'text/plain')])
    yield 'Hello World!'
\end{lstlisting}

Eine WSGI-Anwendung enthält ein globales \emph{callable object}, das typischerweise den Namen
\lstinline{application} besitzt. Dieses Objekt kann wie im Beispiel eine einfache Funktion sein,
aber auch eine Klasse oder eine Klasseninstanz, die \lstinline{__call__} implementiert. Letzteres
ist dabei die insbesondere in größeren Frameworks bevorzugte Methode, da das Objekt alle
frameworkspezifischen Daten wie z.B. eine Routingtabelle für URLs enthalten kann.

Alle für WSGI nötigen Daten werden als Parameter übergeben. \lstinline{environ} enthält dabei die
bereits aus CGI bekannten Umgebungsvariablen wie \emph{REQUEST\_METHOD} für die HTTP-Methode
(\emph{GET}, \emph{POST}, \ldots) und \emph{QUERY\_STRING} für die GET-Parameter in der URL. Neben
diesen Variablen sind weitere, WSGI-spezifische, Variablen enthalten. Diese übermitteln u.a. neben
der verwendete Version der WSGI-Spezifikation auch die Datenstreams für den
\emph{Request Body} (in CGI: \emph{stdin}) und Fehlermeldungen (in CGI: \emph{stderr}).

Beim zweiten Parameter, \lstinline{start_response}, handelt es sich genau wie beim
Applikationsobjekt um ein \emph{callable}. Bei einem erfolgreichen Request wird es von der Anwendung
bzw. dem Framework genau einmal aufgerufen; als Parameter werden der HTTP-Statuscode (in der
Regel \emph{200 OK}) und eine Liste mit den HTTP-Antwortheadern übergeben. Sofern bei der
Abarbeitung des Requests ein Fehler auftritt, kann die Funktion erneut aufgerufen werden, um einen
anderen Statuscode (meist \emph{500 Internal Server Error}) und die Details des aufgetretenen
Fehlers (insbesondere den Stacktrace) zurückzugeben.

Zu diesem Zeitpunkt stellt sich oftmals die Frage, wieso Statuscode und Header nicht einfach mittels
\lstinline{return} zurückgegeben werden, statt den Umweg über einen separaten Funktionsaufruf zu
gehen. Dazu ist ein Blick auf den Aufbau einer HTTP-Antwort notwendig: Statuscode und Header stehen
immer am Anfang der Antwort. Danach folgt der durch eine Leerzeile von den Headern abgegrenzte Body,
der die eigentliche Antwort enthält - oftmals ein HTML-Dokument. Während es bei einer
typischen Webseite kein Problem wäre, diesen Body komplett in einer Variable zu speichern und am
Ende zusammen mit den Headern zurückzugeben, so würde es bei größeren Daten unnötig viel
Arbeitsspeicher verbrauchen und es könnten erst Daten an den Client gesendet werden, wenn die
Antwort vollständig vorhanden ist. Um dieses Problem zu vermeiden, ist der Rückgabewert beim Aufruf
des Applikationsobjekts ein \emph{iterable}. Dabei kann es sich einfach um eine Liste handeln, deren
einziges Element ein String mit der kompletten Antwort ist, aber alternativ auch um einen
\enquote{echten} Generator, der die Daten erst beim Iterieren erzeugt - letzteres bietet sich gerade
bei größeren Daten oder Archiven an, die in diesem Fall blockweise gelesen und gesendet bzw.
on-the-fly erzeugt werden können. Aufgrund dieser Funktionsweise könnte die Antwort in
\autoref{lst:wsgi-app} auch mit \lstinline{return ['Hello World!']} zurückgegeben werden, statt das
für Generatorfunktionen typische \lstinline{yield} zu benutzen.

Um bestehende Anwendungen, die eine \lstinline{write(data)}-Funktion erwarten, ebenfalls ohne
größeren Aufwand WSGI-kompatibel machen zu können, gibt die Funktion \lstinline{start_response()}
eine solche Funktion zurück. Die WSGI-Spezifikation rät allerdings von der Nutzung dieser Funktion
ab, da sie diverse Nachteile mit sich bringt. Anders als bei einer Generatorfunktion kann der
Webserver dort nicht jederzeit die Abarbeitung unterbrechen sondern es ist alleine der Anwendung
überlassen, wann sie Daten sendet.


\subsection{Verbreitung}

Jedes aktuelle Python-Webframework basiert inzwischen auf WSGI, wobei oftmals entsprechende Wrapper
für CGI und FastCGI zur Verfügung gestellt werden, von deren Nutzung allerdings in der Regel
abgeraten wird, sofern es irgendwie möglich ist, WSGI direkt zu nutzen.

Seitens der Webserver ist WSGI ebenfalls verbreitet, wobei dort verschiedene Ansätze genutzt werden:
Der Apache-Webserver integriert mit \emph{mod\_wsgi} einen WSGI-Container direkt im Webserver, was
eine sehr einfache Konfiguration über die bereits existierenden Webserver-Konfigurationsdateien zur
Folge hat. Nichtsdestotrotz unterstützt \emph{mod\_wsgi} alle wichtigen Features - insbesondere
separate Prozesse für den Python-Interpreter.
Andere Webserver wie \emph{nginx} setzen auf einen externen WSGI-Container wie \emph{uWSGI} oder
sind wie \emph{Gunicorn} direkt in Python geschrieben und unterstützen WSGI nativ.
Der große Vorteil eines Containers wie \emph{uWSGI} ist die strikte Trennung zwischen Webserver und
Python, sodass interaktiver Code problemlos unter einem separaten Systemuser ausgeführt werden kann.
Dies hat den Vorteil, dass besseres Performance-Tuning möglich ist, wenn Webserver und
Python-Interpreter nicht eng miteinander verknüpft sind.

\chapter{Grundlagen}

\section{Python}

\subsection{Geschichte}

Python wurde 1989 von Guido van Rossum am \emph{CWI}\footnote{Centrum Wiskunde \& Informatica - das
nationale niederländische Forschungsinstitut für Mathematik und Informatik} entwickelt und 1991 als
Betaversion in der Usenet-Newsgroup \emph{alt.sources} veröffentlicht.

Seitdem wurde die Sprache kontinuierlich weiterentwickelt und 1994 in der Version 1.0 released. Die
erste Version von Python 2.0 erschien im Herbst 2000 und legte den Grundstein für die heute weit
verbreiteten Versionen 2.6 und 2.7, wobei es sich bei letzterer um die letzte Version von Python 2
handelt.

Version 3 der Programmiersprache ist seit 2008 verfügbar, wobei sie in der Verbreitung bisher noch
nicht an Python 2 anknüpfen konnte. Dies liegt hauptsächlich an den teilweise nicht zu Python 2
kompatiblen Änderungen, die zu einem Henne-Ei-Problem führen: Einige populäre Python-Bibliotheken
und -Frameworks sind noch nicht mit Python 3 kompatibel, was Entwickler neuer Anwendungen oftmals
davon abhält, auf Python 3 zu setzen. Dies führt allerdings wieder zu geringer Nachfrage nach mit
Python 3 kompatiblen Bibliotheken. \citep{pywiki:2or3}

Um diesem Problem entgegenzuwirken, wurden in den Versionen 2.6 und 3.3 einige Syntaxelemente
hinzugefügt, die es Entwicklern einfacher machen, mit beiden Versionen kompatiblen Code zu
schreiben. \citep{pep:414}


\subsection{Anwendungsgebiete}

Bei Python handelt es sich um eine höhere \emph{general-purpose}-Programmiersprache mit der sich
nahezu beliebige Anwendungen einfach realisieren lassen. Es existieren diverse Frameworks sowohl für
Webanwendungen als auch für Desktopanwendungen mit grafischer Oberfläche, sodass man als Entwickler
in der Regel nur wenig nicht-anwendungsspezifischen Code schreiben muss.

Während es sich bei den Webframeworks in der Regel um reine Python-Frameworks handelt, so sind die
GUI-Frameworks meist Bindings für in C oder C++ geschriebene Frameworks wie \emph{Qt}, \emph{GTK+}
oder \emph{wxWidgets}, was natürlich den großen Vorteil hat, dass diese Frameworks bereits außerhalb
der Python-Community häufig genutzt werden und dementsprechend stabil sind.

Ein weiteres Gebiet, in dem Python häufig verwendet wird, ist das Scripting von diversen Tools.
Beispielsweise unterstützen populäre Editoren wie \emph{Vim} Python-Scripting, um Aufgaben zu
automatisieren oder beliebige andere Aktionen im Editor auszufühen.

Wie die meisten Open-Source-Programmiersprachen ist Python insbesondere in der Linux-Welt weit
verbreitet und wird dort auch oftmals für einfache Systemtools eingesetzt, wo eine Implementierung
in C deutlich aufwändiger wäre. Beispielsweise ist der Paketmanager \emph{Portage} der
Linux-Distribution \emph{Gentoo} fast vollständig in Python geschrieben; nur einige Bibliotheken
sind in C implementiert, da C-Code deutlich performanter ist als interpretierter Code einer
Scriptsprache.


\subsection{Erweiterbarkeit}

Anders als beispielsweise in PHP, wo Erweiterungen ausschließlich in C geschrieben werden können,
bietet Python verschiedene Möglichkeiten, mit nicht-Python-Code zu interagieren. Natürlich können
native Erweiterungen für Python in C oder einer anderen Sprache, die zu einer \emph{Shared Library}
kompiliert werden kann, implementiert werden und haben dadurch vollen Zugriff auf alle Lowlevel-APIs
von Python. Dies ist allerdings oftmals nicht notwendig und führt nur zu unnötig komplexem Code, da
die C-APIs auf einer sehr niedrigen Ebene arbeiten und beispielsweise die \emph{Reference Counts}
des Garbage Collectors vom Entwickler penibel aktualisiert werden müssen, um Speicherlecks zu
vermeiden.

Ein häufiger Grund für native Erweiterungen ist das Zugänglichmachen von existierenden Bibliotheken,
die in C geschrieben sind. Auf den ersten Blick würde man vermuten, dass es in diesem Fall sinnvoll
ist, eine native Erweiterung zu schreiben. Wenn man bedenkt, was das Ziel eines solchen Interfaces
ist, merkt man jedoch schnell, dass meist weder Lowlevel-APIs noch die erhöhte Performance von
C-Code notwendig ist - alles was benötigt wird ist eine Möglichkeit, Funktionen aus einer
kompilierten Bibliothek aufzurufen. Dazu bietet Python mit dem in der Standardbibliothek enthaltenen
\emph{ctypes}-Paket ein einfach nutzbares FFI\footnote{Foreign Function Interface}, mit welchem ein
Entwickler in reinem Python-Code beliebige Funktionen aus \emph{Shared Libraries} aufrufen und sogar
Python-Funktionen als Callback übergeben kann.

\lstinputlisting[caption=ctypes-Beispiel]{code/ffi-ctypes.py}

Auch wenn die Python-Erweiterung performancekritische Operationen enthält und deshalb nicht in
Python implementiert werden soll, gibt es eine Möglichkeit, fast reinen Python-Code dafür zu
schreiben. \emph{Cython}\footnote{\href{http://cython.org}{http://cython.org}} ist eine mit
C-Elementen erweiterte Abwandlung von Python, die zu nativem Code kompiliert wird. Insbesondere um
Python-Klassen zu erstellen ist Cython ideal geeignet, da die Cython-Syntax dabei große Ähnlichkeit
mit der Python-Syntax hat und man nicht die unintuitive C-API nutzen muss.


\subsection{Implementationen}

Wenn die Rede von \enquote{Python} ist, ist in den meisten Fällen die Referenzimplementation
\emph{CPython} gemeint. Dabei handelt es sich um den ersten und auch am weitesten verbreiteten
Python-Interpreter def, wie der Name schon erahnen lässt, in C geschrieben ist. Er ist Open Source
und wird unter anderem von dem Python-Erfinder Guido van Rossum entwickelt.

Neben CPython existierten weitere Python-Implementationen. Eine der bekanntesten ist das
Java-basierte \emph{Jython}, welches Python-Code zu Java-Bytecode kompiliert und in einer JVM
ausführt. Dadurch lässt sich Python-Code einfach in eine existierende Java-Infrastruktur integrieren
und kann auf beliebige Java-Packages zugreifen.

Insbesondere für Windows-Desktopanwendungen interessant ist das von Microsoft als Open Source
veröffentlichte \emph{IronPython}. Ähnlich wie Jython läuft auch IronPython in einer
sprachunabhängigen Runtime-Umgebung - allerdings nutzt IronPython Microsofts .NET-Framework oder
alternativ das plattformunabhängige \emph{Mono} und kann damit unter anderem auf das
WinForms-GUI-Framework und andere .NET-APIs zugreifen.

Neben diesen Implementierungen existieren noch weitere, die auf spezielle Anwendungen zugeschnitten
sind. Beispielsweise ermöglicht \emph{Stackless Python} extrem speicherschonende Pseudo-Threads und
\emph{PyPy} ist ein in Python selbst geschriebener Python-Interpreter, der einen
JIT-Maschinencode-Compiler enthält und somit teilweise bessere Performance als CPython erzielen
kann. Ein eher unbekannter Python-Interpreter ist \emph{PyMite}, bei dem es sich um eine für
8-Bit-Mikrocontroller optimierte Python-Version handelt, die zwar weder die
Python-Standardbibliothek noch den vollen Funktionsumfang von Python unterstützt, aber dafür trotz
der Einschränkungen einfacher Mikrocontroller funktioniert.


\subsection{Programmierparadigmen}
\todotext{Paradigmen}


\subsection{Lesbarkeit}

Die Sprache legt großen Wert auf lesbaren Code, was insbesondere daran zu erkennen ist, dass Blöcke
nicht wie in den meisten anderen Sprachen durch geschweifte Klammern oder
\emph{begin}/\emph{end}-Statements definiert werden sondern einzig und allein durch die Einrückung.
Dementsprechend führt inkonsistente Einrückung auch zu einer entsprechenden Fehlermeldung wie man in
\autoref{lst:python-ident} sehen kann.

Während die erzwungene Einrückung insbesondere bei Python-Anfängern oftmals negativ aufgenommen
wird, ist der Sinn dahinter klar zu erkennen: Gerade bei Projekten mit mehreren Entwicklern wird so
eine einheitliche Einrückung erzwungen, was die Lesbarkeit des Codes stark erhöht.

Allgemein legt Python großen wert auf \enquote{schönen} Code, was auch im - unter anderem über das
Easter-Egg \lstinline{import this} abrufbaren - \emph{Zen of Python} festgeschrieben ist:
\enquote{Beautiful is better than ugly.} und \enquote{Readability counts.} \citep{zenofpython}

\begin{lstlisting}[caption=Fehlerhafte Einrückung,label=lst:python-ident]
>>> def fail():
...     a = 'b'
...       b = 'c'
  File "<stdin>", line 3
    b = 'c'
    ^
IndentationError: unexpected indent
\end{lstlisting}



\subsection{First-Class-Objekte}
\todotext{Firstclass}


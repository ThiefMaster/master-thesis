\chapter{Grundlagen}

\section{Python}

\subsection{Geschichte}

Python wurde 1989 von Guido van Rossum am \emph{CWI}\footnote{Centrum Wiskunde \& Informatica - das
nationale niederländische Forschungsinstitut für Mathematik und Informatik} entwickelt und 1991 als
Betaversion in der Usenet-Newsgroup \emph{alt.sources} veröffentlicht.

Seitdem wurde die Sprache kontinuierlich weiterentwickelt und 1994 in der Version 1.0 released. Die
erste Version von Python 2.0 erschien im Herbst 2000 und legte den Grundstein für die heute weit
verbreiteten Versionen 2.6 und 2.7, wobei es sich bei letzterer um die letzte Version von Python 2
handelt.

Version 3 der Programmiersprache ist seit 2008 verfügbar, wobei sie in der Verbreitung bisher nicht
an Python 2 anknüpfen konnte. Dies liegt hauptsächlich an den teilweise nicht zu Python 2
kompatiblen Änderungen, die zu einem Henne-Ei-Problem führen: Viele populäre Python-Bibliotheken
sind noch nicht mit Python 3 kompatibel, was Entwickler neuer Anwendungen oftmals davon abhält, auf
Python 3 zu setzen. Dies führt allerdings wieder zu geringer Nachfrage nach mit Python 3 kompatiblen
Bibliotheken.

Um diesem Problem entgegenzuwirken, wurden in den Versionen 2.7 und 3.3 einige Syntaxelemente
hinzugefügt, die es Entwicklern einfacher machen, mit beiden Versionen kompatiblen Code zu
schreiben.


\subsection{Anwendungsgebiete}
\todotext{Anwendungsgebiete}


\subsection{Programmierparadigmen}
\todotext{Paradigmen}


\subsection{Lesbarkeit}
\todotext{Lesbarkeit, sauberer Code}


\subsection{First-Class-Objekte}
\todotext{Firstclass}


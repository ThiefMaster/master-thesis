\section{Indico}

Das derzeit in Indico verwendete System kann zwar grob als Framework betrachtet werden, ist jedoch
nicht wirklich ein Framework, da sehr viele Dinge speziell auf die Anwendung zugeschnitten sind,
statt generisch bzw. wiederverwendbar zu sein. Dies ist bereits daran zu erkennen, dass sich der
Frameworkcode nicht in einem separaten Modul oder Package befindet sondern Teil der Indico-Codebasis
ist.

Den Kern des Frameworks bildet das Package \lstinline{indico.web.wsgi}, welches das in
\autoref{wsgi-interface} beschriebene WSGI-Interface implementiert und eine Emulationsschicht für
\emph{mod\_python} bereitstellt. Neben dem Bereitstellen der entsprechenden APIs und Objekte ist die
wichtigste Aufgabe dieser Schicht, Requests auf die entsprechenden Python-Dateien zu mappen.
Der Aufruf von \emph{/index.py/foo} führt zum Aufruf der Funktion \lstinline{foo()}
in der Datei \emph{index.py} im \emph{htdocs}-Ordner. Da diese Dateien jedoch, um das Verhalten von
\emph{mod\_python} beizubehalten, erst dann geladen werden, wenn ein entsprechender Request
empfangen wird, können sie nicht beim Initialisierung der Anwendung importiert und in einem Mapping
abgelegt werden. Stattdessen wird der gesamte Code der entsprechenden Datei in einen String geladen
und danach ausgeführt. \autoref{lst:mp-emulation} zeigt eine vereinfachte Version des dafür
zuständigen Codes.

\begin{lstlisting}[caption=Laden der Legacy-Python-Dateien,label=lst:mp-emulation]
def mp_legacy_publisher(req, module, handler):
    the_module = open(module).read()
    module_globals = {}
    exec(the_module, module_globals)
    return module_globals[handler](req, **req.form)
\end{lstlisting}

Bekanntermaßen ist es schlechter Stil, Code aus Strings auszuführen. In Python hat es darüberhinaus
noch den Nachteil, dass Debugger keinen Zugriff auf den Code haben, da sie nicht wissen, aus welcher
Datei der Code stammt, und Python den Quelltext selbst nicht zusammen mit dem Bytecode im Speicher
hält. In Indico ist dies allerdings kein größeres Problem, da die Dateien keine Logik enthalten
sondern nur die entsprechende Handlerfunktion aufrufen. Unsauber ist es dennoch.

Neben diesen Legacy-Handlern, die jedoch für fast alle Bestandteile von Indico genutzt werden,
existiert ein sehr einfaches Routingsystem, das Anfragen anhand des ersten Pfadsegments der URL an
eine Handlerfunktion übergibt. Dies ist prinzipiell besser, da so die Nutzung von \lstinline{exec}
vermieden werden könnte. Letzendlich wird jedoch genau dieselbe zuvor schon
beschriebene Funktion verwendet, was leicht dazu verleitet, in der in \autoref{lst:lame-routing}
gezeigten Routingtabelle direkt auf die Handlerfunktion mit der eigentlichen Logik zu verweisen.

\begin{lstlisting}[caption=Einfaches URL-Routing,label=lst:lame-routing]
{'': ((self.htdocs_dir, 'index.py'), 'index', '', None),
 'services': ((DIR_SERVICES, 'handler.py'), 'handler', '', None),
 'export': ((DIR_MODULES, 'wsgi_handler.py'), 'handler', '', None),
 'api': ((DIR_MODULES, 'wsgi_handler.py'), 'handler', '', None)
}
\end{lstlisting}

Dies führt dann dazu, dass das mit \lstinline{exec} verbundene Debuggingproblem plötzlich
auch relevanten Code betrifft statt nur einen einfachen Funktionsaufruf. Umgangen werden kann das
zum Glück sehr einfach, indem ein separates Python-Modul nur den Aufruf der eigentlichen
Handlerfunktion enthält und dieses Modul in der Routingtabelle referenziert wird.

Für die eigentliche Anwendungslogik sind die \lstinline{RH}-Klassen zuständig. Die Basisklasse
enthält dabei die Logik für die Datenbankverbindung und zum Abfangen von Fehlern, während die
Subklassen davon die eigentliche Anwendungslogik implementieren. Die Abgrenzung zwischen Framework
und Anwendung wird teilweise verwässert, da Teile der Basisklasse auf Anwendungscode zugreifen
während diverse andere nur Frameworkcode (in der Regel das \lstinline{req}-Objekt mit den
\emph{mod\_python})-Daten nutzen.

\begin{description}
\item[Modularität] \hfill \\
Indico enthält ein Modulsystem, welches sowohl für Code, der ein fester Bestandteil von Indico ist,
als auch für externen Code genutzt werden kann. In beiden Fällen bietet das Modulsystem einen
Namespace in der Datenbank, der ausschließlich vom jeweiligen Modul genutzt wird, und einen
Menüpunkt in der Administrationsoberfläche, über den das Modul aktiviert bzw. deaktiviert und
konfiguriert werden kann. Dies hat den großen Vorteil, dass die Konfiguration bei den meisten
Modulen sehr einfach ist - sowohl für die Administratoren als auch für den Entwickler.

Einige Module (z.B. das Interface zum CERN-Paymentsystem und zur CERN-Suchmaschine) sind
nicht Teil des Open-Source-Projekts. Daher muss es für solche Module eine Möglichkeit geben, auch
ohne direkt im Indico-Code referenziert zu werden, auf Datenstrukturen zugreifen und bei bestimmten
Ereignissen Code ausführen zu können. Dies wurde in Indico mithilfe von \emph{entry points}
realisiert. Dabei handelt es sich um ein Feature des Paketsystems von Python, über das ein Paket
Objekte für einen bestimmten \emph{entry point} zentral registrieren kann. Indico kann dann unter
Verwendung der Funktion \lstinline{pkg_resources.iter_entry_points('indico.ext')} über diese Objekte
iterieren und die entsprechenden Plugins laden.

Plugins können über reguläre Ausdrücke Code für bestimmte URLs ausführen, sodass sie auch eigene
Seiten hinzufügen können; in der Regel folgt die URL-Struktur dabei den \emph{mod\_python}-URLs,
d.h. \emph{something.py} bzw. \emph{something.py/someAction}. Daneben können Plugins sowohl
Funktionen für die JSON-RPC-API als auch Endpoints für die REST-basierte Export-API registrieren.


\item[URL-Routing] \hfill \\
Wie oben bereits beschrieben, emuliert Indico \emph{mod\_python} und
verwendet daher das bereits aus CGI-Anwendungen Mapping zwischen URLs und Dateisystem. Der WSGI-Kern
von Indico kann Anfragen anhand des ersten Pfadsegments routen, allerdings ist dies für die
wenigsten Use-Cases ausreichend. Für \emph{clean URLs} sind in der Regel alle Pfadsegmente relevant
und es müssen dynamische Segmente unterstützt werden, um IDs und sonstige Parameter in den Pfad
einbauen zu können.

Indico-Plugins haben die Möglichkeit, über einen regulären Ausdruck beliebige URLs zu mappen und
über Regex-Gruppen auch auf einzelne Pfadelemente zuzugreifen, um z.B. eine ID zu
extrahieren. Dieses Feature ist jedoch ausschließlich in Plugins verfügbar und nicht für Code,
der Teil des Indico-Kerns ist. Zusätzlich müssen die regulären Ausdrücke jeweils in der
\lstinline{RH}-Klasse definiert werden, wodurch die URL-Mappings über den gesamten Plugincode
verstreut sind statt übersichtlich in einer einzelnen Datei aufgelistet zu sein.

Die Export-API nutzt ebenfalls reguläre Ausdrücke, um URLs unterhalb von \emph{/export/} und
\emph{/api/} auf die jeweiligen Klassen zu mappen. Allerdings ist dieses Routing speziell auf die
Struktur der API zugeschnitten und weder Teil des eigentlichen Indico-Frameworks noch ohne Weiteres
für andere Bereiche von Indico nutzbar.

Unabhängig davon, wie eine bestimmte URL auf den dazugehörigen Code gemappt wird, bietet Indico mit
den \lstinline{URLHandler}-Klassen eine Möglichkeit, URLs zentral zu definieren und zu generieren.
Dabei können beliebige GET-Parameter hinzugefügt werden, wobei die jeweilige
\lstinline{URLHandler}-Klasse diese modifizieren kann, bevor sie an die URL angehängt werden. Die
Änderung einer URL im \lstinline{URLHandler} wirkt sich jedoch logischerweise nur auf die
Generierung von URLs und nicht auf das Routing auf, was dem DRY-Konzept\footnote{Don't Repeat
Yourself} widerspricht.


\item[Templateengine] \hfill \\
Indico nutzt die Mako-Templateengine\footnote{\href{http://www.makotemplates.org}{http://www.makotemplates.org}}.
Dabei handelt es sich um eine von Indico unabhängige Open-Source-Templateengine unter der
MIT-Lizenz. Sie unterstützt alle von einer modernen Templateengine erwarteten Features wie
Vererbung, Includes, Makros, Blöcke, Variablen, Filter und Kontrollstrukturen. Daneben erlaubt sie
auch, beliebigen Python-Code in einem Template zu verwenden.

Mako wird in Indico unverändert verwendet, allerdings wird der Templatekontext um diverse
Hilfsfunktionen und -variablen erweitert, sodass häufig genutzte Funktionen in jedem Template
verfügbar sind.

\lstinputlisting[caption=Mako-Template]{code/mako.tpl}


\item[Datenbankanbindung] \hfill \\
Indico enthält ein einfaches Interface zum Zugriff auf die ZODB-Datenbank. Dieses wird unter anderem
in den \lstinline{RH}-Klassen und den verschiedenen APIs genutzt, um eine Datenbankverbindung
aufzubauen und im Fehlerfall die Ausführung des Requests zu wiederholen. Während man das
Datenbankinterface selbst definitiv zum Framework zählen kann, ist der Code für die Wiederholungen
stark mit Anwendungscode vermischt.

Das Modulsystem von Indico nutzt die Datenbank sowohl zum Speichern der Konfigurationseinstellungen
als auch für die Metadaten der verschiedenen Module, sodass sie nicht bei jedem Aufruf neu über die
\emph{entry points} geladen werden müssen. Dabei wird massiv darauf gesetzt, dass es sich bei ZODB
um eine Python-Objektdatenbank handelt. Damit können auch komplexe Datentypen wie Python-Module
direkt in der Datenbank gespeichert werden statt nur den Pfad zur dazugehörigen Python-Datei
abzulegen.


\item[Sessions] \hfill \\
Indico speichert Sessions in der ZODB-Datenbank ab. Dies erlaubt es, neben beliebigen anderen
Objekten, auch bereits in der Datenbank gespeicherte Objekte in der Session abzulegen, ohne
zusätzlichen Speicher zu benötigen. Zum Speichern von Daten in der Session existieren zwei
Möglichkeiten. Häufig genutzte Elemente wie der aktuelle User oder seine Spracheinstellung sind
direkt im Sessionobjekt abgelegt und dementsprechend über ihre Gettermethode oder direkt über die
darunterliegende Eigenschaft des Objekts abrufbar. Darüberhinaus können beliebige Daten ähnlich wie
in einem \lstinline{dict} abgelegt werden. Dies geschieht über die Methoden \lstinline{setVar()},
\lstinline{getVar()} und \lstinline{removeVar()}.

Eine Session wird nur dann in der Datenbank abgespeichert, wenn sie auch Daten enthält. Dies hat den
Vorteil, dass insbesondere für nicht eingeloggte Benutzer kein Sessionobjekt in der Datenbank
abgespeichert werden muss. Nichsdestotrotz sammeln sich mit der Zeit sehr viele alte Session an;
mangels einer Möglichkeit, einzelne Objekte in der ZODB automatisch nach einer gewissen Zeit zu
verwerfen, müssen alte Sessions regelmäßig manuell bzw. über ein zeitgesteuertes Script aus der
Datenbank gelöscht zu werden.

Während eines Requests, wird die Session als Eigenschaft im \emph{mod\_python}-Request-Objekt
gespeichert. Das für Sessions zwingend notwendige Cookie wird ebenfalls über dieses Objekt
gesetzt. Dies hat zur Folge, dass das Sessionsystem stark von der \emph{mod\_python}-Emulation
abhängig ist.


\item[Caching] \hfill \\
Indico bietet ein generisches Cache-Interface, das verschiedene Backends nutzen kann.
Standardmäßig werden Memcached, Redis und Dateien unterstützt, wobei es für den Entwickler auch
einfach ist, weitere Backends hinzuzufügen. Das Interface beschränkt sich auf die am häufigsten
genutzten Methoden \lstinline{get}, \lstinline{set} und \lstinline{delete}, wobei jeweils auch
eine \lstinline{_multi}-Variante existiert, die je nach Backend performanter ist sofern mehrere
Einträge auf einmal verändert bzw. ausgelesen werden sollen. Ebenfalls unabhängig vom Backend kann
für jeden Cache-Eintrag eine Expire-Zeit angegeben werden. Dies hat bei der Nutzung des
Datei-Backends jedoch zur Folge, dass abgelaufene Einträge nur gelöscht werden, wenn versucht wird,
auf sie zuzugreifen.

Zur besseren Strukturierung des Codes kann jeder Instanz des \lstinline{GenericCache}-Objekts ein
Namespace zugewiesen werden, der automatisch mit dem jeweiligen Cache-Key kombiniert wird. Dies hat
ebenfalls den Vorteil, dass nicht eine einzelne Instanz der Klasse global verfügbar sein muss
sondern z.B. in einer Klasse, die den Cache nutzen will, eine neue Instanz erstellt wird.
Bei nicht-lokalen Caches wie Redis und Memcached wird dabei instanzübergreifend eine bestehende
Verbindung verwendet, sodass auch bei vielen Instanzen nicht übermäßig viele Verbindungen aufgebaut
werden müssen.


\item[Integrierbarkeit] \hfill \\
Da das derzeitige Framework ein Teil von Indico ist, ist dieser Punkt nicht relevant. Der
Vollständigkeit halber sei erwähnt, dass es nicht praktikabel wäre, das Indico-Framework von der
Indico-Anwendung zu trennen um es in einer anderen Anwendung nutzen zu können.


\item[Erweiterbarkeit] \hfill \\
Als Bestandteil von Indico ist die einfachste Art, das Framework zu erweitern, es direkt zu
modifizieren. Da keine Standalone-Version existiert, ist kein Fork notwendig, weshalb auch die damit
üblicherweise verbundenen Probleme ausbleiben.


\item[Sonstige Features] \hfill \\
Indico bietet mit dem \lstinline{ContextManager} eine Möglichkeit, threadsicher Daten ähnlich wie in
einer globalen Variable abzuspeichern. Diese werden jeweils zu Beginn eines neuen Requests gelöscht,
sodass dort temporäre Daten abgelegt werden können, ohne sie als Funktionsparameter übergeben zu
müssen. Insbesondere für Dinge wie das aktuelle Request-Objekt hat dies den Vorteil, dass es falls
benötigt jederzeit verfügbar ist, ohne immer nicht explizit übergeben zu werden.


\item[Dokumentation] \hfill \\
Es existiert keine echte externe Dokumentation. Der Code ist teilweise kommentiert, wobei der
Kommentarstil nicht einheitlich ist und man kann deutlich erkennen, dass mehrere Entwickler daran
gearbeitet haben. Dementsprechend sind einige Teile sehr ausführlich kommentiert, während andere
fast garnicht kommentiert sind.

Die Dokumention im Indico-Wiki\footnote{\href{http://indico-software.org/wiki/Dev}{http://indico-software.org/wiki/Dev}}
geht hauptsächlich auf anwendungsspezifische Dinge ein, weshalb sie insbesondere auch für einen
neuen Entwickler, der etwas am Framework selbst ändern muss, nicht sehr hilfreich ist.


\item[Lizenz] \hfill \\
Da das Framework Teil von Indico ist und auch nicht \emph{standalone} verfügbar ist, steht es genau
wie Indico selbst unter der GPL.

\end{description}

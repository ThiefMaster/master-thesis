\chapter{Auswahl eines Frameworks}

\section{Migrationspfade}
Sowohl die einzelnen Frameworks als auch die Kombination eines dieser Frameworks mit Teilen des
Indico-Frameworks haben spezifische Vor- und Nachteile. Im Folgenden werden sowohl der vollständige
Umstieg auf ein bestimmtes Framework als auch Hybridlösungen vorgestellt und bewertet.

Der Vollständigkeit halber wird auch die - zugegeben sehr unwahrscheinliche - Option, komplett bei
der bestehenden Lösung zu bleiben, kurz betrachtet.

\subsection{Weiternutzung des Indico-Frameworks}
Getreu dem Motto \enquote{never change a running system} stellt sich natürlich die Frage, ob sich
der nicht zu unterschätzende Aufwand, eine komplexe Software wie Indico auf ein neues Framework
umzustellen, überhaupt lohnt. Diese Frage ist im Fall von Indico aus verschiedenen Gründen zu
bejahen. Insbesondere die Kombination aus einem sehr einfachen Routingsystem und dem
URL-Dateisystem-Mapping im CGI-Stil bietet bei der Entwicklung neuer Features diverse Nachteile:
Sofern das neue Feature eine für den Benutzer direkt zugängliche Seite besitzt ist aus Gründen der
Einheitlichkeit eine \emph{*.py}-URL angebracht. Im Falle eines neuen Moduls wie beispielsweise der
HTTP-API ist dies jedoch nicht gegeben, weshalb sich dort anbietet, ein neues Pfadsegment
einzuführen und über dieses alle Anfragen in das entsprechende Modul zu routen, was jedoch bedeutet,
dass jegliches weitere Routing von dem jeweiligen Modul übernommen werden muss.

Ein enormer Vorteil bei der bestehenden Lösung ist natürlich, dass sie seit Jahren relativ stabil
läuft und von einer breiten Userbasis benutzt wird und damit die \enquote{Kinderkrankheiten} eines
neuen Systems nicht vorhanden sind. Ebenfalls ist beispielsweise die ZODB-Datenbank stark in das
aktuelle System integriert, wobei anzumerken ist, dass dabei teilweise fast derselbe Code an
mehreren Stellen verwendet wird: Die \lstinline{RH}-Klassen und die JSON-RPC-API verwenden jeweils
dasselbe Retry-System, um Datenbankkonflikte zu behandeln, jedoch ist es in den jeweiligen Klassen
unabhängig voneinander implementiert.

Ein Vorteil beim Verzicht auf ein neues Framework wäre natürlich auch, die dafür notwendige
Entwicklungszeit anderweitig nutzen zu können. Dies ist jedoch nur kurzfristig gesehen ein Vorteil,
da die Nichtnutzung eines modernen Frameworks in Zukunft bei neuen Features mit hoher
Wahrscheinlichkeit zu deutlich mehr Entwicklungsaufwand führt als mit einem entsprechenden Framework
notwendig wäre.

Insbesondere aufgrund dieser Folge ist ein neues Framework also unbedingt notwendig. Daher wird die
Option, weiterhin ausschließlich das Indico-Framework, in den folgenden Abschnitten nicht mehr
betrachtet.

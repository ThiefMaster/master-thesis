\chapter{Auswahl eines Frameworks}

\section{Migrationspfade}
Sowohl die einzelnen Frameworks als auch die Kombination eines dieser Frameworks mit Teilen des
Indico-Frameworks haben spezifische Vor- und Nachteile. Im Folgenden werden sowohl der vollständige
Umstieg auf ein bestimmtes Framework als auch Hybridlösungen vorgestellt und bewertet.

Der Vollständigkeit halber wird auch die - zugegeben sehr unwahrscheinliche - Option, komplett bei
der bestehenden Lösung zu bleiben, kurz betrachtet.

\subsection{Weiternutzung des Indico-Frameworks}
Getreu dem Motto \enquote{never change a running system} stellt sich natürlich die Frage, ob sich
der nicht zu unterschätzende Aufwand, eine komplexe Software wie Indico auf ein neues Framework
umzustellen, überhaupt lohnt. Diese Frage ist im Fall von Indico aus verschiedenen Gründen zu
bejahen. Insbesondere die Kombination aus einem sehr einfachen Routingsystem und dem
URL-Dateisystem-Mapping im CGI-Stil bietet bei der Entwicklung neuer Features diverse Nachteile:
Sofern das neue Feature eine für den Benutzer direkt zugängliche Seite besitzt ist aus Gründen der
Einheitlichkeit eine \emph{*.py}-URL angebracht. Im Falle eines neuen Moduls wie beispielsweise der
HTTP-API ist dies jedoch nicht gegeben, weshalb sich dort anbietet, ein neues Pfadsegment
einzuführen und über dieses alle Anfragen in das entsprechende Modul zu routen, was jedoch bedeutet,
dass jegliches weitere Routing von dem jeweiligen Modul übernommen werden muss.

Ein enormer Vorteil bei der bestehenden Lösung ist natürlich, dass sie seit Jahren relativ stabil
läuft und von einer breiten Userbasis benutzt wird und damit die \enquote{Kinderkrankheiten} eines
neuen Systems nicht vorhanden sind. Ebenfalls ist beispielsweise die ZODB-Datenbank stark in das
aktuelle System integriert, wobei anzumerken ist, dass dabei teilweise fast derselbe Code an
mehreren Stellen verwendet wird: Die \lstinline{RH}-Klassen und die JSON-RPC-API verwenden jeweils
dasselbe Retry-System, um Datenbankkonflikte zu behandeln, jedoch ist es in den jeweiligen Klassen
unabhängig voneinander implementiert.

Ein Vorteil beim Verzicht auf ein neues Framework wäre natürlich auch, die dafür notwendige
Entwicklungszeit anderweitig nutzen zu können. Dies ist jedoch nur kurzfristig gesehen ein Vorteil,
da die Nichtnutzung eines modernen Frameworks in Zukunft bei neuen Features mit hoher
Wahrscheinlichkeit zu deutlich mehr Entwicklungsaufwand führt als mit einem entsprechenden Framework
notwendig wäre.

Insbesondere aufgrund dieser Folge ist ein neues Framework also unbedingt notwendig. Daher wird die
Option, weiterhin ausschließlich das Indico-Framework, in den folgenden Abschnitten nicht mehr
betrachtet.


\subsection{Vollständige Migration zu Django}
Aufgrund der Struktur von Django - mehrere möglichst in sich abgeschlossene Apps - und der
empfohlenen Verzeichnisstruktur in den einzelnen Apps ist es schwierig, Django in die bestehende
Indico-Codebasis zu integrieren. Es ist deutlich einfacher und auch sauberer, das Django-Projekt in
einem neuen Verzeichnis unabhängig von der bestehenden Codebasis zu integrieren. Danach müssen
zunächst die in der gesamten Anwendung genutzten Funktionen wie beispielsweise das Benutzer- und
Gruppensystem samt LDAP-Anbindung unter Verwendung der Django-User-App neu implementiert werden.
Obwohl beispielsweise das bestehende Benutzersystem nicht Teil des Frameworks ist und durchaus in
das Django-Projekt übernommen werden könnte, hat die Neuimplementierung den Vorteil, dass die
Django-Infrastruktur im weiteren Verlauf möglichst effizient benutzt werden kann. Da die Benutzer in
der ZODB abgespeichert werden, muss diese Datenbank auch in Django integriert werden; dazu bietet
sich möglicherweise \emph{Django-ZODB} an - dies muss jedoch genauer untersucht werden, da unter
anderem Datenbankkonflikte während eines Commits sauber abgefangen werden sollten. Eine Alternative
wäre natürlich der Umstieg auf eine relationale Datenbank unter Verwendund des ORM-Systems von
Django, allerdings würde dies die bereits sehr aufwändige Migration noch komplexer machen.

Sobald der Kern der Django-Version von Indico funktionsfähig ist, kann die eigentliche Anwendung
migriert werden. Da Indico mit den \lstinline{RH}-Klassen bereits klassenbasierte Views nutzt und
oftmals auch Vererbung nutzt, bietet es sich in den meisten Fällen an, auch weiterhin Klassen statt
einfacher Funktionen zu nutzen. Während die meiste Anwendungslogik dieser Klassen übernommen werden
kann, sind dennoch in fast allen Fällen Änderungen angebracht, um beispielsweise über die von Django
bereitgestellten Variablen auf HTTP-Parameter und dynamische Elemente aus der URL zuzugreifen.
Zusammen mit dem Umstieg auf die Django-Templateengine führt dies also dazu, dass hunderte von
Klassen und Templates angepasst werden müssen. Letzteres kann teilweise automatisiert werden, die
Klassen müssen jedoch manuell umgeschrieben werden.

Für das URL-Routing müssen entsprechende Regeln in Form regulärer Ausdrücke geschrieben werden.
Indico besitzt ca. 700 verschiedene URLs, die zwar oftmals dieselben Parameter wie die ID des Events
nethalten und damit dank der \lstinline{include()}-Funktion des Routingsystems vereinfacht werden
können, aber dennoch manuell in \emph{clean URLs} umgeschrieben werden müssen - die Entscheidung,
beispielsweise statt \emph{/conferenceTimetable.py/pdf} die URL \emph{/event/123/timetable.pdf} zu
nutzen, kann keine Software übernehmen. Zusätzlich zu den neuen URLs müssen für alle direkt
verlinkbaren Seiten auch die alten URLs weiterhin unterstützt werden, um beispielsweise Links von
anderen Seiten oder Suchmaschinen nicht in einer Fehlermeldung enden zu lassen. Da Indico bisher die
\lstinline{URLHandler}-Klassen zur URL-Generierung nutzt, existiert eine zentrale Stelle, die
angepasst werden muss, um die neuen URLs nicht nur zu unterstützen sondern auch überall, wo Indico
eine URL generiert, auch zu erzeugen.

Zusammenfassend kann man sagen, dass es sich bei dieser Migration um ein extrem aufwändiges und
zeitintensives Unterfangen handelt, das schon eher ein Rewrite als eine Migration ist - insbesondere
wenn man es mit einem Wechsel von der bestehenden Objektdatenbank auf eine relationale Datenbank.
kombiniert. Am Ende führt der Umstieg jedoch zu sauberem Code, der sowohl leicht wartbar ist als
auch für neue Entwickler einen einfachen Einstieg ermöglicht, da viele Python-Entwickler bereits mit
Django vertraut sind.


\subsection{Erweiterung durch Django}
Obwohl es wie bereits erwähnt schwierig ist, Django in die bestehende Codebasis von Indico zu
integrieren, ist es natürlich prinzipiell möglich, sofern man die von den Django-Entwicklern
empfohlene Anwendungsstruktur ignoriert. In diesem Fall würde man kein Grundgerüst mit
\emph{django-admin.py} generieren, sondern die entsprechenden Django-Objekte bzw. -Funktionen direkt
in bestehenden oder neuen Dateien innerhalb des Indico-Packages importieren. Dazu bietet sich
beispielsweise eine neues Subpackage \lstinline{indico.web.django} an, welches den gesamten
Django-spezifischen Basiscode enthält. Der WSGI-Code des alten Indico-Frameworks in
\lstinline{indico.web.wsgi} kann in jedem Fall entfernt werden, wobei dementsprechend natürlich auch
das stark auf diesem Code basierende Session-System entfernt und durch das von Django
bereitgestellte System ersetzt werden muss. Danach können entweder die im \emph{mod\_python}-Stil
geschriebenen Funktionen über das Routingsystem von Django aufgerufen werden, oder aber die
\lstinline{RH}-Klassen entsprechend angepasst werden, um direkt von Django aufgerufen werden zu
können. Durch die vollständige Weiterverwendung dieser Klassen würde der ZODB-Zugriff vom
bestehenden Code übernommen werden. Für GET/POST-Daten und URL-Parameter muss natürlich eine
entsprechende Kompatibilitätsschicht entwickelt werden, die sie in das bisher genutzte Format
konvertiert. Durch die Weiterverwendung des Mako-Templatesystems kann ebenfalls viel
Entwicklungsaufwand gespart werden, wobei eine externe Templateengine natürlich bei weitem nicht so
gut mit dem Framework zusammenspielt wie die frameworkeigene Engine.

Letzendlich lässt sich bei diesem Migrationspfad sehr viel Entwicklungszeit einsparen, allerdings
stellt sich die frage, welchen Nutzen man neben den schöneren URLs am Ende hat. Immerhin handelt es
sich bei Django um ein sehr mächtiges und komplexes Framework, welches man bei dieser Lösung nur zu
einem Bruchteil nutzen würde - und ein mit Django vertrauter Entwickler müsste sich dennoch erst
einarbeiten, da fast keine der bei Django üblichen \emph{Best Practices} genutzt werden könnten.


\subsection{Erweiterung durch Flask}\label{migration-flask-partial}
Derzeit nutzt Indico den von Werkzeug bereitgestellten Python-Webserver um bei der Entwicklung auch
ohne Apache und \emph{mod\_wsgi} betrieben werden zu können. Daher bietet es sich an, das auf
Werkzeug basierende Flask-Microframework in Indico zu integrieren. Neben der in einem späteren
Abschnitt beschriebenen Komplettmigration ist insbesondere eine partielle Migration bzw. Erweiterung
der bestehenden Codebasis auf Flask interessant. Zu Beginn bietet es sich an, den WSGI-Kern und
darauf aufbauenden Code - insbesondere also des Session-System - des Indico-Frameworks durch Flask
zu ersetzen, jedoch beispielsweise die einwandfrei funktionierende Datenbankanbindung beizubehalten.
Das Entfernen des WSGI-Kerns ist ein guter Ausgangspunkt, da dadurch garantiert obsolet werdender
Code entfernt wird. Nachdem dies geschehen und Flask integriert ist, ist zu prüfen, ob noch weitere
Kernbestandteile von Indico von dem alten WSGI-Code abhängig sind und angepasst bzw. ersetzt werden
müssen. Danach kann zunächst ein einfacher Wrapper das Routing im \emph{mod\_python}-Stil emulieren
und somit Indico mithilfe des Flask-WSGI-Backends größtenteils lauffähig machen.

Die wichtigste von Flask bereitgestellte Neuerung ist aber natürlich das Routingsystem kombiniert
mit den Modularität von Blueprints. Neben mit geringem bis mittlerem Aufwand integrierbaren Features
wie dem \lstinline{g}-Objekt und dem integrierten Debugger ist das URL-Routing die komplexeste
Flask-basierte Erweiterung, und somit auch die aufwändigste. Allerdings ist auf \emph{Rules} und
\emph{Endpoints} basierende Struktur sehr ähnlich mit der derzeit in Indico genutzten URL-Erzeugung,
sodass gute Chancen bestehen, dass abgesehen vom Definieren der neuen URLs viele Aufgaben in diesem
Bereich automatisiert werden können. Gerade im Hinblick auf die Kompatibilität mit Links auf die
alten URLs ist dies sehr hilfreich.

Der Vorteil darin, Flask zu nutzen, aber nicht rigoros jedes von Flask oder einer Flask-Erweiterung
gebotene Feature sofort einzuführen, liegt insbesondere in der relativ einfachen Migration - neben
der unabhängig vom Framework aufwändigen Umstellung der URLs müssen weder zwangsläufig Templates
konvertiert noch alle View-Klassen angepasst werden. Gleichzeitig kann neuer Code aber durchaus
zusätzliche Flask-Features nutzen, sodass auch nach der Migration die Codequalität von Indico
langfristig immer besser wird, solange neue Features implementiert oder bestehende refactored
werden.


\subsection{Vollständige Migration zu Flask}
Bei einer Komplettmigration bietet die in \autoref{migration-flask-partial} beschriebene
Teilmigration einen guten Ausgangspunkt, da am Ende ein funktionierendes Indico steht, welches
bereits an verschiedenen Stellen Flask-Features nutzt. Die verbleibenden Bereiche - insbesondere
Templates und \lstinline{RH}-Klassen - können zwar relativ unabhängig voneinander migriert werden,
allerdings müssen in beiden Fällen diese Klassen bzw. ihre Nachfolger angepasst werden, weshalb es
sinnvoll ist, diese schrittweise zusammen zu migrieren, d.h. jeweils eine Klasse und die von dieser
genutzten Templates. Die Templatemigration kann größtenteils automatisiert werden, allerdings
enthalten einige Templates Python-Code, der manuell entfernt bzw. in die entsprechende Python-Klasse
verschoben werden muss. Bei den weiteren Änderungen an den Klassen gibt es mehrere Möglichkeiten,
Dinge wie die Datenbankverbindung und die dazugehörigen Transaktionen entsprechend zu behandeln.
Neben einem \emph{Class Decorator} wäre auch eine Superklasse denkbar, die ähnlich wie derzeit die
\lstinline{RH}-Klasse die entsprechende Logik implementiert, aber dabei entsprechend flexibel bleibt
und Flask ideal nutzt.

Da es sich bei dieser Variante um eine Erweiterung der Einbindung von Flask in die bestehende
Infrastruktur handelt, macht es wenig Sinn, diese zu Beginn in Betracht zu ziehen. Während sie
logischerweise zum besseren Code führt ist sie extrem aufwändig und bietet insbesondere im Hinblick
auf neue Entwicklungen nur wenige Vorteile, da diese unabhängig davon bereits alle neuen Features
nutzen können.

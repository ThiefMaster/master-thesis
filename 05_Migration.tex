\chapter{Migration zu Flask}
Bei der Migration zu Flask sind einige Schritte am Anfang zwingend notwendig, während andere
optional sind und in relativ beliebiger Reihenfolge ausgeführt werden können. Im Folgenden
werden die Migrationsschritte in der Reihenfolge beschrieben, wie sie durchgeführt wurden.


\section{Vorbereitung}
In der Vorbereitungsphase wird Flask eingebunden und lauffähig konfiguriert. Sollten dabei jedoch
Konflikte auftreten, werden diese bereits - möglichst unter Nutzung von Features des Frameworks -
behoben. Es findet zu diesem Zeitpunkt jedoch noch keine Migration statt.


\subsection{Installieren von Flask}
Um einem Python-Projekt eine neue Abhängigkeit hinzuzufügen gibt es grundsätzlich zwei
Möglichkeiten. Insbesondere für Python-Libraries und andere über PyPi\footnote{Python Package Index,
\href{https://pypi.python.org}{https://pypi.python.org}} veröffentlichte Packages nutzt man eine
Datei \emph{setup.py}, in der unter anderem die Funktion \lstinline{setup()} aufgerufen wird, an die
neben diversen anderen Metadaten über den Parameter \lstinline{install_requires} die zur
Installation benötigten Pakete übergeben werden. Dies ermöglicht es Paketmanagern, die notwendigen
Abhängigkeiten automatisch zu installieren. Die Alternative zur \emph{setup.py} ist die
\emph{requirements.txt}-Datei. Diese enthält jeweils ein Paket pro Zeile, wobei neben dem Paketnamen
auch eine Version oder ein Verweis auf ein Versionskontrollsystem wie beispielsweise Git angegeben
werden kann. Meist wird die \emph{requirements.txt} bei Python-Anwendungen benutzt, die nicht über
PyPi installiert sondern manuell heruntergeladen werden und weitere Konfiguration benötigen. In
diesem Fall enthalten die Installationsanweisungen meist den Hinweis, die Abhängigkeiten mit den
Befehl \emph{pip install -r requirements.txt} zu installieren.

Da bei Indico ursprünglich vorgesehen war, es systemweit zu installieren und nur die
Konfigurationsdaten und sonstige dynamische Dateien bzw. Verzeichnisse außerhalb des systemweiten
Python-Verzeichnisses abzulegen, besitzt es sowohl eine \emph{setup.py} als auch eine
\emph{requirements.txt}. Daher muss Flask an beiden Stellen als Abhängigkeit definiert werden und
danach installiert werden. Dies kann entweder manuell mittels \emph{pip install Flask} geschehen
oder wie zuvor erwähnt über die \emph{requirements.txt}. Letzteres hat den Vorteil, dass Tippfehler
o.ä. direkt auffallen.

\begin{lstlisting}[caption=Auszug aus der requirements.txt von Indico,label=lst:indicoreqtxt]
git+https://github.com/miracle2k/webassets.git
Werkzeug==0.9
Flask==0.10
\end{lstlisting}

\autoref{lst:indicoreqtxt} zeigt einen Ausschnit aus der \emph{requirements.txt}, wobei unter
anderem Flask und das zugrundeliegende Werkzeug-Toolkit in der gerade aktuellen Version eingebunden
werden. Flask selbst setzt zwar bereits \lstinline{'Werkzeug>=0.7'} voraus, allerdings erlaubt dies
dem Paketmanager, eine beliebige Version ab 0.7 zu installieren. Meist ist dies kein Problem, da
gute Libraries API-Inkompatibilitäten möglichst vermeiden, allerdings ist es sicherer, dennoch von
allen genutzten Abhängigkeiten die Version festzulegen, sodass ein Benutzer exakt dieselben
Versionen nutzt, mit denen die Anwendung auch getestet wurde.

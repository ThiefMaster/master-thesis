\chapter{Migration zu Flask}
Bei der Migration zu Flask sind einige Schritte am Anfang zwingend notwendig, während andere
optional sind und in relativ beliebiger Reihenfolge ausgeführt werden können. Im Folgenden
werden die Migrationsschritte in der Reihenfolge beschrieben, wie sie durchgeführt wurden.


\section{Vorbereitung}
In der Vorbereitungsphase wird Flask eingebunden und lauffähig konfiguriert. Sollten dabei jedoch
Konflikte auftreten, werden diese bereits behoben. Hierzu können u.U. Features des Frameworks
benutzt werden.
Es findet zu diesem Zeitpunkt jedoch noch keine Migration statt. Das Endziel dieser Phase
ist ein temporärer Wrapper, der die \emph{mod\_python}-Funktionen über Flask aufruft, sodass
grundsätzliche Inkompatibilitäten zwischen Flask und dem Indico-Code zu Beginn der nächsten Phase
beseitigt werden können.


\subsection{Installieren von Flask}
Um einem Python-Projekt eine neue Abhängigkeit hinzuzufügen gibt es grundsätzlich zwei
Möglichkeiten. Insbesondere für Python-Libraries und andere über PyPi\footnote{Python Package Index,
\href{https://pypi.python.org}{https://pypi.python.org}} veröffentlichte Packages nutzt man eine
Datei \emph{setup.py}, in der unter anderem die Funktion \lstinline{setup()} aufgerufen wird, an die
neben diversen anderen Metadaten über den Parameter \lstinline{install_requires} die zur
Installation benötigten Pakete übergeben werden. Dies ermöglicht es Paketmanagern, die notwendigen
Abhängigkeiten automatisch zu installieren. Die Alternative zur \emph{setup.py} ist die
\emph{requirements.txt}-Datei. Diese enthält jeweils ein Paket pro Zeile, wobei neben dem Paketnamen
auch eine Version oder ein Verweis auf ein Versionskontrollsystem wie Git angegeben
werden kann. Meist wird die \emph{requirements.txt} bei Python-Anwendungen benutzt, die nicht über
PyPi installiert sondern manuell heruntergeladen werden müssen und weitere Konfiguration benötigen.
In diesem Fall enthalten die Installationsanweisungen meist den Hinweis, die Abhängigkeiten mit den
Befehl \emph{pip install -r requirements.txt} zu installieren.

Da bei Indico ursprünglich vorgesehen war, es systemweit zu installieren und nur die
Konfigurationsdaten und sonstige dynamische Dateien bzw. Verzeichnisse außerhalb des systemweiten
Python-Verzeichnisses abzulegen, besitzt es sowohl eine \emph{setup.py} als auch eine
\emph{requirements.txt}. Daher muss Flask an beiden Stellen als Abhängigkeit definiert werden und
danach installiert werden. Dies kann entweder manuell mittels \emph{pip install Flask} geschehen
oder wie zuvor erwähnt über die \emph{requirements.txt}. Letzteres hat den Vorteil, dass Tippfehler
o.ä. direkt auffallen.

\begin{lstlisting}[caption=Auszug aus der requirements.txt von Indico,label=lst:indicoreqtxt]
git+https://github.com/miracle2k/webassets.git
Werkzeug==0.9
Flask==0.10
\end{lstlisting}

\autoref{lst:indicoreqtxt} zeigt einen Ausschnitt aus der \emph{requirements.txt}, wobei unter
anderem Flask und das zugrundeliegende Werkzeug-Toolkit in der gerade aktuellen Version eingebunden
werden. Flask selbst setzt zwar bereits \lstinline{'Werkzeug>=0.7'} voraus, allerdings erlaubt dies
dem Paketmanager, eine beliebige Version ab 0.7 zu installieren. Meist ist dies kein Problem, da
gute Libraries API-Inkompatibilitäten möglichst vermeiden. Allerdings ist es sicherer, dennoch von
allen genutzten Abhängigkeiten die Version festzulegen, sodass ein Benutzer exakt dieselben
Versionen nutzt, mit denen die Anwendung auch getestet wurde.


\subsection{Einbinden von Flask}

Indico enthält zwei Python-Packages: \lstinline{MaKaC} und \lstinline{indico}, wobei
\lstinline{MaKaC} alten Legacy-Code und \lstinline{indico} hauptsächlich neuen Code enthält.
Web-spezifischer Code ist im Paket \lstinline{indico.web}, weshalb es sich anbietet, für Flask ein
neues Package \lstinline{indico.web.flask} anzulegen. Um den Flask-Code innerhalb dieses Pakets
bereits im Hinblick auf die Migration und zukünftige Entwicklungen zu strukturieren, wird ein Modul
\emph{app.py} angelegt, welches die \emph{Application Factory} und den zum Initialisieren der App
notwendigen Code enthält.

Bei der \emph{Application Factory} handelt es sich um eine Funktion, die eine Flask-Instanz
erstellt, konfiguriert und zurückgibt. Diese Kapselung hat den Vorteil, dass jegliche Initialisierung an
einigen wenigen Stellen stattfindet und z.B. für Unittests eine Instanz mit anderen
Konfigurationsoptionen erstellt werden kann. Ohne Nutzung einer Factory-Funktion würde die
Flask-Instanz zur Importzeit des Moduls erstellt werden und jedes andere Modul könnte das Objekt von
dort importieren und verändern. Zur \enquote{Laufzeit}, also während der Abarbeitung eines Requests,
macht sich der Unterschied nicht mehr bemerkbar, da Flask mit \lstinline{current_app} ein
Proxy-Objekt bereitstellt, welches jeweils auf die gerade aktive Flask-Instanz verweist.

Derzeit wird der in Indico integrierte Webserver mittels \emph{indico\_shell --web-server}
gestartet. Zu Beginn ist es sinnvoll, über einen weiteren Kommandozeilenparameter zwischen Flask und
dem bestehenden Webserver samt dem alten Framework wählen zu können. Aufgrund des einfachen Aufbaus
von WSGI muss dazu nur entweder das bestehende \emph{application-Callable} oder das neue, von
\lstinline{make_app()} erstellte, an den Webserver übergeben werden. Beim Aktualisieren des
dafür zuständigen Codes fällt auf, dass dort eine einfache WSGI-Middleware implementiert wurde, die
die WSGI-Applikation über den Indico-Konfiguration angegebenen Pfad verfügbar macht - also
\emph{http://localhost:8000/indico} statt nur \emph{http://localhost:8000}. Ein Blick
in die Werkzeug-Dokumentation zeigt, dass dort mit \lstinline{DispatcherMiddleware} eine fertige
Middleware enthalten ist, die diese Aufgabe ebenfalls übernehmen kann - allerdings mit weniger
benutzerdefiniertem Code. Da es sich dabei um eine sehr einfache Änderung handelt und Werkzeug
sowieso bereits eingebunden war, bietet es sich an, diese bereits zu diesem Zeitpunkt vorzunehmen.
Dazu dient die in \autoref{lst:indico-dispatcher} gezeigte Funktion.

\begin{lstlisting}[caption=Indico-WSGI-Dispatcher,label=lst:indico-dispatcher]
def make_indico_dispatcher(wsgi_app):
    baseURL = Config.getInstance().getBaseURL()
    path = urlparse.urlparse(baseURL)[2].rstrip('/')
    if not path:
        return wsgi_app
    else:
        return DispatcherMiddleware(NotFound(), {
            path: wsgi_app
        })
\end{lstlisting}

Sie akzeptiert als Parameter eine WSGI-Anwendung - also entweder die Flask-Instanz oder die alte
Indico-WSGI-Applikation und gibt eine Anwendung zurück, die die übergebene Anwendung in dem
konfigurierten Pfad verfügbar macht und für alle anderen Pfade einen \emph{404 Not Found}-Fehler an
den Client sendet. Sofern kein Pfad konfiguriert ist gibt die Funktion direkt die App zurück, da die
Middleware in diesem Fall nicht benötigt wird.

An dieser Stelle enthält Indico jetzt bereits eine lauffähige Flask-Anwendung, der man bereits zum
ersten Testen eine Hello-World-Funktion hinzufügen könnte. Allerdings kann die Anwendung
ausschließlich über den nicht für Produktionszwecke geeigneten Entwicklungswebserver gestartet
werden. Um sie auch über einen WSGI-kompatiblen externen Webserver nutzen zu können, wird eine
\emph{indico.wsgi}-Datei benötigt, die die Anwendung unter dem Namen \lstinline{application}
zugänglich macht.

\begin{lstlisting}[caption=indico.wsgi]
from indico.web.flask.app import make_app
application = make_app()
\end{lstlisting}


\subsection{Interface zu den mod\_python-Funktionen}\label{oldmpinterface}

Beim \emph{mod\_python}-Routing verweisen URLs der Form \emph{/file.py} bzw. \emph{/file.py/func}
auf eine gleichnamige Datei und entweder die angegebene Funktion oder auf eine Funktion namens
\lstinline{index}. Der erste Parameter ist jeweils das Request-Objekt, danach folgen alle GET- und
POST-Parameter als Keyword Arguments. Der Rückgabewert der Funktion wird als Antwort-Body an den
Client gesendet.

Um ein Interface zu diesen Funktionen zu entwickeln, existieren verschiedene Möglichkeiten, die
jedoch alle auf einer Art Reflection basieren. Neben komplexen, auf dem Python-Syntaxbaum
aufbauenden, Varianten, ist die einfachste Lösung, die Dateien zu importieren und dann aus ihrem
globalen Namespace alle aufrufbaren Objekte auszufiltern. Dies ist in einer dynamischen Sprache wie
Python sehr einfach möglich: Mit \lstinline{execfile(path, globals)} bietet Python eine Funktion,
die die angegebene Datei ausführt und dabei das als \lstinline{globals} übergebene \lstinline{dict}
für alle \emph{globals}, d.h. die im aktuellen Modul verfügbaren globalen Variablen, nutzt. Nach dem
Aufruf dieser Funktion muss der Inhalt der Liste also nur noch gefiltert werden, um nicht
ausführbare Objekte zu entfernen, und enthält danach nur noch die relevanten Funktionen. Aus
Dateiname und Funktionsname kann nun mithilfe einfacher Stringoperationen eine entsprechende
Routingregel für Flask erzeugt werden.

Anders als \emph{mod\_python} übergibt Flask jedoch keine Parameter an die Viewfunktionen, sofern
die Routingregel keine solchen enthält. Daher ist eine Wrapperfunktion notwendig, welche die
eigentliche Funktion mit den richtigen Parametern aufruft. Da es sich bei Funktionen in Python um
First-Class-Objekte handelt, die ja wie in \autoref{firstclass} erwähnt zur Laufzeit erstellt und
von anderen Funktionen zurückgegeben werden können, ist dies wie man in
\autoref{lst:flask-mp-wrapper} erkennen kann sehr einfach möglich.

\begin{lstlisting}[caption=mod\_python-Wrapper-Factory,label=lst:flask-mp-wrapper]
def create_flask_mp_wrapper(func):
    def wrapper():
        args = request.args.copy()
        args.update(request.form)
        flat_args = {}
        for key, item in args.iterlists():
            flat_args[key] = map(_to_utf8, item) if len(item) > 1 else _to_utf8(item[0])
        return func(None, **flat_args)
    return wrapper
\end{lstlisting}

Beim Aufruf dieser Funktion mit der \emph{mod\_python}-Funktion als Parameter wird eine neue
Funktion definiert und zurückgegeben, die die GET- und POST-Parameter in einem einzigen Objekt
zusammenfasst und alle Werte in UTF8 konvertiert. Dies ist nötig, da Flask Unicodestrings verwendet,
während Indico intern grundsätzlich mit UTF8-Strings arbeitet. Neben \lstinline{None}, dem
Python-Äquivalent von dem aus anderen Sprachen bekannten \lstinline{NULL}, statt des nicht mehr
verfügbaren Request-Objekts übergibt die neue Funktion zum Schluss die neue Parameterliste in der
von Indico erwarteten Form an die Originalfunktion.


\section{Migration}

\subsection{Inkompatibilitäten beheben}
Da statt eines (emulierten) \emph{mod\_python}-Request-Objekts nun \lstinline{None} übergeben wird,
schlagen Zugriffe auf Methoden wie \lstinline{req.get_remote_ip()} fehl. Flask macht die IP des
Benutzers über \lstinline{request.remote_addr} zwar ebenfalls verfügbar, allerdings handelt es sich
dabei grundsätzlich um die IP, von der aus die Verbindung zum Webserver aufgebaut wurde. Dies ist
zwar bei der Entwicklung und in einfachen Setups die richtige IP-Adresse, allerdings nutzt
z.B. die Indico-Installation am CERN einen Loadbalancer, der die IP-Adresse des Clients
über den HTTP-Header \emph{X-Forwarded-For} übergibt. Flask macht alle in diesem Header enthaltenen
IP-Adressen über \lstinline{request.access_route} zugänglich. Da die IP-Adresse des Loadbalancers
jedoch nie relevant ist und der Header, sofern kein Loadbalancer verwendet wird, vom Benutzer
manipuliert sein kann, ist eine neue Eigenschaft sinnvoll, die jeweils die korrekte IP-Adresse
zurückgibt. Natürlich könnte dies einfach gelöst werden, indem man die bereits in Indico enthaltene
Funktion \lstinline{_get_remote_ip(req)} anpasst, d.h. den Parameter entfernt und auf die Flask-APIs
zugreift. Flask ist jedoch flexibel genug, um eine deutlich sauberere Lösung zu implementieren.
Durch Subclassing der Klasse \lstinline{flask.wrappers.Request} kann die Funktion, die von der
\lstinline{remote_addr}-Eigenschaft genutzt wird, durch eine eigene ersetzt werden, die sofern ein
Proxy vorhanden ist \lstinline{self.access_route[0]} und ansonsten die
\lstinline{remote_addr}-Eigenschaft der Superklasse nutzt. Um die benutzerdefinierte
\lstinline{IndicoRequest}-Klasse in Flask auch zu verwenden, wird einfach \lstinline{Flask}
subclassed und die \lstinline{request_class}-Klasseneigenschaft überschrieben. Alternativ könnte es
auch erst in der durch \lstinline{make_app()} erstellten Instanz überschrieben werden.

Ebenfalls mit Flask inkompatibel ist das Sessionsystem. Dies war aufgrund der starken Integration
mit der \emph{mod\_python}-Emulation zu erwarten und es sollte sowieso durch eine auf Flask
basierende Neuentwicklung ersetzt werden. Da Flask standardmäßig nur auf signierten Cookies
basierende Sessions unterstützt, ist die dort speicherbare Datenmenge auf ca. 4096 Bytes beschränkt.
Ebenfalls treten insbesondere im Zusammenhang mit den in der JSON-RPC-API massiv genutzten AJAX-Requests,
oftmals Probleme auf. Daher muss ein neues Sessioninterface entwickelt werden. Aufgrund des in
Indico bereits vorhandenen Cache-Systems bietet es sich an, dieses zum Speichern der Sessiondaten zu
verwenden. Gerade bei der empfohlenen Nutzung von Memcached oder Redis als Cache-Backend hat dies
den Vorteil, dass für jeden Datensatz eine Time-To-Live angegeben werden kann und er nach Ablauf
dieser Zeit automatisch gelöscht wird. In Flask eingebunden wird das Sessioninterface ähnlich wie
die benutzerdefinierte \lstinline{IndicoRequest}-Klasse: Die Klasse \lstinline{Flask} enthält eine
Eigenschaft, der man einfach eine Instanz der neuen Interfaceklasse zuweist. Das von dieser Klasse
zu implementierende Interface ist dabei sehr simpel. Zu Beginn eines Requests wird die Methode
\lstinline{open_session(app, request)} aufgerufen und gibt das Sessionobjekt zurück, welches im
Anwendungscode dann über den \lstinline{session}-Proxy verfügbar ist. Nach erfolgreicher
Verarbeitung des Requests kann die Methode \lstinline{save_session(app, session, response)} die
Sessiondaten speichern. Aus Performancegründen sollte die Session nur aktualisiert werden, wenn sie
noch nie gespeichert wurde und Daten enthält, verändert wurde oder die Daten bzw. das Cookie bald
ablaufen.

Das Sessioninterface ist jedoch ausschließlich für das Laden und Speichern der Session
verantwortlich. Es enthält selbst keinerlei Sessiondaten. Diese sind in einer separaten Klasse,
welche die Dict-API von Python implementiert und das Objekt nach jeder Änderung als \emph{dirty}
markiert. Die ursprüngliche Sessionimplementierung unterstützt neben einer Dict-ähnlichen API auch
diverse Funktionen wie \lstinline{getUser()}, die den aktuellen Benutzer zurückgeben. Da die neuen
Sessions nicht mehr in der ZODB gespeichert werden, kann dort nicht direkt das Benutzerobjekt
referenziert werden sondern ausschließlich die ID des Benutzers gespeichert werden. Dies macht es
allerdings unmöglich, die Dict-API ohne größeren Aufwand dafür zu nutzen. Stattdessen werden diese
speziellen Eigenschaften als \emph{Properties} implementiert, also Objekteigenschaften, die intern
auf eine Getter- und Setterfunktion zugreifen. Dies ermöglicht es, beim Zugriff auf
\lstinline{session.user} den entsprechenden Benutzer anhand der in der Session gespeicherten ID aus
der Datenbank zu laden, für weitere Zugriff lokal zu cachen, und dann zurückzugeben.

Um das neue Sessionsystem zu nutzen, wird die alte Implementierung vollständig entfernt. Dies ist
problemlos möglich, da \lstinline{MaKaC.webinterface.session} keinen weiteren Code enthält und somit
komplett gelöscht werden kann. Danach müssen alle darauf verweisenden Importe und natürlich der
Code, der die alte Session initialisiert und im \lstinline{RH}-Objekt ablegt, entfernt werden.
Zugriffe auf explizit implementierte Funktionen der Session - also \lstinline{getUser()} und
ähnliche Methoden - lassen sich relativ leicht mit der Suchfunktion des Editors finden und auch fast
automatisch durch ihre neue Variante in der Form \lstinline{session.user} ersetzen. Nur Importe und
Funktionsparameter müssen manuell angepasst bzw. entfernt werden. Sessionzugriffe über
\lstinline{getVar()} bzw. \lstinline{setVar()} könnten zwar prinzipiell mit regulären
Ausdrücken in die Dict-Syntax umgeschrieben werden, jedoch könnten in diesem Fall weder direkt in
der Session abgelegte Datenbankobjekte noch die direkte Veränderung von bereits in der Session
vorhandenen Daten gesondert behandelt werden. Ersteres ist jedoch prinzipiell nicht möglich und
führt zu einem Fehler beim Laden der Session und letzteres würde die geänderten Daten nicht im
Cache-Backend aktualisieren, da ein Container nicht ohne massive Performanceeinbußen feststellen
kann, ob ein enthaltenes Objekt verändert wurde. Daher werden alle weiteren Sessionzugriffe
manuell angepasst. Dies geschieht allerdings noch nicht zu diesem Zeitpunkt, da es für einen
allgemeinen Funktionstest von Indico nicht notwendig ist, dass jedes Feature fehlerfrei
funktioniert.

Erste Tests zeigen an dieser Stelle, dass Indico nun schon prinzipiell auf Flask-Basis funktioniert.
Allerdings schlagen alle Requests fehl, die eine im Dateisystem existierende Datei an den Client
senden sollen, da dort \emph{mod\_python}-spezifische Funktionen genutzt werden statt der
komfortablen \lstinline{send_file()}-Funktion von Flask. Diese Stellen lassen sich jedoch leicht
finden und anpassen.


\subsection{Trial and Error}
An dieser Stelle ist es sinnvoll, alle einfach zugänglichen Bereiche von Indico kurz zu testen um
weitere Probleme zu finden und zu beheben. Idealerweise könnte dies automatisiert durch die
Unittests und funktionalen Tests geschehen, allerdings decken die Unittests hauptsächlich intern
genutzten, nicht-web-spezifischen Code ab. Kombiniert mit sehr wenigen funktionalen Tests stellt
sich schnell heraus, dass die vorhandenen Tests hier nicht wirklich hilfreich sind. Daher existieren
zwei Optionen: Manuell testen oder entsprechend viele Testcases schreiben, um die ganze Software
abzudecken. Dass letzteres langfristig gesehen die bessere Lösung ist steht außer Frage, allerdings
muss hier aus Zeitgründen auf manuelle Tests zurückgegriffen werden.

Die Tests zeigen, dass unter anderem die HTTP-API und JSON-RPC-API nicht funktionieren, da sie das
bei der Beschreibung des Indico-Frameworks erwähnte auf dem ersten Pfadsegment basierende Routing
nutzen und dementsprechend vom zuvor entwickelten Wrapper nicht abgedeckt werden. Dies kann jedoch
durch manuelles Hinzufügen der entsprechenden Routingregeln und geringfügige Modifikationen im davon
aufgerufenen Code leicht behoben werden.

Ebenfalls in dieser Phase behoben werden Zugriffe auf das alte, nicht mehr vorhandene,
Sessionobjekt. Dabei stellt sich heraus, dass das Room-Booking-Plugin extrem viele separate
Einträge in der Session abspeichert, die sinnvoller in einem Dict zusammengefasst werden könnten. Da
dieses Refactoring den Aufwand nur minimal erhöht wird es zusammen mit den sowieso notwendigen
Änderungen durchgeführt.

Das Vorgehen nach dem \emph{Trial-and-Error}-Prinzip hat hier den Vorteil, dass jeweils nur einzelne
zusammengehörende Bereiche von Indico aktualisiert werden und sie direkt nach der Änderung auf
Funktionalität getestet werden können.


\subsection{Routing}

In \autoref{oldmpinterface} wurde ein einfaches Interface zu den \emph{mod\_python}-Funktionen
entwickelt, um möglichst schnell ein lauffähiges System zu haben. Im Hinblick auf saubere URLs und
Kompatibilität mit alten URLs ist es jedoch sinnvoller, die Routingregeln nicht erst zur Laufzeit zu
generieren, damit die alten Python-Dateien, die alle exakt dieselbe Struktur wie das Beispiel in
\autoref{lst:mpabout} haben, gelöscht werden können.

\begin{lstlisting}[caption=about.py im htdocs-Verzeichnis,label=lst:mpabout]
from MaKaC.webinterface.rh import about as rhAbout

def index(req, **params):
    return rhAbout.RHAbout(req).process(params)
\end{lstlisting}


\subsubsection{Generieren der Routingregeln}\label{routingrulegen}

Um die benötigten Daten aus den Python-Dateien zu extrahieren, existieren mehrere Möglichkeiten.
Mittels regulärer Ausdrücke ließen sich sowohl die Importe als auch die Funktions- und Klassennamen
relativ einfach extrahieren. Allerdings müssten Sonderfälle wie der im Beispiel genutzte
Import-Alias entsprechend behandelt werden und auch unterschiedliche Formatierung entweder zuvor
vereinheitlicht oder durch einen komplexeren regulären Ausdruck abgedeckt werden. Da es sich bei
Python um eine extrem dynamische Sprache handelt, kann jedoch auch der \emph{Abstract Syntax Tree}
der jeweiligen Funktion genutzt werden, um an die benötigten Daten zu gelangen. Dieser Syntaxbaum
enthält eine abstrakte Repräsentation des Quellcodes und ist sehr viel einfacher zu parsen als der
nur in Stringform vorliegende Quellcode selbst.

\img{ast.png}{400px}{Abstract Syntax Tree}{Abstract Syntax Tree}

Relevant sind um beim Beispiel aus \autoref{lst:mpabout} zu bleiben neben dem bereits bekannten
Funktionsnamen \lstinline{index}, der zur Sicherheit mit dem Namen in \emph{FunctionDef} verglichen
werden kann, der globale Name \lstinline{rhAbout} des Moduls und der Klassenname
\lstinline{RHAbout}. Diese beiden Werte sind wie man in der \autoref{img.ast.png} leicht erkennt vom
inneren \emph{Call} aus über \lstinline{func.value.id} und \lstinline{func.attr} erreichbar. Über
die Globals des mittels \lstinline{execfile()} ausgeführten Python-Moduls kann das zu
\lstinline{rhAbout} gehörende Python-Modul referenziert und der korrekte Name ausgelesen werden.

Damit sind alle notwendigen Daten zum Generieren einer Routingregel verfügbar. Insbesondere ist die
Kombination aus Datei- und Funktionsname eindeutig  und kann damit als Endpoint im Routingsystem
benutzt werden. Da Funktionsnamen keine Bindestriche enthalten können und keiner der Dateinamen
einen solchen enthält, ist der Dateiname ohne Extension bzw. für eine Funktion, die nicht
\emph{index} heißt, der String \emph{Dateiname-Funktionsname} als Endpoint geeignet, anhand dessen
auch wieder die ursprüngliche URL erzeugt werden kann. Dies ist im späteren Verlauf für die
Kompatibilitätsschicht hilfreich.


\subsubsection{Generieren von URLs}

Indico nutzt zum Generieren von URLs Subklassen der \lstinline{URLHandler}-Klasse, die jeweils eine
Eigenschaft mit der relativen URL enthalten. Diese enthalten eine \emph{classmethod}
\lstinline{getURL()}, die optionale Parameter für GET-Argumente akzeptiert und ein URL-Objekt
zurückgibt, welches ebenfalls Methoden besitzt, um GET-Argumente hinzuzufügen oder zu entfernen.
In einem String-Kontext genutzt gibt das Objekt mithilfe der magischen Methode \lstinline{__str__}
die auf der \lstinline{_relativeURL} basierende absolute URL zurück.

\begin{lstlisting}[caption=URLHandler für about.py]
class UHAbout(URLHandler):
    _relativeURL = 'about.py'
\end{lstlisting}

Man könnte nun denken, dass es sehr einfach ist, die URLHandler-Klasse anzupassen, sodass statt der
relativen URL der Flask-Endpoint übergeben wird und die URL-Instanz die mittels
\lstinline{url_for()} zum Endpoint gehörende URL erhält. Mit den bestehenden Routingregeln
funktioniert dies auch ohne Schwierigkeiten. Problematisch wird es allerdings, sobald eine Regel eine
Variable in der URL enthält. In diesem Fall schlägt \lstinline{url_for()} fehl, sofern die Variable
nicht übergeben wurde. Da die in Indico jedoch nicht immer alle Variablen an \lstinline{getURL()}
übergeben werden sondern teilweise erst über die Methoden der URL-Klasse hinzugefügt werden, darf
\lstinline{url_for()} erst aufgerufen werden, wenn die URL auch als String benötigt wird. Dies lässt
sich dadurch lösen, dass der \lstinline{URLHandler} den Endpoint und die bereits bekannten Parameter
an die URL-Klasse weitergibt, und diese die URL erst dann generiert, wenn sie als String benötigt
wird. Der Nachteil bei dieser Lösung ist, dass Tracebacks im Fehlerfall nicht sehr hilfreich sind:
Oftmals wird die URL erst innerhalb eines Templates in einen String konvertiert und die
Mako-Templateengine zeigt in Tracebacks nicht den im Template vorhandenen Code an sondern den aus
dem Template generierten Python-Code.

Ein weiteres Problem zeigt sich in clientseitgem JavaScript-Code, der URLs generieren muss. Zuvor
konnten dort einfach Parameter mittels Stringoperationen angehängt werden. Dies ist jedoch
nicht mehr möglich, da die URL ohne alle notwendigen Parameter überhaupt nicht generiert werden
kann. Die Lösung dazu liegt im Package \lstinline{werkzeug.contrib.jsrouting}. Dabei handelt es sich
um eine JavaScript-Implementierung des URL-Generators von Flask/Werkzeug. Während diese nicht direkt
in Indico nutzbar ist, da eine Liste mit allen ca. 700 Routingregeln und später nochmal ebenso vielen
Kompatibilitätsregeln die an den Browser gesendeten Daten unnötig aufblähen würde, bietet es eine
sehr gute Basis für eine vereinfachte Version, die nicht die komplette Routingtabelle clientseitig
verfügbar macht, sondern nur einige ausgewählte Regeln. Implementiert ist dies mit einer Eigenschaft
der URL-Klasse, die die notwendigen Informationen der Regel als JSON zurückgibt und einer auf dem
jsrouting-Code basierenden JavaScript-Funktion, die aus der JSON-Regel und den Parametern die URL
generiert.

Um die URL-Generierung von Flask nun tatsächlich zu verwenden müssen alle URLHandler-Subklassen
entsprechend angepasst werden. Dies kann praktischerweise mit einem regulären Ausdruck automatisiert
werden. Da die generierten Legacy-Routingregeln alle in einem Blueprint namens \emph{legacy}
enthalten sind, müssen auch die Endpoints in den URLHandler-Klassen auf diesen Blueprint verweisen.

\begin{lstlisting}[caption=Endpointbasierter URLHandler für about.py]
class UHAbout(URLHandler):
    _endpoint = 'legacy.about'
\end{lstlisting}


\subsubsection{Clean URLs}

Eines der Hauptziele der Migration, neben saubererem Code und dem damit verbundenen Komfort für
Entwickler, ist die Nutzung sauberer URLs. Vor der Implementation macht es Sinn, sich Gedanken über
den Aufbau dieser URLs zu machen. Ein oftmals in diesem Kontext genanntes Paradigma ist
REST\footnote{Representational state transfer}. Dabei handelt es sich um die oftmals fälschlich als
Standard bezeichnete Idee, dass eine URL genau eine Ressource bzw. Collection repräsentiert und
verschiedene HTTP-Methoden unterschiedliche Aktionen darauf ausführen. Beispiel würde \emph{/event/}
die Liste aller Events repräsentieren während \emph{/event/123} das Event mit der ID 123
repräsentieren würde. Insbesondere aufgrund der Verwendung verschiedener HTTP-Methoden wie DELETE
und PUT eignet sich REST eher für APIs als für den von Benutzern direkt zugänglichen Bereich einer
Webanwendung. Außerdem wäre es bei einer Anwendung wie Indico sehr umständlich, alle Ressourcen im
REST-Stil zu repräsentieren. Deutlich einfacher und benutzerfreundlicher ist es, die URL-Struktur
nach Kategorien aufzuteilen, sodass \emph{/event/123/} auf die Startseite des Events verweist,
\emph{/event/123/timetable} auf den Zeitplan und \emph{/event/123/timetable.pdf} auf die PDF-Version
von selbigem. Der Administrationsbereich der Anwendung ist komplett unabhängig davon und somit über
\emph{/admin/} erreichbar.

Die eigentliche Implementierung der sauberen URLs ist prinzipiell einfach, aber sehr zeitaufwändig,
da jede URL manuell definiert werden muss. Die generierten Legacy-Routen vereinfachen die Aufgabe
jedoch etwas, da nach dem Erstellen der Flask-Blueprints nur die entsprechenden Legacy-Routen via
Cut\&Paste im entsprechend Modul eingefügt und die URL und die erlauben HTTP-Methoden angepasst
werden. Da die Endpoints denselben Namen wie im Legacy-Blueprint haben muss in den dazugehörigen
URLHandler-Klassen nur der Name des Blueprints angepasst werden.


\subsubsection{Unterstützung der alten URLs}

Für bestehende Indico-Installationen wäre es extrem problematisch, wenn nach dem Update auf eine
neuere Indico-Version alle alten URLs in einer \emph{404 Not Found}-Fehlermeldung enden würden.
Daher müssen die alten URLs ebenfalls unterstützt werden; allerdings sollten sie nicht auf die
ursprüngliche Zielseite verweisen, sondern mit einem \emph{301 Moved Permanently}-Redirect auf die
neue URL weiterleiten.

Aufgrund der in \autoref{routingrulegen} gewählten Endpoint-Namen sind keine weiteren Informationen
notwendig, um aus diesem Namen die ursprüngliche URL zu generieren. Daher wird einfach eine Funktion
implementiert, die zu einem existierenden Blueprint einen Kompatibilitäts-Blueprint
generiert, der die alten URLs nutzt und auf die neuen weiterleitet. Der einzige zu beachtende
Sonderfall ist bei POST-Requests. Diese werden laut HTTP-Standard bei einer Weiterleitung mit GET
wiederholt, was nicht gewollt ist. Daher muss in diesem Fall auf die Weiterleitung verzichtet werden
und die Originalfunktion direkt aufgerufen werden. Da normale Links und Suchmaschinen jedoch
ausschließlich GET verwenden, ist dies kein Problem.


\section{Probleme bei der Migration}

Die Migration zu Flask lief größtenteils unproblematisch ab, wobei gerade bei einer Anwendung dieser
Größe immer einige kleinere Schwierigkeiten auftreten. Diese beruhten jedoch meist auf unerwarteten
Sonderfällen wie alten \emph{mod\_python}-Dateien, die Debugcode enthielten und somit beim Import
unerwartete Nebeneffekte hatten.

Ein weiterer ursprünglich nicht bedachter Sonderfall machte sich beim Testen von Indico mit dem
\emph{nginx}-Webserver bemerkbar: Bei statischen Dateien, die im Dateisystem existieren, ist es
deutlich effizienter, wenn der Webserver die Datei ausliefern kann, als wenn die Webanwendung dies
tun muss. Dazu unterstützen die meisten Webserver den \emph{X-Sendfile}-Header, der den Pfad zu der
Datei enthält, die der Webserver an den Client senden soll. Aus Sicherheitsgründen unterstützt nginx
jedoch nur \emph{X-Accel-Redirect}, der sich zwar fast identisch verhält, allerdings statt eines
Dateisystempfads eine URL erwartet, die in der Webserverkonfiguration dann auf einen entsprechenden
Pfad gemappt wird. Dank des Middleware-Systems von WSGI ließ sich dieses Problem jedoch leicht
lösen, indem eine neue Middleware den \emph{X-Sendfile}-Header entfernt und mithilfe einer zuvor
konfigurierten Mappingtabelle den Pfad in eine URL konvertiert und diese im
\emph{X-Accel-Redirect}-Header weiterreicht.


\section{Ausblick}

Bei der Entwicklung neuer Features kann Flask deutlich stärker genutzt werden als es beim
bestehenden Code der Fall ist. Insbesondere bei der Generierung von URLs macht es Sinn, direkt auf
\lstinline{url_for()} zuzugreifen, statt den Umweg über eine neue URLHandler-Subklasse zu gehen.
Darüber hinaus kann strenger darauf geachtet werden, dass GET und POST korrekt eingesetzt werden.
Ein normaler Seitenaufruf, der keinerlei Daten verändert, sollte grundsätzlich GET nutzen während
die Veränderung von Daten ausschließlich mit POST möglich sein sollte. Dies kann durch das
Routingsystem von Indico einfach durchgesetzt werden, allerdings hätte es im bestehenden Code
oftmals Änderungen an den HTML-Templates benötigt, da insbesondere im Administrationsbereich häufig
Formularbuttons zur Navigation verwendet werden. Für Fälle, in denen sowohl GET als auch POST
benötigt werden, also hauptsächlich auf Seiten, die Formulare enthalten, ist es jedoch aufgrund
der Architektur von Indico sinnvoller, dies innerhalb der \lstinline{RH}-Klasse zu tun, da ansonsten
im Anwendungscode manuell mittels \lstinline{request.method == 'POST'} unterschieden werden müsste,
welcher Codezweig gerade ausgeführt werden soll. Die Lösung dafür findet sich in der Klasse
\lstinline{MethodView} von Flask. Dabei handelt es sich um eine Klasse, die eine Methode mit
demselben Namen wie das verwendete HTTP-Verb aufruft. Im Kontext von Indicos \lstinline{RH}-Klassen
würde also entweder \lstinline{_process_GET()} oder \lstinline{_process_POST()} aufgerufen.

Mittelfristig könnte man auch nach einer Alternative zu den \lstinline{RH}-Klassen zu suchen, um in
neuem Code weniger Zwischenschichten zu haben. Voraussetzung dafür ist aber, dass alle derzeit von
der \lstinline{RH}-Klasse übernommenen Aufgaben anderweitig gelöst werden können. Insbesondere die
oftmals überschriebene Methode zum Überprüfen, ob der Benutzer die notwendigen Berechtigungen hat,
könnte sich dabei als problematisch erweisen. Typischerweise werden in Flask dafür Decorators
genutzt, was aber nur Sinn macht, wenn dieselbe Logik für möglichst viele Funktionen verwendet
werden kann.

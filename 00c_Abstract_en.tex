\chapter*{Abstract}
\thispagestyle{empty}

The goal of this master thesis is to extend the \emph{Indico} software developed at CERN with a
freely available Python web framework. Such a framework allows more efficient development than
possible by using the current framework that has been developed in-house over multiple years.
Indico is a web application to plan and manage meetings, conferences and similar events. Besides
managing those, it also allows management and reservation of conference rooms.

At first the used technologies Python and WSGI are introduced. Then both the current,
Indico-specific, framework and various other frameworks are introduced and analyzed according to
certain criteria. Based on this analysis the advantages and disadvantages of migrating to one of
these frameworks are analyzed and a framework is chosen. By using this framework parts of Indico
will be migrated or modified.

The solution developed in this thesis is meant to be a basis for using commonly used well-documented
\emph{third party} code and a maintainable developer-friendly system, which will additionally be
more user- and search-engine-friendly than the current version.

Another goal besides the modernization of the framework is developing and integrating a simple
\emph{recommendation engine} for categories and/or events so users can see possibly interesting or
relevant things easily.

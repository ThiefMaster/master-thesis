\section{Flask}

\emph{Flask}\footnote{\href{http://flask.pocoo.org}{http://flask.pocoo.org}} ist ein relativ
leichtgewichtiges Microframework, dessen Entwicklung 2010 begann und derzeit in Version 0.10
verfügbar ist. Neben Python 2.6 und 2.7 unterstützt die aktuellste Version des Frameworks auch
Python 3.3.

Der Begriff \enquote{Microframework} bedeutet bei Flask, dass das Framework dem Entwickler die
größtmögliche Flexibilität lässt, welche Technologien er für verschiedene Bestandteile seiner
Anwendung nutzt.

Kern von Flask ist das WSGI-Tookit \emph{Werkzeug}, welches die Lowlevel-Funktionen für das
WSGI-Interface und alle HTTP-spezifischen Utilityfunktionen bereitstellt. Auch das URL-Routingsystem
ist Teil von Werkzeug, wobei Flask die High-Level-APIs dazu bereitstellt und dadurch den in einem
Framework erwarteten Komfort bietet.


\begin{description}
\item[Modularität] \hfill \\
Flask ermöglicht modulare Anwendungen mithilfe von \emph{Blueprints}. Diese verhalten sich ähnlich
wie eine vollwertige Flask-Anwendung, allerdings fehlt die gesamte für eine Flask-Anwendung
notwendige Logik - stattdessen werden Blueprints einer Anwendung hinzugefügt, und fügen die
enthaltenen Elemente der Anwendung hinzu. Im Gegensatz zu mehreren kleineren Anwendungen haben
Blueprints den Vorteil, dass sie weder für die anwendungsweite Fehlerbehandlung noch für Dinge wie
die Datenbankverbindung zuständig sind, sondern all diese Dinge von der Anwendung selbst übernehmen.
Ein Blueprint kann Templates, statische Daten, URL-Routingdaten und Viewfunktionen enthalten.
Darüberhinaus ist es möglich, Exceptions auf Blueprintebene abzufangen statt sie grundsätzlich an
das Errorhandling der Anwendung selbst weiterzureichen.

Allerdings hat das Blueprint-System auch einen Nachteil: Dadurch, dass es sich gerade nicht um eine
vollwertige Anwendung handelt, eignen sie sich nur bedingt dazu, komplett wiederverwendbare Module
zu entwickeln, die auch von Dritten \emph{as-is} verwendet werden können. Diesem Problem kann jedoch
bei Bedarf entgegengewirkt werden, indem man entsprechende Voraussetzungen an die Anwendung stellt,
die den Blueprint nutzen soll, oder problematische Teile entsprechend abstrahiert. Beispielsweise
könnte eine Klasse, die auf Benutzerdaten zugreift, abstrakte Methoden enthalten, die in der
jeweiligen Anwendung dann überschrieben werden müssen.


\item[URL-Routing] \hfill \\
\todotext{URL-Routing}


\item[Template-Engine] \hfill \\
Flask nutzt standardmäßig die \emph{Jinja2}-Templateengine, die auch unabhängig von Flask verfügbar
ist. Die Templatesyntax ist größtenteils mit der von Django identisch und auch die Features sind
sehr ähnlich. Flask stellt diverse Funktionen - insbesondere \lstinline{url_for()} zum Generieren
von URLs - und u.a. die Request- und Sessionobjekte in allen Templates zur Verfügung und escaped
dynamische Daten in HTML-Templates automatisch, sofern es nicht explizit deaktiviert wird.

Um Jinja2 anzupassen - sei es mit benutzerdefinierten Templatefiltern oder zusätzlichen globalen
Funktionen - stellt Flask im Applikationsobjekt entsprechende Funktionen zur Verfügung, die jeweils
auch als Decorator verwendet werden können.

Jinja2 steht in einer Flask-Anwendung immer zur Verfügung; es handelt sich dabei um eine feste
Abhängigkeit des Frameworks, die auch nicht deaktiviert werden kann. Der Sinn dahinter ist, dass
Flask-Erweiterungen die Engine immer nutzen können und nicht Templates für verschiedene Engines
mitliefern müssen.

Da der Zugriff auf Templates in Flask jedoch über die im \lstinline{flask}-Package definierte
Funktion \lstinline{render_template()} geschieht, die jeweils explizit aufgerufen werden muss, steht
es jedem Entwickler frei, in seiner Anwendung eine andere Templateengine zu nutzen. Allerdings muss
er in diesem Fall selbst darauf achten, Dinge wie Autoescaping zu konfigurieren und die
Flask-Objekte falls benötigt in Templates verfügbar zu machen. Für die \emph{Mako}-Templateengine
gibt es jedoch bereits eine Flask-Extension, die Mako sauber in Flask integriert. Insbesondere wird
exakt derselbe Templatekontext verwendet, der auch an Jinja2 übergeben wird. Dies hat den Vorteil,
dass der entsprechende Decorator in Flask weiterhin benutzt werden kann.


\item[Datenbankanbindung] \hfill \\
\todotext{Datenbankanbindung}


\item[Sessions] \hfill \\
\todotext{Sessions}


\item[Caching] \hfill \\
\todotext{Caching}


\item[Integrierbarkeit] \hfill \\
\todotext{Integrierbarkeit}


\item[Erweiterbarkeit] \hfill \\
\todotext{Erweiterbarkeit}


\item[Sonstige Features] \hfill \\
\todotext{Sonstiges}


\item[Dokumentation] \hfill \\
Flask, Werkzeug und Jinja2 besitzen jeweils eine sehr ausführliche Online-Dokumentation, die sich
jeweils auf die aktuellste Version bezieht. Features, die erst in einer bestimmten Version
hinzugefügt wurde, sind entsprechend markiert, sodass die Dokumentation auch bei Nutzung einer
älteren Version noch hilfreich ist. Darüberhinaus sind für jeden Versionssprung Updatehinweise
verfügbar, die explizit auf potenziell problematische Änderungen hinweisen.

Ein großer Teil der ausführlichen API-Dokumentation ist anhand der im Code verwendeten Docstrings
generiert. Dies hat den Vorteil, dass beim Lesen des Codes die relevante Dokumentation direkt
sichtbar ist. Kommentare im Code sind wie im Python-Styleguide empfohlen an allen Stellen vorhanden,
die nicht selbsterklärend sind.


\item[Lizenz] \hfill \\
Flask, Werkzeug und die übrigen zwingend notwendigen Python-Libraries stehen steht unter der
BSD-Lizenz. Für Flask-Erweiterungen ist eine ähnlich freizügige Lizenz empfohlen bzw. im Falle von
\emph{approved extensions} sogar zwingend notwendig. Die offizielle Dokumentation empfiehlt entweder
die BSD-Lizenz, die MIT-Lizenz oder die größtenteils dem amerikanischen Public-Domain-Konzept
entsprechende WTFPL\footnote{Do What the Fuck You Want to Public License}.
\end{description}

\section{Flask}

\emph{Flask}\footnote{\href{http://flask.pocoo.org}{http://flask.pocoo.org}} ist ein relativ
leichtgewichtiges Microframework, dessen Entwicklung 2010 begann und derzeit in Version 0.10
verfügbar ist. Neben Python 2.6 und 2.7 unterstützt die aktuellste Version des Frameworks auch
Python 3.3.

Der Begriff \enquote{Microframework} bedeutet bei Flask, dass das Framework dem Entwickler die
größtmögliche Flexibilität lässt, welche Technologien er für die einzelnen Bestandteile seiner
Anwendung nutzt.

Kern von Flask ist das WSGI-Toolkit \emph{Werkzeug}, das die Lowlevel-Funktionen für das
WSGI-Interface und alle HTTP-spezifischen Utilityfunktionen bereitstellt. Auch das URL-Routingsystem
ist Teil von Werkzeug, wobei Flask die High-Level-APIs dazu bereitstellt und dadurch den in einem
Framework erwarteten Komfort bietet.


\begin{description}
\item[Modularität] \hfill \\
Flask ermöglicht modulare Anwendungen mithilfe von \emph{Blueprints}. Diese verhalten sich ähnlich
wie eine vollwertige Flask-Anwendung, allerdings fehlt die gesamte für eine Flask-Anwendung
notwendige Logik. Stattdessen fügen Blueprints ihre Bestandteile der Anwendung hinzu, sobald sie bei
der Anwendung registriert werden. Im Gegensatz zu mehreren kleineren Anwendungen haben
Blueprints den Vorteil, dass sie weder für die anwendungsweite Fehlerbehandlung noch für
Funktionalität wie die Datenbankverbindung zuständig sind, sondern all diese Dinge von der Anwendung
selbst übernehmen.
Ein Blueprint kann Templates, statische Daten, URL-Routingdaten und Viewfunktionen enthalten.
Darüber hinaus ist es möglich, Exceptions auf Blueprintebene abzufangen statt sie grundsätzlich an
das Errorhandling der Anwendung selbst weiterzureichen.

Allerdings hat das Blueprint-System auch einen Nachteil: Dadurch, dass es sich gerade nicht um eine
vollwertige Anwendung handelt, eignen sie sich nur bedingt dazu, komplett wiederverwendbare Module
zu entwickeln, die auch von Dritten \emph{as-is} verwendet werden können. Diesem Problem kann jedoch
bei Bedarf entgegengewirkt werden, indem man entsprechende Voraussetzungen an die Anwendung stellt,
die den Blueprint nutzen soll, oder problematische Teile entsprechend abstrahiert. Eine Klasse, die
auf Benutzerdaten zugreift, könnte z.B. abstrakte Methoden enthalten, die in der jeweiligen
Anwendung dann überschrieben werden müssen.


\item[URL-Routing] \hfill \\
Wie jedes moderne Framework setzt auch Flask auf saubere URLs. Dazu nutzt es das von Werkzeug
bereitgestellte URL-Routing-System und erweitert es um eine einfachere API, die insbesondere bei der
Registrierung von Routingregeln mittels Decorators und beim Generieren von URLs ihre Vorteile zeigt.

Die einfachste Möglichkeit, eine Routingregel hinzuzufügen, ist mit dem Decorator
\lstinline{app.route}. Dieser akzeptiert als Parameter die Routingregel und optional die erlaubten
HTTP-Methoden. Die damit dekorierte Funktion wird daraufhin automatisch mit der Regel verknüpft und
der Funktionsname als Endpoint benutzt. Natürlich sind in einer größeren Anwendung nur wenige URLs
komplett statisch, sondern enthält auch diverse Variablen. Da es sich bei diesen in den meisten
Fällen um einfache Strings oder Zahlen handelt, werden diese durch einfache Platzhalter in der Form
\lstinline{'<type:name>'} bzw. bei Strings \lstinline{'<name>'} angegeben statt über oftmals eher
kryptische reguläre Ausdrücke. Standardmäßig unterstützt Flask die Typen \lstinline{string},
\lstinline{int}, \lstinline{float} und \lstinline{path}, wobei es möglich ist, eigene Typen zu
definieren indem man eine Subklasse von \lstinline{BaseConverter} definiert und darin die Methoden
\lstinline{to_url} und \lstinline{to_python} implementiert. Intern nutzt das Routingsystem reguläre
Ausdrücke mit denen man allerdings nur in Kontakt kommt, wenn man einen eigenen Typkonverter
definiert. Die Übergabe der Parameter an die Viewfunktion erfolgt über Keyword Arguments. Auch
optionale Parameter sind möglich; in diesem Fall muss die Funktion mindestens eine Routingregel
besitzen, die den entsprechenden Parameter nicht enthält.

Sofern Decorators keine Option sind - z.B. weil der Code über viele Dateien verstreut ist
und man die Routingregeln an einem zentralen Ort haben will - bietet Flask die Methode
\lstinline{app.add_url_rule}. Diese entspricht prinzipiell dem Decorator, wobei die Viewfunktion
ebenfalls als Parameter übergeben wird.

Neben dem Pfad und dem HTTP-Verb kann das Routingsystem auch die aktuelle Subdomain berücksichtigen,
sowohl als festen Wert als auch als dynamische Variable. Für komplexere Fälle kann auch direkt auf
die \lstinline{Map} des Werkzeug-Routingsystems zugegriffen werden statt die Flask-API zu nutzen.
Dies ist jedoch gerade dank Blueprints nur in den wenigstens Fällen notwendig.

Um über die Routingtabelle eine URL zu generieren, benötigt man den Namen des Endpoints der
entsprechenden Routingregel. Übergibt man diesen an die Funktion \lstinline{url_for}, kann diese
eine URL daraus generieren. Standardmäßig ist diese relativ zur aktuellen Domain, also in der Form
\emph{/foo/bar}, allerdings akzeptiert die Funktion neben den Keyword Arguments für die Variablen
der Routingregel auch die speziellen Argumente \lstinline{_external} um eine vollständige URL zu
generieren, \lstinline{_schema} um das Protokoll in einer solchen absoluten URL festzulegen und
\lstinline{_anchor} um das URL-Fragment (\emph{\#something}) anzugeben. Sofern mehrere Regeln für
denselben Endpoint existieren, verwendet \lstinline{url_for} automatisch die passendste Regel.

Blueprints besitzen dieselben Methoden; der einzige relevante Unterschied ist, dass ein Blueprint
einen Präfix besitzen kann, der jeder Routingregel des Blueprints hinzugefügt wird. Bei der Nutzung
von \lstinline{url_for()} wird ein Blueprint in der Form \lstinline{'blueprint.endpoint'} angegeben,
wobei auch relative Verweise der Form \lstinline{'.endpoint'} möglich sind; in diesem Fall wird der
zum Zeitpunkt des Aufrufs aktive Blueprint benutzt.


\item[Templateengine] \hfill \\
Flask nutzt standardmäßig die \emph{Jinja2}-Templateengine, die auch unabhängig von Flask verfügbar
ist. Die Templatesyntax ist größtenteils mit der von Django identisch und auch die Features sind
sehr ähnlich. Flask stellt diverse Funktionen - insbesondere \lstinline{url_for()} zum Generieren
von URLs - und u.a. die Request- und Sessionobjekte in allen Templates zur Verfügung und escaped
dynamische Daten in HTML-Templates automatisch, sofern es nicht explizit deaktiviert wird.

Um Jinja2 anzupassen - sei es mit benutzerdefinierten Templatefiltern oder zusätzlichen globalen
Funktionen - stellt Flask im Applikationsobjekt entsprechende Funktionen zur Verfügung, die jeweils
auch als Decorator verwendet werden können.

Jinja2 steht in einer Flask-Anwendung immer zur Verfügung; es handelt sich dabei um eine feste
Abhängigkeit des Frameworks, die auch nicht deaktiviert werden kann. Der Sinn dahinter ist, dass
Flask-Erweiterungen die Engine immer nutzen können und nicht Templates für verschiedene Engines
mitliefern müssen.

Da der Zugriff auf Templates in Flask jedoch über die im \lstinline{flask}-Package definierte
Funktion \lstinline{render_template()} geschieht, die jeweils explizit aufgerufen werden muss, steht
es jedem Entwickler frei, in seiner Anwendung eine andere Templateengine zu nutzen. Allerdings muss
er in diesem Fall selbst darauf achten, Dinge wie Autoescaping zu konfigurieren und die
Flask-Objekte falls benötigt in Templates verfügbar zu machen. Für die \emph{Mako}-Templateengine
gibt es jedoch bereits eine Flask-Extension, die Mako sauber in Flask integriert. Insbesondere wird
exakt derselbe Templatekontext verwendet, der auch an Jinja2 übergeben wird. Dies hat den Vorteil,
dass der entsprechende Decorator in Flask weiterhin benutzt werden kann. Ebenfalls via Extension
unterstützt wird die XML-basierte Templateengine \emph{Genshi}.

\newpage
\item[Datenbankanbindung] \hfill \\
Flask selbst nutzt keine Datenbank und enthält, Microframework-typisch, keinen Datenbankcode. Es
exisiert mit \emph{Flask-SQLAlchemy} jedoch eine offizielle Flask-Erweiterung, die das
SQLAlchemy-ORM-System in Flask integriert und dabei denselben Komfort bietet, der auch bei allen
anderen Features von Flask üblich ist. Neben den ORM-Features kann alternativ auch nur die
Datenbankabstraktionsschicht von SQLAlchemy genutzt werden.

Mit \emph{Flask-MongoKit}, \emph{Flask-PyMongo} und \emph{Flask-MongoAlchemy} existieren auch
verschiedene Erweiterungen, um die NoSQL-Datenbank \emph{MongoDB} in Flask zu integrieren. Der Grund
für die drei verschiedenen Extensions ist, dass je nach Anwendung ein höherer oder niedrigerer
Abstraktionsgrad bei der Datenbank-API gewünscht ist.

Auch für die Objektdatenbank \emph{ZODB} existiert eine Flask-Erweiterung, die genau wie
\emph{Flask-SQLAlchemy} den \emph{approved extension}-Status hat und somit den Design Guidelines von
Flask folgt, entsprechend lizenziert ist und eine entsprechend hochwertige Dokumentation besitzt.


\item[Sessions] \hfill \\
Flask stellt über das \lstinline{session}-Objekt ein spezielles \lstinline{dict} bereit, welches
sich für den Entwickler wie ein normales \lstinline{dict} verhält, jedoch Veränderungen automatisch
registriert. Daher kann es in der Regel ohne Weiteres modifiziert werden und die Daten werden
automatisch in der Session abgespeichert. Neben den Standardmethoden enthält das Objekt diverse
Eigenschaften, um den Status der Session auszulesen bzw. zu verändern. \lstinline{new} und
\lstinline{modified} geben an, ob die Session noch nie gespeichert wurde bzw. ob sie seit dem
letzten Speichern verändert wurde. Die \lstinline{permanent}-Eigenschaft setzt die Ablaufzeit des
Session-Cookies; standardmäßig ist es nur bis zum Schließen des Browsers gültig.

Da Flask weder eine Datenbank noch einen Cache voraussetzt, werden Sessions standardmäßig
clientseitig als signiertes Cookie gespeichert. Dies ist eine sehr einfache aber oftmals
ausreichende Lösung, sofern man weder geheime noch größere Daten abspeichern will, da Cookies
oftmals auf 4096 Bytes beschränkt sind und die Signatur zwar vor Veränderungen schützt, nicht jedoch
vor Auslesen.

Aus Sicherheitsgründen werden Sessiondaten standardmäßig als \emph{JSON} serialisiert statt das
mächtigere \emph{Pickle} zu nutzen. Während die Signatur die Sessiondaten schützt, würde die Nutzung
von Pickle im Falle eines Leaks des geheimen Schlüssels nicht nur die Manipulation der Sessiondaten
ermöglichen sondern auch das Ausführen beliebigen Codes.

Sofern cookiebasierte Sessions nicht ausreichen, kann das \lstinline{session_interface} des
Anwendungsobjekts auf eine benutzerdefinierte Klasse verweisen, die die Sessiondaten z.B. in Redis
oder einer SQL-Datenbank abspeichert. Das Interface ist dabei sehr einfach gehalten; eine
Redis-basierende Implementierung ist mit weniger als 70 Zeilen Code möglich.

Standardmäßig enthält Flask jedoch nur das Cookie-Interface, allerdings hat der Maintainer von Flask
mit dem \lstinline{RedisSessionInterface} ein entsprechendes Interface für serverseitige Sessions
als Erweiterung veröffentlicht.


\item[Caching] \hfill \\
Flask selbst enthält keinen Cache, allerdings bietet Werkzeug eine Lowlevel-Abstraktion
verschiedener Cache-Backends, und stellt dabei die Methoden \lstinline{get}, \lstinline{set},
\lstinline{delete} sowieso entsprechende \lstinline{_many}-Varianten davon zur Verfügung. Dieses
Cache-Interface ist oftmals ausreichend, allerdings bietet die Erweiterung \emph{Flask-Cache} auch
eine über die zentrale Flask-Konfiguration konfigurierbare High-Level-API dafür, die neben den
Lowlevel-Funktionen Decorators zum Cachen ganzer Views und ein Jinja2-Plugin zum Cachen einzelner
Templatefragmente.

Neben diesen relativ üblichen Cache-APIs bietet \emph{Flask-Cache} einen
\lstinline{memoize}-Decorator. Dieser kann zum Cachen beliebiger Funktionen genutzt werden; er nutzt
den Namen der Funktion und die übergebenen Parameter als Cache-Key und führt die eigentliche
Funktion nur aus, wenn das Ergebnis noch nicht im Cache vorhanden ist.


\item[Integrierbarkeit] \hfill \\
Flask benötigt weder eine bestimmte Datenbank noch setzt es eine bestimmte Verzeichnisstruktur in
der Anwendung voraus. Während URL-Routingregeln meist über Decorators definiert werden, so ist die
Nicht-Decorator-API genauso mächtig und ähnlich komfortabel. Daher ist es sehr einfach, Flask in
eine existierende Anwendung zu integrieren.

Selbstverständlich kann eine größere Anwendung, in der Flask nachträglich integriert wurde, nicht
alle Vorteile des Frameworks ausnutzen ohne dass der gesamte Code modifiziert werden muss.
Nichtsdestotrotz ist Flask sehr flexibel und so ist es kein Problem, bestehende Views funktionsfähig
in Flask einzubinden und neuen Code im \enquote{Flask-Stil} zu schreiben.


\item[Erweiterbarkeit] \hfill \\
Flask bietet mit \emph{Signals} eine API, um Callbacks zu registrieren, die zu verschiedenen
Zeitpunkten ausgeführt werden. Diese sind unter anderem das Rendern eines Templates und vor bzw.
nach einem Request. Sofern dies nicht ausreicht, kann die \lstinline{Flask}-Klasse subclassed
werden. Dies bietet die Möglichkeit, bestehende Funktionalität zu erweitern bzw. zu ersetzen und
z.B. die \lstinline{Request}-Klasse durch eine eigene Subklasse davon zu ersetzen.

Flask-Extensions nutzen in der Regel ausschließlich die Callbacks, sodass sie einfach zu einer
Anwendung hinzugefügt werden können und unabhängig von anderen Extensions sind. Um eine Extension zu
benutzen, sind zwei Methoden verbreitet. Für einfache Extensions, die keine eigene für den
Entwickler relevante Klasse besitzen, wird empfohlen, eine Funktion \lstinline{init_app(app)} zu
definieren. Diese akzeptiert als ersten Parameter die Flask-Anwendung und registriert relevante
Callbacks und führt sonstigen Initialisierungscode aus. Komplexere Extensions enthalten in der Regel
eine Klasse, über die vom Entwickler auf die Extension zugegriffen werden kann. Dabei wird das
Applikationsobjekt über einen optionalen Parameter an den Konstruktor übergeben; sofern er nicht
genutzt wird, sollte die Klasse die zuvor erwähnte Methode \lstinline{init_app(app)} besitzen. Der
Zweck davon ist, die Nutzung von \emph{application factories} zu ermöglichen, d.h. Funktionen, die
die Applikation erst erstellen und damit möglicherweise nie global verfügbar machen. In diesem Fall
ist es möglicherweise notwendig, die Instanz der Extensionklasse zu erstellen, obwohl noch keine
Flask-Instanz existiert.

Sofern keine dieser Optionen mächtig genug sind, besteht auch die Möglichkeit, eine der
Werkzeug-Klassen durch eine eigene Subklasse zu ersetzen um beispielsweise das URL-Routing-System
direkt verändern zu können. Wie bei jedem Framework besteht darüber hinaus die Möglichkeit, das
Framework zu forken und daraufhin den Frameworkcode selbst zu verändern.


\item[Sonstige Features] \hfill \\
Flask nutzt Proxy-Objekte um \lstinline{request}, \lstinline{session} und \lstinline{g} global
verfügbar zu machen und trotzdem jeweils - auch bei mehreren Threads - auf das richtige Objekt zu
verweisen. Dadurch müssen diese Objekte niemals als Funktionsparameter übergeben werden, sondern
können einfach aus dem \lstinline{flask}-Package importiert werden.

Während \lstinline{request} und \lstinline{session} ziemlich selbsterklärend sind, handelt es sich
bei \lstinline{g} um ein von Flask selbst nicht genutztes Objekt, welches allerdings sowohl
Anwendungscode als auch Flask-Erweiterungen eine Möglichkeit bietet, requestspezifische Daten
abzulegen. Beispielsweise ist der aktuell eingeloggte User meist in \lstinline{g.user} gespeichert
und oftmals nutzen Datenbank-Extensions das Objekt, um die gerade aufgebaute Datenbankverbindung
beim nächsten Zugriff nicht erneut aufbauen zu müssen.

Für viele Standardfeatures existieren bereits fertige Erweiterungen, sodass man diese nicht selbst
implementieren muss. Neben komplexen Erweiterungen wie einem Django-ähnlichen dynamischen
CRUD-Administrationsbereich, ORM-Systemen und einer kompletten Benutzerverwaltung samt Rechtesystem
gibt es auch viele kleinere Extensions, die hauptsächlich dem Entwicklerkomfort dienen und z.B. eine
Redis-Verbindung anhand der Daten in der Flask-Konfiguration aufbauen oder das Verschicken von
E-Mails über SMTP mithilfe verschiedener Python-Libraries mit einer intuitiven API vereinfachen.

Flask nutzt den von Werkzeug als WSGI-Middleware bereitgestellten webbasierten Debugger. Während
dieser nicht den Funktionsumfang eines vollwertigen Debuggers bietet - insbesondere Stepping und
Breakpoints werden nicht unterstützt - so ist er bei der Entwicklung sehr hilfreich, um kleinere
Fehler schnell zu finden und mithilfe der integrierten Python-Shell Code im selben Kontext
auszuführen, in dem die Exception verursacht wurde.


\item[Dokumentation] \hfill \\
Flask, Werkzeug und Jinja2 besitzen jeweils eine sehr ausführliche Online-Dokumentation, die sich
jeweils auf die aktuellste Version bezieht. Features, die erst in einer bestimmten Version
hinzugefügt wurde, sind entsprechend markiert, sodass die Dokumentation auch bei Nutzung einer
älteren Version noch hilfreich ist. Darüber hinaus sind für jeden Versionssprung Updatehinweise
verfügbar, die explizit auf potenziell problematische Änderungen hinweisen.

Ein großer Teil der ausführlichen API-Dokumentation ist anhand der im Code verwendeten Docstrings
generiert. Dies hat den Vorteil, dass beim Lesen des Codes die relevante Dokumentation direkt
sichtbar ist. Kommentare im Code sind wie im Python-Styleguide empfohlen an allen Stellen vorhanden,
die nicht selbsterklärend sind.

\newpage
\item[Lizenz] \hfill \\
Sowohl Flask als auch Werkzeug und die übrigen zwingend notwendigen Python-Libraries stehen steht
unter der BSD-Lizenz. Für Flask-Erweiterungen wird eine ähnlich freizügige Lizenz empfohlen bzw. im
Falle von \emph{approved extensions} sogar zwingend notwendig. Die offizielle Dokumentation
empfiehlt entweder die BSD-Lizenz, die MIT-Lizenz oder die größtenteils dem amerikanischen
Public-Domain-Konzept entsprechende WTFPL\footnote{Do What the Fuck You Want to Public License}.
\end{description}

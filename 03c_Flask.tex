\section{Flask}

\emph{Flask}\footnote{\href{http://flask.pocoo.org}{http://flask.pocoo.org}} ist ein relativ
leichtgewichtiges Microframework, dessen Entwicklung 2010 begann und derzeit in Version 0.10
verfügbar ist. Neben Python 2.6 und 2.7 unterstützt die aktuellste Version des Frameworks auch
Python 3.3.

Der Begriff \enquote{Microframework} bedeutet bei Flask, dass das Framework dem Entwickler die
größtmögliche Flexibilität lässt, welche Technologien er für verschiedene Bestandteile seiner
Anwendung nutzt.

Kern von Flask ist das WSGI-Tookit \emph{Werkzeug}, welches die Lowlevel-Funktionen für das
WSGI-Interface und alle HTTP-spezifischen Utilityfunktionen bereitstellt. Auch das URL-Routingsystem
ist Teil von Werkzeug, wobei Flask die High-Level-APIs dazu bereitstellt und dadurch den in einem
Framework erwarteten Komfort bietet.


\begin{description}
\item[Modularität] \hfill \\
\todotext{Modulatität}


\item[URL-Routing] \hfill \\
\todotext{URL-Routing}


\item[Template-Engine] \hfill \\
\todotext{Template-Engine}


\item[Datenbankanbindung] \hfill \\
\todotext{Datenbankanbindung}


\item[Sessions] \hfill \\
\todotext{Sessions}


\item[Caching] \hfill \\
\todotext{Caching}


\item[Integrierbarkeit] \hfill \\
\todotext{Integrierbarkeit}


\item[Erweiterbarkeit] \hfill \\
\todotext{Erweiterbarkeit}


\item[Sonstige Features] \hfill \\
\todotext{Sonstiges}


\item[Dokumentation] \hfill \\
\todotext{Dokumentation}


\item[Lizenz] \hfill \\
Flask, Werkzeug und die übrigen zwingend notwendigen Python-Libraries stehen steht unter der
BSD-Lizenz. Für Flask-Erweiterungen ist eine ähnlich freizügige Lizenz empfohlen bzw. im Falle von
\emph{approved extensions} sogar zwingend notwendig. Die offizielle Dokumentation empfiehlt entweder
die BSD-Lizenz, die MIT-Lizenz oder die größtenteils dem amerikanischen Public-Domain-Konzept
entsprechende WTFPL\footnote{Do What the Fuck You Want to Public License}.
\end{description}

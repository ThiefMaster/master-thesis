\chapter{Python-Webframeworks}

\section{Vergleichskriterien}
Um die Frameworks miteinander vergleichen zu können, müssen einige Kriterien festgelegt werden,
anhand derer sich alle Frameworks messen lassen.

\begin{description}
\item[Modularität] \hfill \\
Gerade bei einer großen Anwendung wie Indico gibt es diverse Bereiche, die relativ unabhängig
voneinander sind. Ein Framework, das es dem Entwickler erlaubt, die Anwendung in einzelne Module
aufzuteilen, ist hilfreich, um den Code besser zu strukturieren. Sofern diese Module auch
verschiedene Namespaces bereitstellen, wird zudem das Risiko von Namenskollisionen verhindert.

\item[URL-Routing] \hfill \\
In modernen Webanwendungen erwartet man in der Regel saubere, semantische URLs. Dabei handelt es
sich um URLs, aus deren Pfad weder Rückschlüsse auf die verwendete Technologie (z.B. \emph{help.php}
oder \emph{conferenceDisplay.py}) gezeogen werden können noch ausschließlich nichtssagende Parameter
enthalten sind. Ein einfaches Beispiel für solche Parameter ist eine URL dieser Form:
\emph{/view.py?type=3\&id=12345}. Während \emph{view} zwar aussagt, dass irgendein Objekt angezeigt
wird und dieses Objekt die interne Identifikationsnummer \emph{12345} besitzt, kann man der URL
nicht ansehen, worum es sich jetzt genau handelt. Im Gegensatz dazu steht eine URL der Form
\emph{/view/event/12345-welcome}. An dieser sieht man sofort, dass es sich um ein Event handelt,
welches höchstwahrscheinlicht den Titel \enquote{welcome} hat.

Um solche URLs zu realisieren, arbeiten Frameworks normalerweise mit einer Routingtabelle, die URLs
auf Funktionen, Klassen, o.ä. mappt. Der Aufbau der URLs, insbesondere wenn dynamische Parameter
enthalten sind, variiert von Framework zu Framework. So bieten sich reguläre Ausdrücke für maximale
Flexibilität an, allerdings sind einfache Platzhalter in den meisten Fällen ausreichend und sorgen
für lesbarere URLs im Code.

\item[Templateengine] \hfill \\
Trotz der Nutzung von AJAX besteht Bedarf an dynamisch generierten HTML-Seiten. Eine Templateengine
bietet dazu eine spezielle Syntax, mit deren Hilfe sowohl HTML-Dateien als auch beliebige andere
textbasierte Dateiformate durch Variablen und einfache Kontrollstrukturen erweitert werden können.
Während es prinzipiell möglich ist, die Templateengine zu wechseln und die meisten Templates
automatisiert in die Syntax der neuen Engine zu konvertieren, hat es Vorteile, wenn das Framework
mit einer beliebigen Templateengine benutzt werden kann.

\item[Datenbankanbindung] \hfill \\
Viele Framework benötigen eine Datenbank, um frameworkspezifische Daten abzuspeichern oder um für
den möglicherweise vorhandenen Administrationsbereich eine Benutzer- und Rechteliste zu führen. Da
die in Indico genutzte ZODB, nicht sehr verbreitet ist, ist insbesondere darauf zu achten, ob zur
Nutzung des Frameworks eine separate Datenbank benötigt wird.

\item[Sessions] \hfill \\
Sowohl beim Login eines Benutzers als auch bei der Nutzung diverser Funktionen von Indico müssen
Daten zwischengespeichert werden. Dazu ist es praktisch, wenn das Framework bereits ein
Sessionsystem bietet.

\item[Caching] \hfill \\
Insbesondere bei häufig aufgerufenen Seiten und komplexen Datenbankzugriffen kann die Performance
einer Anwendung massiv gesteigert werden, indem entweder ganze Seiten oder Teile der zur Generierung
selbiger notwendigen Daten in einem Cache abgelegt werden. Ein Framework mit einer integrierten
Caching-Lösung kann dabei behilflich sein, indem es beispielsweise bestimmte Seiten automatisch
cacht und dabei Parameter und evtl. vorhandene Berechtigungen des Benutzers beachtet.

\item[Integrierbarkeit] \hfill \\
Bei Indico handelt es sich um eine komplexe Anwendung mit vielen Funktionen, sodass es nicht
praktikabel ist, wenn der gesamte Code geändert oder in großen Teilen neu geschrieben werden muss.
Daher ist bei dem neuen Framework darauf zu achten, dass es möglichst leicht in eine bestehende
Anwendung integriert werden kann.

\item[Erweiterbarkeit] \hfill \\
Unter Umständen ist es notwendig, das Framework durch zusätzliche Funktionalität zu erweitern oder
bestehende Funktionen anzupassen. Dies kann über eine Plugin-API, durch Subclassing oder aber durch
Forking des Frameworks geschehen. Letzteres bedeutet, den Code des Frameworks selbst zu verändern
und regelmäßig auf den aktuellen Stand der offiziellen Versionen zu bringen.

\item[Sonstige Features] \hfill \\
Viele Frameworks haben neben den üblichen Features zusätzliche Funktionen, die in anderen Frameworks
nicht vorhanden sind. Je nach Feature können diese für Indico nützlich sein.

\item[Performance] \hfill \\
Die unterschiedlichen Frameworks sind bei der Abarbeitung von Requests möglicherweise
unterschiedlich schnell. Dies ist darauf zurückzuführen, dass komplexere Frameworks mehr Dinge
implizit bei jedem Request tun, während dieselbe Funktionalität in einem kompakten Framework
entweder überhaupt nicht enthalten ist oder optional ist. Allerdings spielt dieser Performanceaspekt
im Rahmen dieser Arbeit nur eine untergeordnete Rolle, da Indico zwar große Datenmengen enthält und
dabei performant sein muss, jedoch der Overhead des Frameworks verglichen mit Datenbankzugriffen
minimal ist und somit für den Endbenutzer in der Regel nicht spürbar ist.

\item[Dokumentation] \hfill \\
Da mehrere Entwickler mit dem Framework arbeiten müssen und starke Fluktuation herrscht, da oftmals
Studenten für 3 bis 12 Monate an Indico arbeiten, ist eine gute Dokumentation wichtig, da es nicht
sehr produktiv ist, wenn man erst den Quellcode des Frameworks lesen und verstehen muss, um es
benutzen zu können. Insbesondere ist es hilfreich, wenn die Dokumentation Beispielcode enthält und
\emph{Best Practices} beschreibt, statt nur die APIs zu dokumentieren.

\item[Lizenz] \hfill \\
Die meisten Frameworks sind unter einer Open-Source-Lizenz verfügbar.
Da Indico unter der
GNU~GPL\footnote{\href{http://www.gnu.org/licenses/gpl-3.0.txt}{http://www.gnu.org/licenses/gpl-3.0.txt}}
steht, ist auf Kompatibilität mit dieser Lizenz zu achten. \autoref{img.floss-license-slide.png}
bietet einen kurzen Überblick über die verbreitetsten Open-Source-Lizenzen und zeigt die
Kompatibilität: \enquote{To see if software can be combined, just start at their respective
licenses, and find a common box you can reach following the arrows.} \citep{osslic} \\
Da Indico unter der GPLv3 lizenziert ist, sind fast alle Open-Source-Lizenzen außer der \emph{Affero
GPL}\footnote{GPL mit der Erweiterung, dass auch bei einem Netzwerkzugriff auf das laufende Programm
eine \enquote{Verbreitung} stattfindet und der Quellcode zugänglich gemacht werden muss.}
kompatibel. Die einzige Ausnahme wären Frameworks, die ausschließlich unter der GPLv2 lizenziert
sind. Da die GPL jedoch bei Frameworks selten genutzt wird und insbesondere die
\emph{GPLv2-only}-Option allgemein nicht sehr verbreitet ist, ist dieses Risiko eher gering.
\img{floss-license-slide.png}{400px}{Kompatibilitätsdiagramm div. Open
Source-Lizenzen \citep{osslic}}{Open-Source-Lizenz-Kompatibilität}
\end{description}


\section{Indico}

Das derzeit in Indico verwendete System kann zwar grob als Framework betrachtet werden, ist jedoch
nicht wirklich ein Framework, da sehr viele Dinge speziell auf die Anwendung zugeschnitten sind,
statt generischen bzw. wiederverwendbaren zu sein. Dies ist bereits daran zu erkennen, dass sich der
Frameworkcode nicht in einem separaten Modul oder Package befindet sondern Teil der Indico-Codebasis
ist.

Den Kern des Frameworks bildet das Package \lstinline{indico.web.wsgi}, welches das in
\autoref{wsgi-interface} beschriebene WSGI-Interface implementiert und Emulationsschicht für
\emph{mod\_python} bereitstellt. Neben dem Bereitstellen der entsprechenden APIs und Objekte ist due
wichtigste Aufgabe dieser Schicht, Requests auf die entsprechenden Python-Dateien zu mappen.
Beispielsweise führt ein Aufruf von \emph{/index.py/foo} zum Aufruf der Function \lstinline{foo()}
in der Datei \emph{index.py} im \emph{htdocs}-Ordner. Da diese Dateien jedoch - um das Verhalten von
\emph{mod\_python} beizubehalten - erst dann geladen werden, wenn ein entsprechender Request
empfangen wird, können sie nicht beim Initialisierung der Anwendung importiert und in einem Mapping
abgelegt werden. Stattdessen wird der gesamte Code der entsprechenden Datei in einen String geladen
und danach ausgeführt. \autoref{lst:mp-emulation} zeigt eine vereinfachte Version des dafür
zuständigen Codes.

\begin{lstlisting}[caption=Laden der Legacy-Python-Dateien,label=lst:mp-emulation]
def mp_legacy_publisher(req, module, handler):
    the_module = open(module).read()
    module_globals = {}
    exec(the_module, module_globals)
    return module_globals[handler](req, **req.form)
\end{lstlisting}

Bekanntermaßen ist es schlechter Stil, Code aus Strings auszuführen. In Python hat es darüberhinaus
noch den Nachteil, dass Debugger keinen Zugriff auf den Code haben, da sie nicht wissen, aus welcher
Datei der Code stammt, und Python den Quelltext selbst nicht zusammen mit dem Bytecode im Speicher
hält. In Indico ist dies allerdings kein größeres Problem, da die Dateien keine Logik enthalten
sondern nur die entsprechende Handlerfunktion aufrufen, allerdings ist es dennoch unsauber.

Neben diesen Legacy-Handlern, die jedoch für fast alle Bestandteile von Indico genutzt werden,
existiert ein sehr einfaches Routingsystem, das Anfragen anhand des ersten Pfadsegments der URL an
eine Handler-Funktion übergibt. Dies ist zwar prinzipiell besser, da so die Nutzung von
\lstinline{exec} vermieden werden könnte, allerdings wird letzendlich genau dieselbe zuvor schon
beschriebene Funktion verwendet, was leicht dazu verleitet, in der in \autoref{lst:lame-routing}
gezeigten Routingtabelle direkt auf die Handlerfunktion mit der eigentlichen Logik zu verweisen.
Dies führt dann allerdings dazu, dass das mit \lstinline{exec} verbundene Debuggingproblem plötzlich
auch relevanten Code betrifft statt nur einen einfachen Funktionsaufruf. Umgangen werden kann das
zum Glück sehr einfach, indem eine separate Python-Datei nur den Aufruf der eigentlichen
Handler-Funktion enthält und diese Datei in der Routingtabelle referenziert wird.

\begin{lstlisting}[caption=Einfaches URL-Routing,label=lst:lame-routing]
{'': ((self.htdocs_dir, 'index.py'), 'index', '', None),
 'services': ((DIR_SERVICES, 'handler.py'), 'handler', '', None),
 'export': ((DIR_MODULES, 'wsgi_handler.py'), 'handler', '', None),
 'api': ((DIR_MODULES, 'wsgi_handler.py'), 'handler', '', None)
}
\end{lstlisting}

Für die eigentliche Anwendungslogik sind die \lstinline{RH}-Klassen zuständig. Die Basisklasse
enthält dabei die Logik für die Datenbankverbindung und zum Abfangen von Fehlern, während die
Subklassen davon die eigentliche Anwendungslogik implementieren. Die Abgrenzung zwischen Framework
und Anwendung wird teilweise verwässert, da Teile der Basisklasse auf Anwendungscode zugreifen
während diverse andere nur Frameworkcode (in der Regel das \lstinline{req}-Objekt mit den
\emph{mod\_python})-Daten nutzen.

\begin{description}
\item[Modularität] \hfill \\
Indico enthält ein Modulsystem, welches sowohl für Code, der ein fester Bestandteil von Indico ist,
als auch für externen Code genutzt werden kann. In beiden Fällen bietet das Modulsystem einen
Namespace in der Datenbank, der ausschließlich vom jeweiligen Modul genutzt wird, und einen
Menüpunkt in der Administrationsoberfläche, über den das Modul aktiviert bzw. deaktiviert und
konfiguriert werden kann. Dies hat den großen Vorteil, dass die Konfiguration bei den meisten
Modulen sehr einfach ist - sowohl für die Administratoren als auch für den Entwickler.

Einige Module, beispielsweise das Interface zum CERN-Paymentsystem und zur CERN-Suchmaschine, sind
nicht Teil des Open-Source-Projekts. Daher muss es für solche Module eine Möglichkeit geben, auch
ohne direkt im Indico-Code referenziert zu werden, auf Datenstrukturen zugreifen und bei bestimmten
Ereignissen Code ausführen zu können. Dies wurde in Indico mithilfe von \emph{entry points}
realisiert. Dabei handelt es sich um ein Feature des Paketsystems von Python, über das ein Paket
Objekte für einen bestimmten \emph{entry point} zentral registrieren kann. Indico dann dann mit der
Funktion \lstinline{pkg_resources.iter_entry_points('indico.ext')} über diese Objekte iterieren und
die entsprechenden Plugins laden.

Plugins können über reguläre Ausdrücke Code für bestimmte URLs ausführen, sodass sie auch eigene
Seiten hinzufügen können; in der Regel folgt die URL-Struktur dabei den \emph{mod\_python}-URLs,
d.h. \emph{something.py} bzw. \emph{something.py/someAction}. Daneben können Plugins sowohl
Funktionen für die JSON-RPC-API als auch Endpoints für die REST-basierte Export-API registrieren.

\end{description}

\section{Django}

\emph{Django}\footnote{\href{https://www.djangoproject.com}{https://www.djangoproject.com}} ist das
wohl bekannteste und funktionsreichste Full-Stack-Webframework für Python. \emph{Full-Stack}
bedeutet dabei, dass alle für eine typische Webanwendung wichtigen Komponenten wie Datenbankzugriff,
Routing, Templates und Sessions vom Framework bereitgestellt werden.

Es ist primär auf \emph{Rapid Application Development} ausgelegt und wird daher von den Entwicklern
auch mit dem Slogan \enquote{for perfectionists with deadlines} beworben. Wie für RAD-Frameworks
typisch bietet Django über das Management-Tool \emph{django-admin.py} diverse
Scaffolding-Funktionen, d.h. die Möglichkeit, die Grundstruktur für ein neues Projekt bzw. neue
Module innerhalb eines Projekts automatisch zu generieren.

Die Entwicklung von Django begann 2003, wobei die erste Open-Source-Version 2005 veröffenlicht
wurde. Seitdem wurde das Framework konstant weiterentwickelt und ist derzeit in der Version 1.6
verfügbar, die unter anderem Kompatibilität mit Python 3 bietet.


\begin{description}
\item[Modularität] \hfill \\
Django bietet Modularität in verschiedenen Schichten. Bei der Webanwendung, die der Benutzer sieht
handelt es sich - um bei der in der Django-Dokumentation genutzten Terminologie zu bleiben - um das
\emph{Project}. Dabei handelt es sich in der Regel um ein in sich abgeschlossenes Projekt, das die
Konfiguration sowohl für Django selbst, die genutzte Datenbank und die Anwendung enthält. Ein
\emph{Project} enthält eine oder mehrere \emph{Apps}, wobei es sich bei jeder App um ein
Python-Paket handelt, das Datenbankmodelle, Viewfunktionen, Templates und Anwendungslogik enthält.
Die Anwendung auf viele kleine Apps aufzuteilen hat den Vorteil, dass in sich
abgeschlossene Funktionalität wie Tagging oder auch ein Loginsystem sauber von dem
restlichen Projekt abgegrenzt ist und damit in der Regel wiederverwendbar ist. Dies wird von Django
insofern unterstützt, dass auch problemlos Apps aus anderen Projekten eingebunden werden können und
jede App in der Regel unabhängig von ihrem \emph{Project} ist sofern die nötigen Konfigurationsdaten
vorhanden sind. Django selbst enthält diverse Apps für optionale Zusatzfunktionen wie eine
Benutzerverwaltung und Kommentarfunktion.

Unabhängig von den installierten Apps bietet Django den Middleware-Stack, über den auf einer relativ
niedrigen Ebene in die Abarbeitung von Requests eingegriffen werden kann.
\lstinline{process_request} ermöglicht es der Middleware, vor dem Routing eines Requests
einzugreifen und entweder die reguläre Verarbeitung fortsetzen oder sie mit einer HTTP-Antwort
vorzeitig beenden. Dies bietet sich z.B. an, um Seiten anhand der aufgerufenen URL zu cachen.
\lstinline{process_view} verhält sich ähnlich, allerdings ist bei der Ausführung dieses
Middleware-Hooks bereits bekannt, welche Viewfunktion den Request abarbeiten wird.

Middleware hat außerdem die Möglichkeit, nach der Verarbeitung eines Requests einzugreifen. Dazu
stehen die Methoden \lstinline{process_response} und \lstinline{process_exception} zur Verfügung.
Diese könnten z.B. einen Cache aktualisieren oder einen webbasierten Debugger starten.


\item[URL-Routing] \hfill \\
Django legt großen Wert auf saubere URLs, die weder Dateiendungen noch kryptische Elemente wie
\emph{/0,2097,1-1-1928,00} enthalten. Aus diesem Grund bietet Django ein flexibles
und mächtiges Routingsystem. Da der Kern des Routingsystems der Vergleich von definierten Mustern
mit der aufgerufenen URL ist, bieten sich reguläre Ausdrücke perfekt an und werden dementsprechend
auch verwendet.

Der Entwickler definiert in seiner App dazu eine Liste mit \emph{Patterns}, wobei jedes Element aus
einem regulären Ausdruck und der Viewfunktion, wobei es sich bei letzterem entweder direkt um die
Funktion handeln kann oder aber um einen String der Form \lstinline{'myapp.views.somefunc'}. Für
dynamische Elemente in der URL wird das Gruppierungs-Feature der Regex-Engine verwendet. Dies
bedeutet, dass der Ausdruck \lstinline{r'news/(\d+)'} eine URL nach dem Schema \emph{news/123}
matchen würde und die Zahl als positionalen Parameter an die Viewfunktion übergibt. Da dies gerade
in komplexeren URLs mit mehreren Parametern schnell unübersichtlich würde, können auch Namen für die
Parameter in der Form \lstinline{r'news/(?P<id>\d+)'} vergeben werden. Diese werden dann als Keyword
Arguments an die Viewfunktion übergeben.

Es ist zu erwähnen, dass das Routing ausschließlich die regulären Ausdrücke benutzt und somit nur
den Pfad in der URL berücksichtigt. Das HTTP-Verb, also meist \emph{GET} oder \emph{POST} und der
Domainname sind also nicht Teil des Routings.

Insbesondere mit Apps, die zusätzliche Funktionalität wie eine Kommentarfunktion
bereitstellen, würde es dem Prinzip der sauberen URLs widersprechen, alle URLs auf der Rootebene zu
registrieren. Dazu bietet Django die \lstinline{include()}-Funktion, mit der anstelle auf eine
Viewfunktion auch auf eine andere URL-Routingtabelle verwiesen werden kann.

Um anhand der Viewfunktion oder eines optionalen Identifiers eine URL zu generieren, bietet Django
sowohl die Python-Funktion \lstinline{reverse()} als auch eine spezielle Syntax für Templates. Ganz
im Sinne von DRY führt dies dazu, dass URLs niemals manuell erzeugt werden müssen und Änderungen
keine toten internen Links zur Folge haben, da Dispatching und URL-Generierung beide auf dieselben
Routingtabellen zugreifen.


\item[Templateengine] \hfill \\
Django verwendet eine eigene Templateengine, die speziell für das Framework entwickelt wurde und
dementsprechend gut integriert ist. Sie unterstützt alle von einer modernen Templateengine
erwarteten Features und ermöglicht direkten Zugriff auf das Routingsystem bzw. die dazu gehörende
URL-Generierung von Django. Aus Sicherheitsgründen werden alle dynamischen Daten in HTML-Templates
escaped, sofern es nicht explizit deaktiviert wird.

Da es sich bei Templates offensichtlich um die Präsentationsschicht handelt und dort höchstens
ausgabespezifische Logik vorhanden sein soll, ist es in Django-Templates nicht möglich, Python-Code
direkt einzubinden. Allerdings können Python-Funktionen in Templates zugänglich gemacht werden und
beliebige Methoden von übergebenen Python-Objekten ausgeführt werden.

\lstinputlisting[caption=Django-Template]{code/django.tpl}


\item[Datenbankanbindung] \hfill \\
Django enthält ein ORM-System, welches die meisten verbreiteten relationalen Datenbanksysteme wie
PostgreSQL, MySQL, Oracle und SQLite offiziell unterstützt und über inoffizielle Zusatzmodule auch
weitere Datenbanken wie den Microsoft SQL Server nutzen kann.

Während es bei einer neuen Anwendung üblich ist, die Model-Klassen für die Datenbanktabellen manuell
zu erstellen und im Anschluss auf diesen basierend die Tabellen zu erstellen, bietet Django auch die
Möglichkeit, eine existierende Datenbank zu analysieren und die dazu passenden Klassen zu
generieren.

Teile von Django selbst benötigen eine mit dem Django-ORM kompatible SQL-Datenbank. Allerdings gibt
es die Django-Erweiterung \emph{Django-ZODB} um die ZODB für Anwendungsdaten nutzen zu können.


\item[Sessions] \hfill \\
Das Session-System von Django ist zwar Teil des \lstinline{django}-Pakets, allerdings ist es nicht
Teil des Django-Kerns sondern als App und Middleware realisiert, was ein gutes Beispiel für die
Modularität des Frameworks ist. Wenn es aktiviert ist stellst es in \lstinline{request.session} ein
\lstinline{dict}-artiges Objekt bereit, über das auf die Daten der Session zugegriffen werden kann.
Bei \lstinline{request} handelt es sich um den an jede Viewfunktion übergebenen Parameter, über den
auf alle zum aktuellen Request gehörenden Daten zugegriffen werden kann.

Die meisten Eigenschaften der Session-App können konfiguriert werden was sie sehr flexibel macht.
Insbesondere können verschiedene Storage-Backends verwendet werden - u.a. einen Cache wie
\emph{memcached}, signierte Cookies oder eine Tabelle in der Datenbank - und man hat die Wahl
zwischen \emph{Pickle}, \emph{JSON} oder einer eigenen Implementierung bei der Serialisierung der
Sessiondaten. Letzteres ist insbesondere bei clientseitigen Cookie-Sessions wichtig, da es sich bei
\emph{Pickle} zwar um ein sehr mächtiges Format handelt, das fast jeden Python-Datentyp
serialisieren kann, diese Flexibilität jedoch zur Folge hat, dass beim Deserialisieren auch
beliebiger Code ausgeführt werden kann. Durch die kryptografische Signatur kann ein Benutzer zwar
prinzipiell keine bösartigen Daten als gültiges Sessioncookie übermitteln, allerdings führt ein Leak
des geheimen Schlüssels sofort auch zu einer \emph{Remote Code Execution}-Lücke, da mit diesem
Schlüssel gültige Cookiesignaturen erstellt werden können.

Seitens der Entwickler wird das Cache-Backend empfohlen; in diesem Fall bietet sich der
Pickle-Serializer an, da bei dieser Kombination sowohl die hohe Performance des Caches als auch die
Flexibilität von Pickle zur Verfügung stehen und der Client niemals mit den Sessiondaten in
Berührung kommt und somit keine Möglichkeit hat, sie zu manipulieren.

Das Sessionmodul nutzt unabhängig vom Backend ein Cookie; entweder zum Speichern des einzigartigen
Session-Identifiers oder für die Daten selbst. Dies hat zur Folge, dass das Session-System bei
deaktivierten Cookies nicht funktioniert. Dies ist heutzutage allerdings kein relevantes Problem
mehr, da fast jede Website spätestens beim Login Cookies voraussetzt.


\item[Caching] \hfill \\
Wie zuvor schon erwähnt, eignet sich Djangos Middleware-System unter anderem ideal dafür, ganze
Seiten zu cachen. Dementsprechend ist es nur logisch, dass Django eine entsprechende Middleware
mitliefert.  Diese realisiert Caching wie zu erwarten auf Seitenebene, d.h. jede via GET aufgerufene
Seite wird gecacht, sofern es nicht durch entsprechende Header unterbunden wird. Insbesondere auf
Seiten, deren Content nicht extrem oft aktualisiert wird, bietet sich diese Cachemethode an, da sie
mit minimalem Aufwand realisierbar ist.

Eine flexiblere Cachemethode ist der Decorator-basierte View-Cache. Dieser ermöglicht es, einzelne
Views zu cachen, sodass z.B. die Startseite gecacht werden kann aber andere, häufiger
aktualisierte Seiten wie ein Gästebuch, nicht gecacht werden. Sofern die Granularität
immernoch nicht ausreicht, erlaubt Django es auch, einzelne Templatefragmente zu cachen. Dies bietet
sich an, wenn Teile eines Templates zu dynamisch zum cachen sind, andere jedoch nur relativ selten
aktualisiert werden.

Neben diesen Methoden bietet Django auch eine Lowlevel-API, die direkten Cachezugriff ermöglicht.
Diese unterstützt mehrere Caches, die auch verschiedene Backends nutzen können, und bietet neben den
üblichen Methoden \lstinline{get}, \lstinline{set}, \lstinline{delete} und den Varianten für mehrere
Keys auch \lstinline{incr} und \lstinline{decr} zum Inkrementieren bzw. Dekrementieren von
numerischen Werten. Dabei handelt es sich, sofern es vom Backend unterstützt wird, um atomare
Operationen.

Standardmäßig werden Memcached, Datenbanktabellen, Dateien und ein insbesondere während der
Entwicklung hilfreicher prozess-lokaler In-Memory-Cache unterstützt. Es ist auch möglich, ein
benutzerdefiniertes Cache-Backend zu verwenden.
\newpage

\item[Integrierbarkeit] \hfill \\
Da es sich bei Django um ein Full-Stack-Framework handelt ist es darauf ausgelegt, von Anfang an
verwendet zu werden. Es ist zwar prinzipiell möglich, Teile von Django, wie das Routingsystem, zu
nutzen, ohne die fortgeschritteneren Features, wie das ORM, zu verwenden, allerdings muss man dabei
in der Regel auf viele Vorteile des Frameworks verzichten. So müssten z.B. alle Templates der
bestehenden Anwendung angepasst werden, um die URL-Generierung des Routingsystems zu verwenden.

Explizit unterstützt wird jedoch die Nutzung einer bereits existierenden Datenbank: Wie bereits
erwähnt, kann Django die Model-Klassen anhand des Schemas der Datenbank generieren. Somit spart man
sich den Aufwand, die Model-Klassen zu erstellen, und kann sich darauf konzentrieren, die
Anwendungslogik neu zu implementieren.


\item[Erweiterbarkeit] \hfill \\
Der einfachste Weg, Django zu erweitern, ist durch Middleware. Sofern es nicht ausreicht,
vor und nach der Abarbeitung eines Requests Code auszuführen, bietet Django mit \emph{Signals} ein
Callback-System, welches insbesondere eine Reaktion auf verschiedene Datenbankereignisse ermöglicht.

Um das Verhalten des Django-Kerns selbst zu verändern, ist es jedoch in der Regel notwendig, den
Frameworkcode zu modifizieren. Dies bedeutet, dass entweder ein Fork notwendig ist oder aber dass
die Änderungen allgemein genug sind, um eine Chance zu haben, von den Django-Entwicklern in die
offizielle Django-Version übernommen zu werden.


\item[Sonstige Features] \hfill \\
RAD-typisch kann Django für einfache CRUD-Aufgaben in der Administration einer Website sowohl die
Logik als auch das Benutzerinterface dynamisch generieren. Dabei handelt es sich nicht um das
in vielen Frameworks übliche Scaffolding, bei dem der entsprechende Code einmalig generiert und
danach modifiziert wird. Stattdessen werden die Formulare anhand der Datenbankmodelle
dynamisch generiert, wobei dieser Vorgang vom Entwickler beeinflusst werden kann, um auch komplexere
Elemente wie One-To-Many-Beziehungen in der gewünschten Art und Weise in einem Formular
repräsentieren zu können.

Auch das wohl wichtigste Feature in den meisten Webapplikationen, die Authentifikation von
Benutzern, wird von Django standardmäßig unterstützt, sofern man die entsprechende App aktiviert.
Neben der Benutzerverwaltung selbst und der Authentifikation mittels eines Passworts oder über einen
Drittanbieter enthält das Benutzersystem ein Gruppen- und Rechtesystem und diverse Decorators um den
Zugriff auf einzelne Views zu beschränken.


\item[Dokumentation] \hfill \\
Django besitzt eine sehr ausführliche Online-Dokumentation sowohl für die aktuelle als auch für
ältere Versionen, die jeweils auch in verschiedenen Formaten heruntergeladen werden kann. Neben
einer allgemeinen Beschreibung der verschiedenen Bestandteile des Frameworks enthält sie auch ein
Step-By-Step-Tutorial und eine Beschreibung aller in Django genutzten Klassen und Funktionen.

Der Code von Django ist größtenteils sauber dokumentiert bzw. kommentiert, wobei leider im
Django-Kern einige interne Funktionen für einen Außenstehenden, der nicht mit dem Code vertraut ist,
nicht selbsterklärend sind. Allerdings ist es in der Regel nicht notwendig oder angebracht,
Frameworkcode zu verändern und dank der guten Dokumentation der APIs muss man auch nur selten etwas
direkt im Code nachschauen.


\item[Lizenz] \hfill \\
Django steht unter der BSD-Lizenz. Dabei handelt es sich um eine \emph{permissive}
Open-Source-Lizenz, die nur verlangt, dass der Copyrighthinweise im Code erhalten bleibt und der
Name des ursprünglichen Entwicklers nicht missbräuchlich verwendet wird. Damit ist die Lizenz ideal
für ein Framework geeignet, da sie der eigentlichen Anwendung keine bestimmte Lizenz aufzwingt und
auch die kommerzielle Nutzung ohne Einschränkungen erlaubt.


\end{description}

\section{Flask}

\emph{Flask}\footnote{\href{http://flask.pocoo.org}{http://flask.pocoo.org}} ist ein relativ
leichtgewichtiges Microframework, dessen Entwicklung 2010 begann und derzeit in Version 0.10
verfügbar ist. Neben Python 2.6 und 2.7 unterstützt die aktuellste Version des Frameworks auch
Python 3.3.

Der Begriff \enquote{Microframework} bedeutet bei Flask, dass das Framework dem Entwickler die
größtmögliche Flexibilität lässt, welche Technologien er für die einzelnen Bestandteile seiner
Anwendung nutzt.

Kern von Flask ist das WSGI-Toolkit \emph{Werkzeug}, das die Lowlevel-Funktionen für das
WSGI-Interface und alle HTTP-spezifischen Utilityfunktionen bereitstellt. Auch das URL-Routingsystem
ist Teil von Werkzeug, wobei Flask die High-Level-APIs dazu bereitstellt und dadurch den in einem
Framework erwarteten Komfort bietet.


\begin{description}
\item[Modularität] \hfill \\
Flask ermöglicht modulare Anwendungen mithilfe von \emph{Blueprints}. Diese verhalten sich ähnlich
wie eine vollwertige Flask-Anwendung, allerdings fehlt die gesamte für eine Flask-Anwendung
notwendige Logik. Stattdessen fügen Blueprints ihre Bestandteile der Anwendung hinzu, sobald sie bei
der Anwendung registriert werden. Im Gegensatz zu mehreren kleineren Anwendungen haben
Blueprints den Vorteil, dass sie weder für die anwendungsweite Fehlerbehandlung noch für
Funktionalität wie die Datenbankverbindung zuständig sind, sondern all diese Dinge von der Anwendung
selbst übernehmen.
Ein Blueprint kann Templates, statische Daten, URL-Routingdaten und Viewfunktionen enthalten.
Darüber hinaus ist es möglich, Exceptions auf Blueprintebene abzufangen statt sie grundsätzlich an
das Errorhandling der Anwendung selbst weiterzureichen.

Allerdings hat das Blueprint-System auch einen Nachteil: Dadurch, dass es sich gerade nicht um eine
vollwertige Anwendung handelt, eignen sie sich nur bedingt dazu, komplett wiederverwendbare Module
zu entwickeln, die auch von Dritten \emph{as-is} verwendet werden können. Diesem Problem kann jedoch
bei Bedarf entgegengewirkt werden, indem man entsprechende Voraussetzungen an die Anwendung stellt,
die den Blueprint nutzen soll, oder problematische Teile entsprechend abstrahiert. Eine Klasse, die
auf Benutzerdaten zugreift, könnte z.B. abstrakte Methoden enthalten, die in der jeweiligen
Anwendung dann überschrieben werden müssen.


\item[URL-Routing] \hfill \\
Wie jedes moderne Framework setzt auch Flask auf saubere URLs. Dazu nutzt es das von Werkzeug
bereitgestellte URL-Routing-System und erweitert es um eine einfachere API, die insbesondere bei der
Registrierung von Routingregeln mittels Decorators und beim Generieren von URLs ihre Vorteile zeigt.

Die einfachste Möglichkeit, eine Routingregel hinzuzufügen, ist mit dem Decorator
\lstinline{app.route}. Dieser akzeptiert als Parameter die Routingregel und optional die erlaubten
HTTP-Methoden. Die damit dekorierte Funktion wird daraufhin automatisch mit der Regel verknüpft und
der Funktionsname als Endpoint benutzt. Natürlich sind in einer größeren Anwendung nur wenige URLs
komplett statisch, sondern enthält auch diverse Variablen. Da es sich bei diesen in den meisten
Fällen um einfache Strings oder Zahlen handelt, werden diese durch einfache Platzhalter in der Form
\lstinline{'<type:name>'} bzw. bei Strings \lstinline{'<name>'} angegeben statt über oftmals eher
kryptische reguläre Ausdrücke. Standardmäßig unterstützt Flask die Typen \lstinline{string},
\lstinline{int}, \lstinline{float} und \lstinline{path}, wobei es möglich ist, eigene Typen zu
definieren indem man eine Subklasse von \lstinline{BaseConverter} definiert und darin die Methoden
\lstinline{to_url} und \lstinline{to_python} implementiert. Intern nutzt das Routingsystem reguläre
Ausdrücke mit denen man allerdings nur in Kontakt kommt, wenn man einen eigenen Typkonverter
definiert. Die Übergabe der Parameter an die Viewfunktion erfolgt über Keyword Arguments. Auch
optionale Parameter sind möglich; in diesem Fall muss die Funktion mindestens eine Routingregel
besitzen, die den entsprechenden Parameter nicht enthält.

Sofern Decorators keine Option sind - z.B. weil der Code über viele Dateien verstreut ist
und man die Routingregeln an einem zentralen Ort haben will - bietet Flask die Methode
\lstinline{app.add_url_rule}. Diese entspricht prinzipiell dem Decorator, wobei die Viewfunktion
ebenfalls als Parameter übergeben wird.

Neben dem Pfad und dem HTTP-Verb kann das Routingsystem auch die aktuelle Subdomain berücksichtigen,
sowohl als festen Wert als auch als dynamische Variable. Für komplexere Fälle kann auch direkt auf
die \lstinline{Map} des Werkzeug-Routingsystems zugegriffen werden statt die Flask-API zu nutzen.
Dies ist jedoch gerade dank Blueprints nur in den wenigstens Fällen notwendig.

Um über die Routingtabelle eine URL zu generieren, benötigt man den Namen des Endpoints der
entsprechenden Routingregel. Übergibt man diesen an die Funktion \lstinline{url_for}, kann diese
eine URL daraus generieren. Standardmäßig ist diese relativ zur aktuellen Domain, also in der Form
\emph{/foo/bar}, allerdings akzeptiert die Funktion neben den Keyword Arguments für die Variablen
der Routingregel auch die speziellen Argumente \lstinline{_external} um eine vollständige URL zu
generieren, \lstinline{_schema} um das Protokoll in einer solchen absoluten URL festzulegen und
\lstinline{_anchor} um das URL-Fragment (\emph{\#something}) anzugeben. Sofern mehrere Regeln für
denselben Endpoint existieren, verwendet \lstinline{url_for} automatisch die passendste Regel.

Blueprints besitzen dieselben Methoden; der einzige relevante Unterschied ist, dass ein Blueprint
einen Präfix besitzen kann, der jeder Routingregel des Blueprints hinzugefügt wird. Bei der Nutzung
von \lstinline{url_for()} wird ein Blueprint in der Form \lstinline{'blueprint.endpoint'} angegeben,
wobei auch relative Verweise der Form \lstinline{'.endpoint'} möglich sind; in diesem Fall wird der
zum Zeitpunkt des Aufrufs aktive Blueprint benutzt.


\item[Templateengine] \hfill \\
Flask nutzt standardmäßig die \emph{Jinja2}-Templateengine, die auch unabhängig von Flask verfügbar
ist. Die Templatesyntax ist größtenteils mit der von Django identisch und auch die Features sind
sehr ähnlich. Flask stellt diverse Funktionen - insbesondere \lstinline{url_for()} zum Generieren
von URLs - und u.a. die Request- und Sessionobjekte in allen Templates zur Verfügung und escaped
dynamische Daten in HTML-Templates automatisch, sofern es nicht explizit deaktiviert wird.

Um Jinja2 anzupassen - sei es mit benutzerdefinierten Templatefiltern oder zusätzlichen globalen
Funktionen - stellt Flask im Applikationsobjekt entsprechende Funktionen zur Verfügung, die jeweils
auch als Decorator verwendet werden können.

Jinja2 steht in einer Flask-Anwendung immer zur Verfügung; es handelt sich dabei um eine feste
Abhängigkeit des Frameworks, die auch nicht deaktiviert werden kann. Der Sinn dahinter ist, dass
Flask-Erweiterungen die Engine immer nutzen können und nicht Templates für verschiedene Engines
mitliefern müssen.

Da der Zugriff auf Templates in Flask jedoch über die im \lstinline{flask}-Package definierte
Funktion \lstinline{render_template()} geschieht, die jeweils explizit aufgerufen werden muss, steht
es jedem Entwickler frei, in seiner Anwendung eine andere Templateengine zu nutzen. Allerdings muss
er in diesem Fall selbst darauf achten, Dinge wie Autoescaping zu konfigurieren und die
Flask-Objekte falls benötigt in Templates verfügbar zu machen. Für die \emph{Mako}-Templateengine
gibt es jedoch bereits eine Flask-Extension, die Mako sauber in Flask integriert. Insbesondere wird
exakt derselbe Templatekontext verwendet, der auch an Jinja2 übergeben wird. Dies hat den Vorteil,
dass der entsprechende Decorator in Flask weiterhin benutzt werden kann. Ebenfalls via Extension
unterstützt wird die XML-basierte Templateengine \emph{Genshi}.

\newpage
\item[Datenbankanbindung] \hfill \\
Flask selbst nutzt keine Datenbank und enthält, Microframework-typisch, keinen Datenbankcode. Es
exisiert mit \emph{Flask-SQLAlchemy} jedoch eine offizielle Flask-Erweiterung, die das
SQLAlchemy-ORM-System in Flask integriert und dabei denselben Komfort bietet, der auch bei allen
anderen Features von Flask üblich ist. Neben den ORM-Features kann alternativ auch nur die
Datenbankabstraktionsschicht von SQLAlchemy genutzt werden.

Mit \emph{Flask-MongoKit}, \emph{Flask-PyMongo} und \emph{Flask-MongoAlchemy} existieren auch
verschiedene Erweiterungen, um die NoSQL-Datenbank \emph{MongoDB} in Flask zu integrieren. Der Grund
für die drei verschiedenen Extensions ist, dass je nach Anwendung ein höherer oder niedrigerer
Abstraktionsgrad bei der Datenbank-API gewünscht ist.

Auch für die Objektdatenbank \emph{ZODB} existiert eine Flask-Erweiterung, die genau wie
\emph{Flask-SQLAlchemy} den \emph{approved extension}-Status hat und somit den Design Guidelines von
Flask folgt, entsprechend lizenziert ist und eine entsprechend hochwertige Dokumentation besitzt.


\item[Sessions] \hfill \\
Flask stellt über das \lstinline{session}-Objekt ein spezielles \lstinline{dict} bereit, welches
sich für den Entwickler wie ein normales \lstinline{dict} verhält, jedoch Veränderungen automatisch
registriert. Daher kann es in der Regel ohne Weiteres modifiziert werden und die Daten werden
automatisch in der Session abgespeichert. Neben den Standardmethoden enthält das Objekt diverse
Eigenschaften, um den Status der Session auszulesen bzw. zu verändern. \lstinline{new} und
\lstinline{modified} geben an, ob die Session noch nie gespeichert wurde bzw. ob sie seit dem
letzten Speichern verändert wurde. Die \lstinline{permanent}-Eigenschaft setzt die Ablaufzeit des
Session-Cookies; standardmäßig ist es nur bis zum Schließen des Browsers gültig.

Da Flask weder eine Datenbank noch einen Cache voraussetzt, werden Sessions standardmäßig
clientseitig als signiertes Cookie gespeichert. Dies ist eine sehr einfache aber oftmals
ausreichende Lösung, sofern man weder geheime noch größere Daten abspeichern will, da Cookies
oftmals auf 4096 Bytes beschränkt sind und die Signatur zwar vor Veränderungen schützt, nicht jedoch
vor Auslesen.

Aus Sicherheitsgründen werden Sessiondaten standardmäßig als \emph{JSON} serialisiert statt das
mächtigere \emph{Pickle} zu nutzen. Während die Signatur die Sessiondaten schützt, würde die Nutzung
von Pickle im Falle eines Leaks des geheimen Schlüssels nicht nur die Manipulation der Sessiondaten
ermöglichen sondern auch das Ausführen beliebigen Codes.

Sofern der Funktionsumfang von cookiebasierten Sessions nicht ausreichen sollte, kann die
Eigenschaft \lstinline{session_interface} des Anwendungsobjekts auf eine benutzerdefinierte Klasse
verweisen, die die Sessiondaten z.B. in Redis oder einer SQL-Datenbank abspeichert. Das Interface
ist dabei sehr einfach gehalten; eine Redis-basierende Implementierung ist mit weniger als 70 Zeilen
Code möglich.

Standardmäßig enthält Flask jedoch nur das Cookie-Interface, allerdings hat der Maintainer von Flask
mit dem \lstinline{RedisSessionInterface} ein entsprechendes Interface für serverseitige Sessions
als Erweiterung veröffentlicht.


\item[Caching] \hfill \\
Flask selbst enthält keinen Cache, allerdings bietet Werkzeug eine Lowlevel-Abstraktion
verschiedener Cache-Backends, und stellt dabei die Methoden \lstinline{get}, \lstinline{set},
\lstinline{delete} sowieso entsprechende \lstinline{_many}-Varianten davon zur Verfügung. Dieses
Cache-Interface ist oftmals ausreichend, allerdings bietet die Erweiterung \emph{Flask-Cache} auch
eine über die zentrale Flask-Konfiguration konfigurierbare High-Level-API dafür, die neben den
Lowlevel-Funktionen Decorators zum Cachen ganzer Views und ein Jinja2-Plugin zum Cachen einzelner
Templatefragmente.

Neben diesen relativ üblichen Cache-APIs bietet \emph{Flask-Cache} einen
\lstinline{memoize}-Decorator. Dieser kann zum Cachen beliebiger Funktionen genutzt werden; er nutzt
den Namen der Funktion und die übergebenen Parameter als Cache-Key und führt die eigentliche
Funktion nur aus, wenn das Ergebnis noch nicht im Cache vorhanden ist.


\item[Integrierbarkeit] \hfill \\
Flask benötigt weder eine bestimmte Datenbank noch setzt es eine bestimmte Verzeichnisstruktur in
der Anwendung voraus. Während URL-Routingregeln meist über Decorators definiert werden, so ist die
Nicht-Decorator-API genauso mächtig und ähnlich komfortabel. Daher ist es sehr einfach, Flask in
eine existierende Anwendung zu integrieren.

Selbstverständlich kann eine größere Anwendung, in der Flask nachträglich integriert wurde, nicht
alle Vorteile des Frameworks ausnutzen ohne dass der gesamte Code modifiziert werden muss.
Nichtsdestotrotz ist Flask sehr flexibel und so ist es kein Problem, bestehende Views funktionsfähig
in Flask einzubinden und neuen Code im \enquote{Flask-Stil} zu schreiben.


\item[Erweiterbarkeit] \hfill \\
Flask bietet mit \emph{Signals} eine API, um Callbacks zu registrieren, die zu verschiedenen
Zeitpunkten ausgeführt werden. Diese sind unter anderem das Rendern eines Templates und vor bzw.
nach einem Request. Sofern dies nicht ausreicht, kann die \lstinline{Flask}-Klasse subclassed
werden. Dies bietet die Möglichkeit, bestehende Funktionalität zu erweitern bzw. zu ersetzen und
z.B. die \lstinline{Request}-Klasse durch eine eigene Subklasse davon zu ersetzen.

Flask-Extensions nutzen in der Regel ausschließlich die Callbacks, sodass sie einfach zu einer
Anwendung hinzugefügt werden können und unabhängig von anderen Extensions sind. Um eine Extension zu
benutzen, sind zwei Methoden verbreitet. Für einfache Extensions, die keine eigene für den
Entwickler relevante Klasse besitzen, wird empfohlen, eine Funktion \lstinline{init_app(app)} zu
definieren. Diese akzeptiert als ersten Parameter die Flask-Anwendung und registriert relevante
Callbacks und führt sonstigen Initialisierungscode aus. Komplexere Extensions enthalten in der Regel
eine Klasse, über die vom Entwickler auf die Extension zugegriffen werden kann. Dabei wird das
Applikationsobjekt über einen optionalen Parameter an den Konstruktor übergeben; sofern er nicht
genutzt wird, sollte die Klasse die zuvor erwähnte Methode \lstinline{init_app(app)} besitzen. Der
Zweck davon ist, die Nutzung von \emph{application factories} zu ermöglichen, d.h. Funktionen, die
die Applikation erst erstellen und damit möglicherweise nie global verfügbar machen. In diesem Fall
ist es möglicherweise notwendig, die Instanz der Extensionklasse zu erstellen, obwohl noch keine
Flask-Instanz existiert.

Sofern keine dieser Optionen mächtig genug sind, besteht auch die Möglichkeit, eine der
Werkzeug-Klassen durch eine eigene Subklasse zu ersetzen um beispielsweise das URL-Routing-System
direkt verändern zu können. Wie bei jedem Framework besteht darüber hinaus die Möglichkeit, das
Framework zu forken und daraufhin den Frameworkcode selbst zu verändern.


\item[Sonstige Features] \hfill \\
Flask nutzt Proxy-Objekte um \lstinline{request}, \lstinline{g} und \lstinline{session} global
verfügbar zu machen und trotzdem jeweils - auch bei mehreren Threads - auf das richtige Objekt zu
verweisen. Dadurch müssen diese Objekte niemals als Funktionsparameter übergeben werden, sondern
können einfach aus dem \lstinline{flask}-Package importiert werden.

Während \lstinline{request} und \lstinline{session} ziemlich selbsterklärend sind, handelt es sich
bei \lstinline{g} um ein von Flask selbst nicht genutztes Objekt, welches allerdings sowohl
Anwendungscode als auch Flask-Erweiterungen eine Möglichkeit bietet, requestspezifische Daten
abzulegen. Beispielsweise ist der aktuell eingeloggte User meist in \lstinline{g.user} gespeichert
und oftmals nutzen Datenbank-Extensions das Objekt, um die gerade aufgebaute Datenbankverbindung
beim nächsten Zugriff nicht erneut aufbauen zu müssen.

Für viele Standardfeatures existieren bereits fertige Erweiterungen, sodass man diese nicht selbst
implementieren muss. Neben komplexen Erweiterungen wie einem Django-ähnlichen dynamischen
CRUD-Administrationsbereich, ORM-Systemen und einer kompletten Benutzerverwaltung samt Rechtesystem
gibt es auch viele kleinere Extensions, die hauptsächlich dem Entwicklerkomfort dienen und z.B. eine
Redis-Verbindung anhand der Daten in der Flask-Konfiguration aufbauen oder das Verschicken von
E-Mails über SMTP mithilfe verschiedener Python-Libraries mit einer intuitiven API vereinfachen.

Flask nutzt den von Werkzeug als WSGI-Middleware bereitgestellten webbasierten Debugger. Während
dieser nicht den Funktionsumfang eines vollwertigen Debuggers bietet - insbesondere Stepping und
Breakpoints werden nicht unterstützt - so ist er bei der Entwicklung sehr hilfreich, um kleinere
Fehler schnell zu finden und mithilfe der integrierten Python-Shell Code im selben Kontext
auszuführen, in dem die Exception verursacht wurde.


\item[Dokumentation] \hfill \\
Flask, Werkzeug und Jinja2 besitzen jeweils eine sehr ausführliche Online-Dokumentation, die sich
jeweils auf die aktuellste Version bezieht. Features, die erst in einer bestimmten Version
hinzugefügt wurde, sind entsprechend markiert, sodass die Dokumentation auch bei Nutzung einer
älteren Version noch hilfreich ist. Darüber hinaus sind für jeden Versionssprung Updatehinweise
verfügbar, die explizit auf potenziell problematische Änderungen hinweisen.

Ein großer Teil der ausführlichen API-Dokumentation ist anhand der im Code verwendeten Docstrings
generiert. Dies hat den Vorteil, dass beim Lesen des Codes die relevante Dokumentation direkt
sichtbar ist. Kommentare im Code sind wie im Python-Styleguide empfohlen an allen Stellen vorhanden,
die nicht selbsterklärend sind.

\newpage
\item[Lizenz] \hfill \\
Sowohl Flask als auch Werkzeug und die übrigen zwingend notwendigen Python-Libraries stehen steht
unter der BSD-Lizenz. Für Flask-Erweiterungen wird eine ähnlich freizügige Lizenz empfohlen bzw. im
Falle von \emph{approved extensions} sogar zwingend notwendig. Die offizielle Dokumentation
empfiehlt entweder die BSD-Lizenz, die MIT-Lizenz oder die größtenteils dem amerikanischen
Public-Domain-Konzept entsprechende WTFPL\footnote{Do What the Fuck You Want to Public License}.
\end{description}


\chapter{Python-Webframeworks}

\section{Vergleichskriterien}
Um die Frameworks miteinander vergleichen zu können, müssen einige Kriterien festgelegt werden,
anhand derer sich alle Frameworks messen lassen.

\begin{description}
\item[Modularität] \hfill \\
Gerade bei einer großen Anwendung wie Indico gibt es diverse Bereiche, die relativ unabhängig
voneinander sind. Ein Framework, das es dem Entwickler erlaubt, die Anwendung in einzelne Module
aufzuteilen, ist hilfreich, um den Code besser zu strukturieren. Sofern diese Module auch
verschiedene Namespaces bereitstellen, wird zudem das Risiko von Namenskollisionen verhindert.

\item[URL-Routing] \hfill \\
In modernen Webanwendungen erwartet man in der Regel saubere, semantische URLs. Dabei handelt es
sich um URLs, aus deren Pfad weder Rückschlüsse auf die verwendete Technologie (z.B. \emph{help.php}
oder \emph{conferenceDisplay.py}) gezeogen werden können noch ausschließlich nichtssagende Parameter
enthalten sind. Ein einfaches Beispiel für solche Parameter ist eine URL dieser Form:
\emph{/view.py?type=3\&id=12345}. Während \emph{view} zwar aussagt, dass irgendein Objekt angezeigt
wird und dieses Objekt die interne Identifikationsnummer \emph{12345} besitzt, kann man der URL
nicht ansehen, worum es sich jetzt genau handelt. Im Gegensatz dazu steht eine URL der Form
\emph{/view/event/12345-welcome}. An dieser sieht man sofort, dass es sich um ein Event handelt,
welches höchstwahrscheinlicht den Titel \enquote{welcome} hat.

Um solche URLs zu realisieren, arbeiten Frameworks normalerweise mit einer Routingtabelle, die URLs
auf Funktionen, Klassen, o.ä. mappt. Der Aufbau der URLs, insbesondere wenn dynamische Parameter
enthalten sind, variiert von Framework zu Framework. So bieten sich reguläre Ausdrücke für maximale
Flexibilität an, allerdings sind einfache Platzhalter in den meisten Fällen ausreichend und sorgen
für lesbarere URLs im Code.

\item[Template-Engine] \hfill \\
Trotz der Nutzung von AJAX besteht Bedarf an dynamisch generierten HTML-Seiten. Eine Templateengine
bietet dazu eine spezielle Syntax, mit deren Hilfe sowohl HTML-Dateien als auch beliebige andere
textbasierte Dateiformate durch Variablen und einfache Kontrollstrukturen erweitert werden können.
Während es prinzipiell möglich ist, die Templateengine zu wechseln und die meisten Templates
automatisiert in die Syntax der neuen Engine zu konvertieren, hat es Vorteile, wenn das Framework
mit einer beliebigen Templateengine benutzt werden kann.

\item[Datenbankanbindung] \hfill \\
Viele Framework benötigen eine Datenbank, um frameworkspezifische Daten abzuspeichern oder um für
den möglicherweise vorhandenen Administrationsbereich eine Benutzer- und Rechteliste zu führen. Da
die in Indico genutzte ZODB, nicht sehr verbreitet ist, ist insbesondere darauf zu achten, ob zur
Nutzung des Frameworks eine separate Datenbank benötigt wird.

\item[Sessions] \hfill \\
Sowohl beim Login eines Benutzers als auch bei der Nutzung diverser Funktionen von Indico müssen
Daten zwischengespeichert werden. Dazu ist es praktisch, wenn das Framework bereits ein
Sessionsystem bietet.

\item[Caching] \hfill \\
Insbesondere bei häufig aufgerufenen Seiten und komplexen Datenbankzugriffen kann die Performance
einer Anwendung massiv gesteigert werden, indem entweder ganze Seiten oder Teile der zur Generierung
selbiger notwendigen Daten in einem Cache abgelegt werden. Ein Framework mit einer integrierten
Caching-Lösung kann dabei behilflich sein, indem es beispielsweise bestimmte Seiten automatisch
cacht und dabei Parameter und evtl. vorhandene Berechtigungen des Benutzers beachtet.

\item[Integrierbarkeit] \hfill \\
Bei Indico handelt es sich um eine komplexe Anwendung mit vielen Funktionen, sodass es nicht
praktikabel ist, wenn der gesamte Code geändert oder in großen Teilen neu geschrieben werden muss.
Daher ist bei dem neuen Framework darauf zu achten, dass es möglichst leicht in eine bestehende
Anwendung integriert werden kann.

\item[Erweiterbarkeit] \hfill \\
Unter Umständen ist es notwendig, das Framework durch zusätzliche Funktionalität zu erweitern oder
bestehende Funktionen anzupassen. Dies kann über eine Plugin-API, durch Subclassing oder aber durch
Forking des Frameworks geschehen. Letzteres bedeutet, den Code des Frameworks selbst zu verändern
und regelmäßig auf den aktuellen Stand der offiziellen Versionen zu bringen.

\item[Sonstige Features] \hfill \\
Viele Frameworks haben neben den üblichen Features zusätzliche Funktionen, die in anderen Frameworks
nicht vorhanden sind. Je nach Feature können diese für Indico nützlich sein.

\item[Performance] \hfill \\
Die unterschiedlichen Frameworks sind bei der Abarbeitung von Requests möglicherweise
unterschiedlich schnell. Dies ist darauf zurückzuführen, dass komplexere Frameworks mehr Dinge
implizit bei jedem Request tun, während dieselbe Funktionalität in einem kompakten Framework
entweder überhaupt nicht enthalten ist oder optional ist. Allerdings spielt dieser Performanceaspekt
im Rahmen dieser Arbeit nur eine untergeordnete Rolle, da Indico zwar große Datenmengen enthält und
dabei performant sein muss, jedoch der Overhead des Frameworks verglichen mit Datenbankzugriffen
minimal ist und somit für den Endbenutzer in der Regel nicht spürbar ist.

\item[Dokumentation] \hfill \\
Da mehrere Entwickler mit dem Framework arbeiten müssen und starke Fluktuation herrscht, da oftmals
Studenten für 3 bis 12 Monate an Indico arbeiten, ist eine gute Dokumentation wichtig, da es nicht
sehr produktiv ist, wenn man erst den Quellcode des Frameworks lesen und verstehen muss, um es
benutzen zu können. Insbesondere ist es hilfreich, wenn die Dokumentation Beispielcode enthält und
\emph{Best Practices} beschreibt, statt nur die APIs zu dokumentieren.

\item[Lizenz] \hfill \\
Die meisten Frameworks sind unter einer OpenSource-Lizenz verfügbar.
Da Indico unter der
GNU~GPL\footnote{\href{http://www.gnu.org/licenses/gpl-3.0.txt}{http://www.gnu.org/licenses/gpl-3.0.txt}}
steht, ist auf Kompatibilität mit dieser Lizenz zu achten. \autoref{img.floss-license-slide.png}
bietet einen kurzen Überblick über die verbreitetsten Open Source-Lizenzen und zeigt die
Kompatibilität: \enquote{To see if software can be combined, just start at their respective
licenses, and find a common box you can reach following the arrows.} \citep{osslic} \\
Da Indico unter der GPLv3 lizenziert ist, sind fast alle Open Source-Lizenzen außer der \emph{Affero
GPL}\footnote{GPL mit der Erweiterung, dass auch bei einem Netzwerkzugriff auf das laufende Programm
eine \enquote{Verbreitung} stattfindet und der Quellcode zugänglich gemacht werden muss.}
kompatibel. Die einzige Ausnahme wären Frameworks, die ausschließlich unter der GPLv2 lizenziert
sind. Da die GPL jedoch bei Frameworks selten genutzt wird und insbesondere die
\emph{GPLv2-only}-Option allgemein nicht sehr verbreitet ist, ist dieses Risiko eher gering.
\img{floss-license-slide.png}{400px}{Kompatibilitätsdiagramm div. Open
Source-Lizenzen \citep{osslic}}{OpenSource-Lizenz-Kompatibilität}
\end{description}


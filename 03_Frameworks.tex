\chapter{Python-Webframeworks}

\section{Vergleichskriterien}
Um die Frameworks miteinander vergleichen zu können, müssen einige Kriterien festgelegt werden,
anhand derer sich alle Frameworks messen lassen.

\begin{description}
\item[Lizenz] \hfill \\
Die meisten Frameworks sind unter einer OpenSource-Lizenz verfügbar.
Da Indico unter der
GNU~GPL\footnote{\href{http://www.gnu.org/licenses/gpl-3.0.txt}{http://www.gnu.org/licenses/gpl-3.0.txt}}
steht, ist auf Kompatibilität mit dieser Lizenz zu achten. \autoref{img.floss-license-slide.png}
bietet einen kurzen Überblick über die verbreitetsten Open Source-Lizenzen und zeigt die
Kompatibilität: \enquote{To see if software can be combined, just start at their respective
licenses, and find a common box you can reach following the arrows.} \citep{osslic} \\
Da Indico unter der GPLv3 lizenziert ist, sind fast alle Open Source-Lizenzen außer der \emph{Affero
GPL}\footnote{GPL mit der Erweiterung, dass auch bei einem Netzwerkzugriff auf das laufende Programm
eine \enquote{Verbreitung} stattfindet und der Quellcode zugänglich gemacht werden muss.}
kompatibel. Die einzige Ausnahme wären Frameworks, die ausschließlich unter der GPLv2 lizenziert
sind. Da die GPL jedoch bei Frameworks selten genutzt wird und insbesondere die
\emph{GPLv2-only}-Option allgemein nicht sehr verbreitet ist, ist dieses Risiko eher gering.
\img{floss-license-slide.png}{400px}{Kompatibilitätsdiagramm div. Open
Source-Lizenzen \citep{osslic}}{OpenSource-Lizenz-Kompatibilität}
\end{description}


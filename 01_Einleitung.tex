\chapter{Einleitung}
\pagenumbering{arabic}

Ein Webframework ist in einer modernen Webanwendung schon fast Pflicht. Während vor einigen
Jahren noch Technologien wie \emph{CGI}\footnote{Common Gateway Interface -
\href{http://www.ietf.org/rfc/rfc3875}{RFC 3875}\citep{rfc3875}} oder die Nutzung von PHP in seiner
einfachsten Form, d.h. eine Datei pro Seite, die sowohl HTML als auch Geschäftslogik enthält, weit
verbreitet waren, so setzen inzwischen immer mehr Entwickler, sowohl von komplexen Webanwendungen als
auch von einfachen Websites mit wenigen dynamischen Elementen, auf ein Framework. Dabei stellt sich
natürlich die Frage, welche Vorteile ein Framework bietet - schließlich macht es keinen Sinn, ein
Framework zu nutzen, nur weil Frameworks \enquote{in} sind.

Ein gutes Framework ermöglicht es dem Entwickler, sich vollständig auf die Anwendungslogik zu
konzentrieren, während das Framework alle Aufgaben übernimmt, die in der Regel für jede Anwendung
identisch sind, wenn man von Konfigurationseinstellungen absieht. Beispiele für solche Aufgaben sind
Sessions, Templates oder der Aufbau einer Datenbankverbindung - keine dieser Bereiche benötigt
normalerweise anwendungsspezifischen Code sondern ausschließlich Konfigurationsdaten wie z.B. die
Logindaten für die Datenbank oder die Lebensdauer einer Session.

Ein weiterer Vorteil der meisten Frameworks ist die Entkopplung von URLs und Dateisystem.
Während in den zuvor erwähnten CGI- oder PHP-Lösungen der Pfad in der URL so gut wie immer einem
Pfad im Dateisystem entspricht, erlauben Frameworks normalerweise ein Mapping von URLs
(einschließlich dynamischer Elemente) zu Funktionen oder Klassen, sodass der Code der Anwendung
unabhängig von der Struktur der URLs sauber strukturiert werden kann.


\section{Zielsetzung}

Die \emph{Indico}-Software nutzt WSGI\footnote{\href{http://www.python.org/dev/peps/pep-0333/}{Web
Server Gateway Interface} \citep{wsgi}} zur Kommunikation mit dem Webserver. Da jedoch ursprünglich
das inzwischen veraltete \emph{mod\_python} eingesetzt wurde, und der für die Implementierung von
WSGI zuständige Entwickler darauf aufbauenden Code nicht verändern wollte, enthält Indico noch immer
eine Kompatibilitätsschicht, die die von \emph{mod\_python} bereitgestellten APIs emuliert. Diese
APIs sind weder entwicklerfreundlich noch unterstützen sie performancerelevante Funktionen wie das
Übergeben eines Dateideskriptors an den Webserver, ohne den kompletten
Dateiinhalt in den Speicher einzulesen. Bei der Migration zu einem modernen Framework sollte diese
Kompatibilitätsschicht vollständig entfernt werden und der Code, der sie nutzt, einem Refactoring
unterzogen werden.

Vor der Auswahl eines neuen Frameworks muss zuerst analysiert werden, welche Funktionalität das
bestehende Framework zur Verfügung stellt und welche Probleme existieren, damit beim neuen
Framework darauf geachtet werden kann, dass diese Probleme dort nicht ebenfalls vorhanden sind.
Falls das Framework eine eigene Datenbankschicht enthält muss geprüft werden, wie weit diese mit
der in Indico genutzten Objektdatenbank kompatibel ist.

Nachdem ein Framework ausgewählt wurde, muss es zunächst in Indico integriert werden. Danach muss
die Anwendung wieder in einen vollständig lauffähigen Zustand gebracht werden, um eventuelle
Konflikte zwischen dem Framework und bestehendem Code zu beheben. Ab diesem Zeitpunkt kann
analysiert werden, welche Elemente des alten Frameworks neben dem WSGI-Kern ebenfalls durch das neue
Framework ersetzt werden können bzw. ob es Teile der Anwendung gibt, in denen es sinnvoll ist,
bereits vorhandenen Code zu nutzen statt sie durch eine entsprechende Lösung des neuen Frameworks zu
ersetzen.


\section{Aufbau der Arbeit}

Im Einleitungskapitel wird auf die Aufgabenstellung eingegangen und die groben Schritte zum
Ziel beschrieben. Dann wird kurz auf die Firma eingegangen, an der das Projekt durchgeführt
wurde.
Ebenfalls wird die Indico-Software, die im Rahmen dieser Arbeit um ein Framework erweitert wird,
vorgestellt, um einen Überblick über ihre Funktion zu geben.

Im Grundlagenkapitel werden die genutzten Technologien und ihre Besonderheiten vorgestellt. Nach
einem Überblick über die Programmiersprache Python werden ihre Besonderheiten kurz
vorgestellt und das WSGI-Interface näher betrachtet.

Auf die Funktionen des bestehenden Frameworks wird im dritten Kapitel eingegangen. Desweiteren
werden dort die in Frage kommenden Frameworks vorgestellt und anhand verschiedener Kriterien
analysiert, sodass die Entscheidung für ein Framework später gut begründet werden kann.

Im vierten Kapitel wird die Entscheidung für das Flask-Microframework begründet, nachdem die
verschiedenen Migrationspfade erläutert und die jeweiligen Vor- und Nachteile diskutiert wurden.

Die wichtigsten Schritte der eigentlichen Migration werden im fünften Kapitel näher beschrieben. Zum
Schluss wird auf die dabei aufgetretenen Probleme eingegangen und ein Ausblick auf mögliche
Erweiterungen gegeben.

Zusätzlich zur Frameworkmigration soll das Benutzerdashboard durch automatische Vorschläge zu
Kategorien und Events erweitert werden. Lösungsansätze zu diesem Teilprojekt werden im sechsten
Kapitel vorgestellt.

Kapitel sieben schließt diese Arbeit mit einem Rückblick ab.


\section{CERN}

Beim CERN\footnote{Conseil Européen pour la Recherche Nucléaire - Europäische Organisation für
Kernforschung} handelt es sich um eine internationale Forschungseinrichtung im Schweizer Kanton
Genf. Der Hauptaufgabenbereich liegt in der Erforschung von Grundlagen der Teilchenphysik, wobei der
weltgrößte Teilchenbeschleuniger \emph{LHC}\footnote{Large Hadron Collider, der
Teilchenbeschleuniger am CERN} zum Einsatz kommt.

Neben der physikalischen Forschung wird am CERN auch Software entwickelt, die zwar in erster
Linie zur internen Nutzung dient, jedoch oftmals auch in Form von \emph{Open Source} der
Allgemeinheit kostenfrei zur Verfügung gestellt wird. Diese Software erstreckt sich über viele
Bereiche der IT, so werden am CERN z.B. Business-Software für den Human-Resources-Bereich,
Grid-Lösungen für die verteilte Datenverarbeitung, Steuersysteme für den Teilchenbeschleuniger und
Webanwendungen für die Archivierung von Medien entwickelt.

Die Abteilung \emph{IT-CIS-AVC}\footnote{Collaboration \& Information Services - AudioVisual and
Collaborative Services} ist zuständig für die Verwaltung und Einrichtung von Videokonferenzsystemen,
Aufzeichnung und Onlinestreaming von Vorträgen, Meetings und Konferenzen und die Software zu deren
Planung. Ebenfalls im Zuständigkeitsbereich der Abteilung sind die Informationsbildschirme, die in
verschiedenen Gebäuden des CERN installiert sind und aktuelle Informationen zum nächstgelegenen
Konferenzraum oder aber den Status des \emph{Large Hadron Colliders} anzeigen.


\section{Indico}

\subsection{Überblick}
Indico\footnote{Integrated Digital Conference - \href{http://indico-software.org/}{http://indico-software.org/}}
ist eine von der Abteilung \emph{IT-CIS-AVC} am CERN entwickelte Webapplikation, die zum Planen und
Organisieren von Events dient. Dabei kann es sich sowohl um einfache Vorträge handeln, aber auch um
Meetings und mehrtägige Konferenzen mit mehreren Sessions und Beiträgen. Zur Integration in eine
bestehende Benutzerinfrastruktur unterstützt die Software außerdem externe
Benutzerauthentifizierung. Insbesondere für Meetings und Konferenzen wichtig sind Features wie das
\emph{Paper Reviewing}\footnote{Im Rahmen eines \emph{Call for Papers} oder \emph{Call for
Abstracts} bei einer Konferenz}, elektronische Sitzungsprotokolle und ein Archiv für die in einer
Konferenz benutzten Materialien. \citep{indico}

Diese Featureliste ist jedoch schon lange nicht mehr aktuell, da im Laufe der Zeit immer mehr
Funktionen hinzugefügt wurden. Inzwischen enthält Indico u.a. ein Reservierungssystem für
Konferenzzimmer, sodass beim Erstellen eines Events sichergestellt werden kann, dass der gewünschte
Raum auch verfügbar ist und nicht gerade anderweitig benutzt wird. Diverse Videokonferenzsysteme
sind ebenfalls in Indico integriert, sodass diese - sofern sie im jeweiligen Raum verfügbar sind -
vollautomatisch eingerichtet und aktiviert werden können. Ein neueres Feature, welches gerade in
Entwicklung ist, ermöglicht es den Organisatoren einer Konferenz, Tickets mit QR-Code zu generieren,
die vor Ort dann mithilfe einer Smartphone-App eingescannt und auf Gültigkeit geprüft werden können.

\subsection{Aufbau}
Kategorien und Events in Indico sind in einer Baumstruktur organisiert: Auf der Rootebene finden
sich in der Regel ausschließlich Kategorien, die jeweils entweder weitere Kategorien oder beliebige
Events (Vorträge, Meetings und Konferenzen) enthalten können. Ein solches Event wiederum kann
diverse Elemente enthalten, wobei deren Zusammensetzung vom Typ des Events abhängt. Ein Vortrag oder
Meeting enthält in Indico z.B. grundsätzlich kein Registrierungsformular.

Die folgende Auflistung beinhaltet die wichtigsten Bestandteile von Events:
\begin{itemize}
\item \emph{Call for Abstracts}
\item Chaträume
\item Evaluation
\item Materialien
\item \emph{Paper Reviewing}
\item Registrierungsformular (inkl. Integration von Payment-Gateways)
\item Videokonferenzen
\item Zeitplan
\end{itemize}

\subsection{Architektur}
Der serverseitige Code von Indico ist komplett in Python 2.6 geschrieben und nutzt die
ZODB\footnote{Zope Object Database, \href{http://www.zodb.org}{http://www.zodb.org}} als Datenbank.
Der Code ist in einer mehrschichtigen Architektur organisiert: Die Anfrage des Benutzers wird vom
Webserver über WSGI\footnote{\href{http://www.python.org/dev/peps/pep-0333/}{Web Server Gateway
Interface}} an die Anwendung weitergegeben, in der die \emph{Request Handler (RH)}-Ebene die
Anwendungslogik ausführt. Diese baut die Datenbankverbindung auf, prüft die Zugangsberechtigung des
Benutzers, verarbeitet vom Client übermittelte Daten und führt letztendlich auch die gewünschte
Aktion aus. Um eine dynamische HTML-Seite auszugeben, nutzt der \emph{Request Handler} die
\emph{Web Pages (WP)}-Ebene. Dort werden diverse Aktionen ausgeführt, die die Anzeige der Daten
beeinflussen - insbesondere werden dort die benötigten JavaScript- bzw. CSS-Packages mithilfe der
\emph{webassets}-Bibliothek geladen. Die eigentliche Erzeugung der HTML-Datei übernimmt die
\emph{Components (W)}-Ebene. Bei dieser Ebene handelt es sich um ein Überbleibsel aus der
Anfangszeit von Indico, als Templates noch keinerlei Logik enthalten konnten. Jede Klasse in dieser
Ebene entspricht einem Template. In der Regel werden dort nur die aus der WP-Ebene übergebenen Daten
an das Template weitergereicht.

\autoref{img.indico-layers.png} zeigt die ursprüngliche Struktur, die sich jedoch im Laufe der Zeit
verändert hat: Statt des obsoleten
\emph{mod\_python}\footnote{\href{http://www.modpython.org}{http://www.modpython.org}} wird
inzwischen WSGI verwendet.

\img{indico-layers.png}{200px}{Die Architektur von Indico \citep{indicoarch}}{Indico-Architektur}

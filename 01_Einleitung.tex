\chapter{Einleitung}
\pagenumbering{arabic}

Ein Webframework ist aus einer modernen Webanwendung nicht mehr wegzudenken. Während vor einigen
Jahren noch Technologien wie \emph{CGI}\footnote{Common Gateway Interface -
\href{http://www.ietf.org/rfc/rfc3875}{RFC 3875}\citep{rfc3875}} oder die Nutzung von PHP in seiner
einfachsten Form, d.h. eine Datei pro Seite, die sowohl HTML als auch Geschäftslogik enthält, weit
verbreitet waren, so setzen inzwischen immer mehr Entwickler sowohl von komplexen Webanwendungen als
auch von einfachen Websites mit wenigen dynamischen Elementen auf ein Framework. Dabei stellt sich
natürlich die Frage, welche Vorteile ein Framework bietet - schließlich macht es keinen Sinn, ein
Framework zu nutzen, nur weil Frameworks \enquote{in} sind.

Ein gutes Framework ermöglicht es dem Entwickler, sich vollständig auf die Anwendungslogik zu
konzentrieren, während das Framework alle Aufgaben übernimmt, die in der Regel für jede Anwendung
identisch sind, wenn man von Konfigurationseinstellungen absieht. Beispiele für solche Aufgaben sind
Sessions, Templates oder der Aufbau einer Datenbankverbindung - keine dieser Bereiche benötigt
normalerweise anwendungsspezifischen Code sondern ausschließlich Konfigurationsdaten wie z.B. die
Logindaten für die Datenbank oder die Lebensdauer einer Session.

Ein weiterer Vorteil der meisten Frameworks ist die Entkopplung von URLs und dem Dateisystem.
Während in den zuvor erwähnten CGI- oder PHP-Lösungen der Pfad in der URL so gut wie immer einem
Pfad im Dateisystem entspricht, erlauben Frameworks normalerweise ein Mapping von URLs
(einschließlich dynamischer Elemente) zu Funktionen oder Klassen, sodass der Code der Anwendung
unabhängig von der Struktur der URLs sauber strukturiert werden kann.


\section{Zielsetzung}

Die \emph{Indico}-Software nutzt WSGI\footnote{\href{http://www.python.org/dev/peps/pep-0333/}{Web
Server Gateway Interface} \citep{wsgi}} zur Kommunikation mit dem Webserver. Da jedoch ursprünglich
das inzwischen veraltete \emph{mod\_python} eingesetzt wurde, und der für die Implementierung von
WSGI zuständige Entwickler darauf aufbauenden Code nicht verändern wollte, enthält Indico noch immer
eine Kompatibilitätsschickt, die die von \emph{mod\_python} bereitgestellten APIs emuliert. Diese
APIs sind weder entwicklerfreundlich noch unterstützen sie performancerelevante Funktionen wie
beispielsweise das Übergeben eines Dateideskriptors an den Webserver, ohne den kompletten
Dateiinhalt in den Speicher einzulesen. Bei der Migration zu einem modernen Framework bietet es sich
an, diese Schicht vollständig zu entfernen und den Code, der sie nutzt, einem Refactoring zu
unterziehen.

Vor der Auswahl eines neuen Frameworks muss zuerst analysiert werden, welche Funktionalität das
bestehende Framework zur Verfügung stellt und welche Probleme es dort gibt, damit beim neuen
Framework darauf geachtet werden kann, dass diese Probleme dort nicht ebenfalls vorhanden sind.
Sofern das Framework eine eigene Datenbankschicht enthält muss geprüft werden, inwiefern diese mit
der in Indico genutzten Objektdatenbank kompatibel ist.

Nachdem ein Framework ausgewählt wurde, muss dieses in die Software integriert werden. Der erste
Schritt danach ist, die Anwendung wieder in einen vollständig lauffähigen Zustand zu bringen und
Konflikte zwischen dem Framework und bestehendem Code zu beheben. Ab diesem Zeitpunkt kann
analysiert werden, welche Elemente des alten Frameworks neben dem WSGI-Kern ebenfalls durch das neue
Framework ersetzt werden können bzw. ob es Teile der Anwendung gibt, in denen es sinnvoll ist,
bereits vorhandenen Code zu nutzen statt sie durch eine entsprechende Lösung des neuen Frameworks zu
ersetzen.


\section{Aufbau der Arbeit}

Im Einleitungskapitel wird kurz auf die Aufgabenstellung eingegangen und die groben Schritte zum
Ziel beschrieben. Desweiteren wird kurz auf die Firma eingegangen, an der das Projekt durchgeführt
wurde.
Ebenfalls wird die Software, die im Rahmen dieser Arbeit modifiziert wurde, kurz vorgestellt, sodass
man sich einen Überblick darüber verschaffen kann.

Im Grundlagenkapitel werden die genutzen Technologien und ihre Besonderheiten näher betrachtet und
erläutert. Nach einem kurzen Überblick über die Programmiersprache Python werden ihre Besonderheiten
kurz vorgestellt und das WSGI-Interface näher betrachtet.

Auf die Funktionen des bestehenden Frameworks wird im dritten Kapitel kurz eingegangen. Desweiteren
werden dort die in Frage kommenden Frameworks vorgestellt, anhand von verschiedenen Gesichtspunkten
analysiert und mit dem aktuellen Framework verglichen.

\todotext{Aufbau fertigschreiben}


\section{CERN}

Beim CERN (Europäische Organisation für Kernforschung), handelt es sich um eine
Forschungseinrichtung im Schweizer Kanton Genf. Der Hauptaufgabenbereich liegt in der Erforschung
von Grundlagen der Physik, wobei der weltgrößte Teilchenbeschleuniger \emph{LHC} zum Einsatz kommt.

Neben der physikalischen Forschung wird am CERN auch Software entwickelt, die zwar in erster Linie
zur internen Nutzung dient, jedoch oftmals auch als \emph{Open Source} der Allgemeinheit zur Verfügung
gestellt wird. Diese Software erstreckt sich über viele Bereiche der IT, so werden am CERN
beispielsweise Steuersysteme für den Teilchenbeschleuniger, Business-Software für die Verwaltung des
Personals, Grid-Lösungen für die verteilte Datenspeicherung und Webanwendungen für die Archivierung
und Indizierung von Medien entwickelt.

Die Abteilung \emph{IT-CIS-AVC}\footnote{Collaboration \& Information Services - AudioVisual and
Collaborative Services} am CERN ist zuständig für die Verwaltung und Einrichtung von video- und
telefonbasierten Konferenzsystemen, Aufzeichnung und Onlinestreaming von Vorträgen, Meetings und
Konferenzen und die Software zu deren Planung. Ebenfalls im Zuständigkeitsbereich der Abteilung sind
die Informationsbildschirme, die in verschiedenen Gebäuden des CERN aufgebaut sind und beispielsweise
den Status des \emph{LHC}\footnote{Large Hadron Collider, der Teilchenbeschleuniger am CERN} und
für die Mitarbeiter relevante Neuigkeiten anzeigen.


\section{Indico}

\subsection{Überblick}
Indico\footnote{Integrated Digital Conference - \href{http://indico-software.org/}{http://indico-software.org/}}
ist eine von der Abteilung \emph{IT-CIS-AVC} am CERN entwickelte Webapplikation, die zum Planen und
Organisieren von Events dient. Diese Events können sowohl einfache Vorträge als auch Meetings und
Konferenzen mit mehreren Sessions und Beiträgen sein. Die Applikation unterstützt außerdem externe
Benutzerauthentifizierung, \emph{Paper Reviewing}\footnote{Im Rahmen eines \emph{Call for Papers}
oder \emph{Call for Abstracts} bei einer Konferenz}, elektronische Sitzungsprotokolle und ein Archiv
für die in einer Konferenz benutzten Materialien. \citep{indico}

Im Laufe der Zeit wurden immer mehr Funktionen hinzugefügt, sodass diese Featureliste schon lange
nicht mehr aktuell ist. Inzwischen enthält Indico u.a. ein Buchungs- und Reservierungssystem für
Konferenzzimmer, sodass beim Erstellen eines Events sichergestellt werden kann, dass der gewünschte
Raum auch verfügbar ist und nicht gerade anderweitig benutzt wird. Ebenfalls in Indico integriert
sind die am CERN genutzten Audio- und Videokonferenzsysteme, sodass diese vollautomatisch
eingerichtet und aktiviert werden können, sofern diese Systeme im gewählten Raum verfügbar sind.
Eines der neuesten Features ist die Möglichkeit, sich die Konferenzzimmer auf einer
\href{http://maps.google.com/}{\emph{Google Maps}}-basierten Karte anzeigen zu lassen und anhand der
Nähe zu einem bestimmten Gebäude auszuwählen.

\subsection{Aufbau}
Events in Indico sind Teil einer Baumstruktur: Auf der Top-Level-Ebene finden sich ausschließlich
Kategorien, die jeweils wieder Kategorien oder beliebige Events (Konferenzen, Meetings und
Vorträge) enthalten können. Ein Event wiederum kann diverse Elemente enthalten; welche das sind,
hängt vom Typ des Events ab, so kann z.B. nur eine Konferenz ein Registrierungsformular
besitzen.

\newpage
Die folgende Auflistung beinhaltet die wichtigsten Bestandteile von Events:
\begin{itemize}
\item Audio-/Videokonferenzen
\item \emph{Call for Abstracts}
\item Chaträume
\item Evaluation
\item Materialien
\item \emph{Paper Reviewing}
\item Registrierung
\item Zeitplan
\end{itemize}

\subsection{Architektur}
Der serverseitige Code von Indico ist in Python geschrieben und nutzt
ZODB\footnote{\href{http://www.zodb.org}{http://www.zodb.org}} als Datenbank. Der Code ist in einer
Multi-Tier-Architektur organisiert: Die Anfrage des Webservers wird über
WSGI\footnote{\href{http://www.python.org/dev/peps/pep-0333/}{Web Server Gateway Interface}} an die
Applikation weitergegeben, in der die \emph{Request Handler (RH)}-Ebene die Geschäftlogik ausführt.
Sie prüft die Zugangsberechtigung des Benutzers, verarbeitet GET- bzw. POST-Daten, baut die
Datenbankverbindung auf und führt letztendlich auch die gewünschte Aktion aus. Um eine HTML-Seite
auszugeben, nutzt der \emph{Request Handler} die \emph{Web Pages (WP)}-Ebene. Dort werden diverse
Aktionen ausgeführt, welche die Anzeige der Daten beeinflussen - zum einen werden die benötigten
JavaScript- bzw. CSS-Packages geladen, zum anderen werden, falls vorhanden, Tabs als aktiv markiert
oder Daten sortiert. Die eigentliche Erzeugung der HTML-Datei übernimmt die \emph{Components
(W)}-Ebene. Jede Klasse dieser Ebene repräsentiert ein Template mit demselbem Namen. In der Regel
werden in dieser Ebene nur die aus der WP-Ebene übergebenen Daten an das Template weitergereicht und
falls nötig aufbereitet.

Wie \autoref{img.indico-layers.png} verdeutlicht, hat sich Indico im Laufe der Zeit verändert: Statt
des obsoleten \emph{mod\_python}\footnote{\href{http://www.modpython.org}{http://www.modpython.org}}
wird inzwischen WSGI verwendet.

\img{indico-layers.png}{200px}{Die Architektur von Indico \citep{indicoarch}}{Indico-Architektur}

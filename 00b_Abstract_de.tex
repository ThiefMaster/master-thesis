\chapter*{Zusammenfassung}
\thispagestyle{empty}

Im Rahmen dieser Master Thesis soll die am CERN entwickelte \emph{Indico}-Software um ein frei
verfügbares Python-Webframework erweitert werden. Ein solches Framework erlaubt es, effizienter zu
entwickeln, als es mit dem derzeit vorhandenen Framework, das über viele Jahre hinweg in-house
entwickelt wurde, möglich ist.
Bei Indico handelt es sich um eine Webapplikation zur Planung und Verwaltung von Meetings,
Konferenzen und ähnlichen Events, wobei auch die Verwaltung und Reservierung von Konferenzräumen
integriert ist.

Zu Beginn werden die genutzten Technologien Python und WSGI vorgestellt. Danach werden sowohl
das derzeitige, speziell auf Indico zugeschnittene, Mini-Framework als auch einige andere
Frameworks vorgestellt und mithilfe verschiedener Kriterien analysiert. Aufbauend auf dieser Analyse
werden die Vor- und Nachteile der Migration zu einem dieser Frameworks untersucht und anhand dieser
ein Framework ausgewählt. Aufbauend auf dem gewählten Framework werden dann Teile von Indico
migriert bzw. angepasst.

Die durch diese Thesis erarbeitete Lösung soll dabei eine Grundlage für die Nutzung von
verbreitetem, gut dokumentiertem \emph{Third Party}-Code und ein wartbares, entwicklerfreundliches
System bieten, welches gleichzeitig auch benutzer- und suchmaschinenfreundlicher als die aktuelle
Version ist.

Ein weiteres Ziel neben der Modernisierung des Frameworks ist die Entwicklung und Integration einer
einfachen \emph{Recommendation Engine} für Kategorien und/oder Events, sodass Benutzer auf den
ersten Blick sehen, was sie evtl. ebenfalls interessieren könnte.

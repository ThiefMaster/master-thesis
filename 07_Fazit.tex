\chapter{Fazit}

Im Rahmen dieser Masterarbeit am CERN wurden nicht nur die beiden immer weiter verbreiteten
Python-Webframeworks \emph{Django} und \emph{Flask} unter die Lupe genommen, sondern auch den
frameworkähnlichen Code von Indico untersucht. Bei allen Frameworks haben sich sowohl Stärken als
auch Schwächen gezeigt, wobei klar erkennbar wurde, dass eine Schwäche in einer anderen Anwendung
durchaus eine Stärke sein kann und umgekehrt. So ist beispielsweise ist das ORM-System von Django
für eine Anwendung wie Indico nicht von Nutzen, sofern man sie nicht von Grund auf neu entwickelt,
aber wenn man eine weniger umfangreiche Webapplikation von Anfang an auf Django basiert, ist ein
integriertes Datenbankframework sehr nützlich.

Alle mit Flask verbundenen Änderungen einschließlich der sauberen URLs sind inzwischen im
Hauptentwicklungszweig von Indico integriert und dokumentiert, sodass andere Indico-Entwickler von
den Neuerungen profitieren können. Im Laufe der Entwicklung hat sich gezeigt, dass es sehr viele
Bereiche in Indico gibt, die durch Flask-Features verbessert werden könnten, aber um den Übergang
auch für andere Entwickler möglichst einfach zu gestalten, noch nicht umgesetzt wurden. Dies ist
leicht zu erklären, indem man das Entwicklungsmodell bei der Nutzung von Git betrachtet. Jedes
Feature wird in einer separaten Branch des Codes entwickelt. Dies kann prinzipiell jederzeit
aktualisiert werden, sodass alle eigenen Änderungen auf der aktuellsten \emph{master}-Version
basieren. Bei größeren Änderungen wie einer Frameworkmigration sind Konflikte bei diesen Updates
jedoch extrem wahrscheinlich und müssen jeweils manuell behoben werden. Daher war es wünschenswert,
die Migration möglichst schnell soweit abzuschließen, dass der Code relativ fehlerfrei in den
\emph{master}-Branch integriert werden konnte.

Ebenfalls gezeigt hat sich, wie hilfreich eine vollständige Testabdeckung ist. Eine große Anwendung
wie Indico manuell zu testen ist sehr aufwändig.

Langfristig wäre es aus Entwicklersicht natürlich angenehm, wenn alle Bestandteile von Indico
Flask-Features nutzen würden und jeglicher alte Code entfernt würde. Dies ist jedoch aufgrund des
enormen Aufwands, der damit verbunden ist, sehr unwahrscheinlich.

Die Entwicklung der \emph{Recommendation Engine} für Kategorien und die Planung einer solchen für
Events hat gezeigt, dass ein gerade aus Onlineshop eigentlich als alltäglich erscheinendes Feature
alles andere als einfach zu implementieren ist und die Tatsache, dass Indico kein Onlineshop sondern
ein \emph{Conference Management System} ist, die Aufgabe nicht einfacher macht. Da das existierende
Dashboard jedoch sehr positiv aufgenommen wurde ist damit zu rechnen, dass auch ein relativ simples
Vorschlagssystem für Kategorien gut aufgenommen wird.
